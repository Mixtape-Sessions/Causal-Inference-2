\documentclass{beamer}

\input{preamble.tex}
\usepackage{breqn} % Breaks lines

\usepackage{amsmath}
\usepackage{mathtools}

\usepackage{pdfpages} % \includepdf

\usepackage{listings} % R code
\usepackage{verbatim} % verbatim

% Video stuff
\usepackage{media9}

% packages for bibs and cites
\usepackage{natbib}
\usepackage{har2nat}
\newcommand{\possessivecite}[1]{\citeauthor{#1}'s \citeyearpar{#1}}
\usepackage{breakcites}
\usepackage{alltt}

% Setup math operators
\DeclareMathOperator{\E}{E} \DeclareMathOperator{\tr}{tr} \DeclareMathOperator{\se}{se} \DeclareMathOperator{\I}{I} \DeclareMathOperator{\sign}{sign} \DeclareMathOperator{\supp}{supp} \DeclareMathOperator{\plim}{plim}
\DeclareMathOperator*{\dlim}{\mathnormal{d}\mkern2mu-lim}
\newcommand\independent{\protect\mathpalette{\protect\independenT}{\perp}}
   \def\independenT#1#2{\mathrel{\rlap{$#1#2$}\mkern2mu{#1#2}}}
\newcommand*\colvec[1]{\begin{pmatrix}#1\end{pmatrix}}

\newcommand{\myurlshort}[2]{\href{#1}{\textcolor{gray}{\textsf{#2}}}}


\begin{document}

\imageframe{./lecture_includes/mixtape_did_cover.png}


% ---- Content ----

\section{Introduction}

\subsection{Managing expectations}


\begin{frame}{Introduction}

\begin{itemize}
\item Welcome to Mixtape Sessions workshop on difference-in-differences and synthetic control (``Causal Inference II'')
\item 8:00am to 5:00pm CST, 15 min breaks every hour, 1 hour lunch at noon CST
\item Lecture, discussion, exercises, application
\end{itemize}

\end{frame}


\begin{frame}{What my pedagogy is like}

\begin{itemize}
\item Long days that don't feel long because it's high energy, with regular breaks including lunch
\item Move between the econometrics, history of thought, videos, applications, code, spreadsheets, exercises
\item Ask questions at any point; I'll do my best to answer them
\end{itemize}

\end{frame}


\begin{frame}{Class goals}
Pedagogical goal is to break down the procedures into plain English, rebuilding it into something you can and want to use, but also:

  \begin{enumerate}
    \item \textbf{Confidence}: You will feel like you have a good enough understanding of diff-in-diff and synthetic control, both in its basics and some more contemporary issues, so that by the end of the week it a very intuitive, friendly, and useful tool
    \item \textbf{Comprehension}: You will have learned a lot both conceptually and in the specifics, particularly with regards to issues around identification and estimation in the diff-in-diff and synth context
    \item \textbf{Competency}: You will have more knowledge of programming syntax in Stata and R so that later you can apply this in your own work
  \end{enumerate}

\end{frame}



\begin{frame}{Day 1 outline}

Introduction to DiD basics 
	\begin{itemize}
	\item Potential outcomes review and the ATT parameter
	\item DiD equation (``four averages and three differences''), parallel trends and estimation with OLS
	\item Evaluating parallel trends with falsifications, event studies 
	\item Compositional changes, triple differences and covariates
	\end{itemize}

\end{frame}


\begin{frame}{Day 2 outline}

 Differential timing
	\begin{itemize}
	\item TWFE Pathologies in static and dynamic specifications (``event study'')
	\item Solution 1: Aggregating group-time ATTs
	\item Solution 2: Imputation estimators
	\item Solution 3: Stacked regression
	\end{itemize}

\end{frame}

\begin{frame}{Day 3 outline}

\begin{itemize}
\item Canonical synth (Abadie papers)
\item Augmented synth (Ben-Michael, et al)
\item Matrix completion with nuclear norm regularization (Athey, et al.)
\item Exercises and estimation 
\end{itemize}

\end{frame}

\subsection{Introducing difference-in-differences}



\begin{frame}{What is difference-in-differences (DiD)}

\begin{itemize}
\item DiD is a very old, relatively straightforward, intuitive research design
\item A group of units are assigned some treatment and then compared to a group of units that weren't
\item One of the most widely used quasi-experimental methods in economics and increasingly in industry
\item Mostly associated with ``big shocks'' happening in space over time
\end{itemize}


\begin{quote}
``A good way to do econometrics is to look for good natural experiments and use statistical methods that can tidy up the confounding factors that nature has not controlled for us.'' -- Daniel McFadden (Nobel Laureate recipient with Heckman 2002)
\end{quote}

\end{frame}


\begin{frame}

	\begin{figure}
	\caption{Currie, et al. (2020)}
	\includegraphics[scale=0.25]{./lecture_includes/currie_did.png}
	\end{figure}


\end{frame}













\begin{frame}{Origins of diff-in-diff}

\begin{itemize}
\item Difference-in-differences (DiD) was quietly and largely unnoticed introduced in the 19th century as a way to convince skeptics in health policy arguments
\item Dominant disease theory in 19th century was \emph{miasma} -- disease caused by smelly vapor
\item Keep in mind -- microorganisms would not be identified until much later, partly caused by poor resolution in microscopes (Freedman 2007)

\end{itemize}

\end{frame}

\begin{frame}{Miasma I: Ignaz Semmelweis and washing hands}

\begin{itemize}
\item 1840s, Vienna maternity wards had high postpartum infections in one wing compared to other wings
\item One division had doctors and trainee doctors, but another had midwives and trainee midwives
\end{itemize}

\end{frame}

\begin{frame}{Miasma I: Ignaz Semmelweis and washing hands}

\begin{itemize}
\item Ignaz Semmelweis notes the difference in 1841 when hospitals moved to ``anatomical'' training involving cadavers (Pamela Jakeila lecture notes on DiD)
\item New training happens to one but not the other and Semmelweis thinks the mortality is caused by working with cadavers
\item Proposes washing hands with chlorine in 1847 in the midwives' wing and uses a DiD design of pre and post
\end{itemize}

\end{frame}



\begin{frame}{Miasma II: John Snow and cholera}

\begin{itemize}
\item Three major waves of cholera in the early to mid 1800s in London
\item John Snow believed cholera was spread through the Thames water supply which contradicted dominant theory about ``dirty air'' transmission
\item Grand experiment: Lambeth moves its pipe between 1849 and 1854; Southwark and Vauxhall delay
\item He can evaluate the effect in three ways (one of which is DiD)
\end{itemize}

\end{frame}


\begin{frame}

	\begin{figure}
	\caption{Two water utility companies in London 1854}
	\includegraphics[scale=0.225]{./lecture_includes/lambeth.png}
	\end{figure}


\end{frame}




\begin{frame}{1) Simple cross-sectional design}

\begin{table}\centering
		\caption{Lambeth and Southwark and Vauxhall, 1854}
		\begin{center}
		\begin{tabular}{ll}
		\toprule
		\multicolumn{1}{l}{\textbf{Company}}&
		\multicolumn{1}{c}{\textbf{Cholera mortality}}\\
		\midrule
		Lambeth  & $Y=L + D$ \\
		Southwark and Vauxhall  & $Y=SV$ \\
		\bottomrule
		\end{tabular}
		\end{center}
	\end{table}

\bigskip

\begin{eqnarray*}
\widehat{\delta}_{cs} = D + (L-SV)
\end{eqnarray*}What is $L$ and $SV$?  

\end{frame}

\begin{frame}{1) Simple cross-sectional design}

\begin{table}\centering
		\caption{Lambeth and Southwark and Vauxhall, 1854}
		\begin{center}
		\begin{tabular}{ll}
		\toprule
		\multicolumn{1}{l}{\textbf{Company}}&
		\multicolumn{1}{c}{\textbf{Cholera mortality}}\\
		\midrule
		Lambeth  & $Y=L + D$ \\
		Southwark and Vauxhall  & $Y=SV$ \\
		\bottomrule
		\end{tabular}
		\end{center}
	\end{table}

\bigskip

\begin{eqnarray*}
\widehat{\delta}_{cs} = D + (L-SV)
\end{eqnarray*}This is biased if $\L \neq SV$ (selection bias). Give an example when we're pretty sure they are equal.

\end{frame}

\begin{frame}{2) Interrupted time series design}

	\begin{table}\centering
		\caption{Lambeth, 1849 and 1854}
		\begin{center}
		\begin{tabular}{lll}
		\toprule
		\multicolumn{1}{l}{\textbf{Company}}&
		\multicolumn{1}{c}{\textbf{Time}}&
		\multicolumn{1}{c}{\textbf{Cholera mortality}}\\
		\midrule
		Lambeth & 1849 & $Y=L$ \\
		& 1854 & $Y=L + (L_t + D)$ \\
		\bottomrule
		\end{tabular}
		\end{center}
	\end{table}

\begin{eqnarray*}
\widehat{\delta}_{its} = D + L_t
\end{eqnarray*}What is required for this estimator to be unbiased?


\end{frame}

\begin{frame}{3) Difference-in-differences}

\begin{table}\centering
\scriptsize
		\caption{Lambeth and Southwark and Vauxhall, 1849 and 1854}
		\begin{center}
		\begin{tabular}{lll|lc}
		\toprule
		\multicolumn{1}{l}{\textbf{Companies}}&
		\multicolumn{1}{c}{\textbf{Time}}&
		\multicolumn{1}{c}{\textbf{Outcome}}&
		\multicolumn{1}{c}{$D_1$}&
		\multicolumn{1}{c}{$D_2$}\\
		\midrule
		Lambeth & Before & $Y=L$ \\
		& After & $Y=L + L_t + D$ & $\textcolor{red}{L_t}+D$\\
		\midrule
		& & & & $D + (\textcolor{red}{L_t}- SV_t)$ \\
		\midrule
		Southwark and Vauxhall & Before & $Y=SV$ \\
		& After & $Y=SV + SV_t$ & $SV_t$\\
		\bottomrule
		\end{tabular}
		\end{center}
	\end{table}

\begin{eqnarray*}
\widehat{\delta}_{did} = D + (\textcolor{red}{L_t}- SV_t)
\end{eqnarray*}This method yields an unbiased estimate of D if $\textcolor{red}{L_t} = SV_t$

\end{frame}


\begin{frame}{Difference-in-differences and empirical crisis in labor economics}

\begin{itemize}
\item Empirical crisis in empirical labor back in the 1970s (26:31 to 32:00)  \\ \url{https://youtu.be/1soLdywFb_Q?t=1579}
\item Orley Ashenfelter graduated from Princeton in the 1970s, takes a job in Washington DC and begins studying ``job trainings programs'' where he develops the difference-in-differences design
\item Listen to Orley explain the connection he made between twoway fixed effects and difference-in-differences; it was born out of a need to explain OLS to an American bureaucrat
\\ \url{https://youtu.be/WnB3EJ8K7lg?t=126}

\end{itemize}


\end{frame}

\begin{frame}{Steps of a project}

\begin{enumerate}
\item Convert research question into causal parameter
\item Deduce beliefs needed to estimate that causal parameter with data
\item Create a calculator that will use data and estimate the causal parameter
\end{enumerate}

\bigskip

Most of us skip (1) and maybe even (2) and instead simply ``run regressions'' and cross our fingers that that coefficient is causal, but is it? And why is it?  And what is it?  Let's dig into Orley's comment a little more. 

\end{frame}

\subsection{Potential outcomes}

\begin{frame}{Equivalence}
$$Y_{ist} = \alpha_0 + \alpha_1 Treat_{is} + \alpha_2 Post_{t} + \textcolor{blue}{\delta} (Treat_{is} \times Post_t) + \varepsilon_{ist} $$

\bigskip

$$\widehat{\textcolor{blue}{\delta}} = \bigg ( \overline{y}_k^{post(k)} - \overline{y}_k^{pre(k)} \bigg ) - \bigg ( \overline{y}_U^{post(k)} - \overline{y}_U^{pre(k)} \bigg ) $$

\begin{itemize}
\item Orley claims that the OLS estimator of $\delta$ and the ``four averages and three subtractions'' are the same thing numerically
\item And they are -- they are numerically \emph{identical}
\item And under a particular assumption, they are also unbiased estimates of an aggregate causal parameter
\item But to see this we need new notation -- potential outcomes
\end{itemize}

\end{frame}





\begin{frame}{Potential outcomes notation}
	
	\begin{itemize}
	\item Let the treatment be a binary variable: $$D_{i,t} =\begin{cases} 1 \text{ if in job training program $t$} \\ 0 \text{ if not in job training program at time $t$} \end{cases}$$where $i$ indexes an individual observation, such as a person

	\end{itemize}
\end{frame}

\begin{frame}{Potential outcomes notation}
	
	\begin{itemize}

	\item Potential outcomes: $$Y_{i,t}^j =\begin{cases} 1 \text{: wages at time $t$ if trained} \\ 0 \text{: wages at time $t$ if not trained} \end{cases}$$where $j$ indexes a counterfactual state of the world

	\end{itemize}
\end{frame}



\begin{frame}{Treatment effect definitions}


	\begin{block}{Individual treatment effect}
	    The individual treatment effect,  $\delta_i$, equals $Y_i^1-Y_i^0$
	\end{block}

Missing data problem:  I don't know my own counterfactual
	
\end{frame}


\begin{frame}{Conditional Average Treatment Effects}	
	\begin{block}{Average Treatment Effect on the Treated (ATT)}
	The average treatment effect on the treatment group is equal to the average treatment effect conditional on being a treatment group member:
		\begin{eqnarray*}
		E[\delta|D=1]&=&E[Y^1-Y^0|D=1] \nonumber \\
		&=&E[Y^1|D=1]-\textcolor{red}{E[Y^0|D=1]}
		\end{eqnarray*}
	\end{block}
	
	\bigskip

This is one of the most important policy parameters, if not the most important, and coincidentally it's also the parameter you get with diff-in-diff (even with heterogeneity)

	
\end{frame}

\begin{frame}{Potential outcomes vs data}

\begin{itemize}
\item ATT is expressed in terms of potential outcomes, but we do not use potential outcomes for estimation; we use data
\item Potential outcomes are unknown and \emph{hypothetical} possibilities describing states of the world but our data are realized outcomes, or ``data'', that actually occurred
\item Potential outcomes become realized under treatment assignment $$Y_{it}=D_{it}Y_{it}^1 + (1-D_{it})Y_{it}^0$$
\item Depending on how the treatment is assigned really dictates whether correlations reveal causal effects or bias

\end{itemize}
\end{frame}

\subsection{Estimation}


\begin{frame}{DiD equation}

Orley's ``four averages and three subtractions'', or what Bacon will call the 2x2

\begin{eqnarray*}
\widehat{\delta} = \bigg ( E[Y_k|Post] - E[Y_k|Pre] \bigg ) - \bigg ( E[Y_U | Post ] - E[ Y_U | Pre] \bigg) \\
\end{eqnarray*}$k$ are the people in the job training program, $U$ are the untreated people not in the program, $Post$ is after the trainees took the class, $Pre$ is the period just before they took the class, and $E[y]$ is mean earnings. 

\bigskip

Does $\widehat{\delta}$ equal the ATT?  If so when? If not why not?

\end{frame}



\begin{frame}{Potential outcomes and the switching equation}

\begin{eqnarray*}
\widehat{\delta} &=& \bigg ( \underbrace{E[Y^1_k|Post] - E[Y^0_k|Pre] \bigg ) - \bigg ( E[Y^0_U | Post ] - E[ Y^0_U | Pre]}_{\mathclap{\text{Switching equation}}} \bigg)  \\
&&+ \underbrace{\textcolor{red}{E[Y_k^0 |Post] - E[Y^0_k | Post]}}_{\mathclap{\text{Adding zero}}} 
\end{eqnarray*}

\end{frame}

\begin{frame}{Parallel trends bias}

\begin{eqnarray*}
\widehat{\delta} &=& \underbrace{E[Y^1_k | Post] - \textcolor{red}{E[Y^0_k | Post]}}_{\mathclap{\text{ATT}}} \\
&& + \bigg [  \underbrace{\textcolor{red}{E[Y^0_k | Post]} - E[Y^0_k | Pre] \bigg ] - \bigg [ E[Y^0_U | Post] - E[Y_U^0 | Pre] }_{\mathclap{\text{Non-parallel trends bias in 2x2 case}}} \bigg ]
\end{eqnarray*}


\end{frame}

\begin{frame}{Identification through parallel trends}
	

	\begin{block}{Parallel trends}
	Assume two groups, treated and comparison group, then we define parallel trends as:	 $$\textcolor{red}{E(}\textcolor{red}{\Delta Y^0_k)} = E(\Delta Y^0_U)$$
	\end{block}

\textbf{In words}: ``The \textcolor{red}{evolution of earnings for our trainees \emph{had they not trained}} is the same as the evolution of mean earnings for non-trainees''.  

\bigskip

It's in \textcolor{red}{red} because parallel trends is untestable and critically important to estimation of the ATT using any method, OLS or ``four averages and three subtractions''

	

	
\end{frame}

\begin{frame}{Work together \#1}

Work together: 

\begin{eqnarray*}
\widehat{\delta} &=& \bigg ( E[Y_k|Post] - E[Y_k|Pre] \bigg ) - \bigg ( E[Y_U | Post ] - E[ Y_U | Pre] \bigg) \\
\end{eqnarray*}Assume that the $U$ group is treated in both periods.  Replace expectations with potential outcomes and rewrite using the ``add zero'' trick we did.  How is this similar to what we did before?  Is parallel trends enough?


\end{frame}

\begin{frame}{Work together \#2}

Work together: 

\begin{eqnarray*}
\widehat{\delta} &=& \bigg ( E[Y_k|Post] - E[Y_k|Pre] \bigg ) - \bigg ( E[Y_U | Post ] - E[ Y_U | Pre] \bigg) \\
\end{eqnarray*}In the pre-period, how important is it that the $U$ group and $k$ group have the same treatment status?  


\end{frame}






\begin{frame}{What is parallel trends}

\begin{itemize}
\item Parallel trends assumes away the selection bias associated with comparisons
\item The assumption is thought to be more plausible than simply assuming simple comparisons held equal $E[Y^0|D=0]=E[Y^0|D=1]$
\item But it is still a strong assumption, and differs from the assumptions have in the RCT which though also untestable, is nearly guaranteed by randomization
\item Most of the hard part of the work involves the old fashioned detective work and the work of making good arguments with good exhibits (tables and figures)
\end{itemize}

\end{frame}



\begin{frame}{Understanding parallel trends through worksheets}

Before we move into regression, let's go through a simple exercise to really pin down these core ideas with simple calculations

\bigskip 

\url{https://docs.google.com/spreadsheets/d/1onabpc14JdrGo6NFv0zCWo-nuWDLLV2L1qNogDT9SBw/edit?usp=sharing}

\end{frame}


\begin{frame}{No Anticipation}

\begin{itemize}
\item Additional assumption is ``no anticipation'' -- poorly named as it doesn't require literally no anticipation
\item No anticipation means that the treatment effect happens only at the time that the treatment occurs or after, but not before
	\begin{itemize}
	\item \textbf{Example 1}: Tomorrow I win the lottery, but don't get paid yet. I decide to buy a new house today. That violates NA
	\item \textbf{Example 2}: Next year, a state lets you drive without a driver license and you know it. But you can't drive without a driver license today.  This satisfies NA.
	\end{itemize}
\item We need NA because we are comparing to a baseline period and it needs to not be treated
\end{itemize}

\end{frame}

\begin{frame}{SUTVA}

\begin{itemize}
\item Stable Unit Treatment Value Assumption (Imbens and Rubin 2015) focuses on what happens when in our analysis we are combining units (versus defining treatment effects)
	\begin{enumerate}
	\item \textbf{No Interference}: a treated unit cannot impact a control unit such that their potential outcomes change (unstable treatment value)
	\item \textbf{No hidden variation in treatment}: When units are indexed to receive a treatment, their dose is the same as someone else with that same index
	\item \textbf{Scale}: If scaling causes interference or changes inputs in production process, then \#1 or \#2 are violated
	\end{enumerate}
\item Shifts from defining treatment effects to estimating them, which means being careful about who is the control group, how you define treatments and what questions can and cannot be answered with this method
\end{itemize}

\end{frame}





\begin{frame}{OLS Specification}
	
	\begin{itemize}
	\item Simple DiD equation will identify ATT under parallel trends
	\item But so will a particular OLS specification (two groups and no covariates)
	\item OLS was historically preferred because
		\begin{itemize}
		\item OLS estimates the ATT under parallel trends
		\item Easy to calculate the standard errors
		\item Easy to include multiple periods
		\end{itemize}
	\item People liked it also because of differential timing, continuous treatments and covariates, but those are more complex so we address them later
	\end{itemize}
\end{frame}

\begin{frame}{Minimum wages}

\begin{itemize}
\item Card and Krueger (1994) have a famous study estimating causal effect (ATT) of minimum wages on employment
\item Let's listen to Card tell its origin \\ \url{https://youtu.be/1soLdywFb_Q?t=2680}
\item Exploited a policy change in New Jersey between February and November in mid-1990s where minimum wage was increased, but neighbor PA did not
\item Using DiD, they do not find a negative effect of the minimum wage on employment which is part of its legacy today, but I mainly present it to illustrate the history and the design principles
\end{itemize}

\end{frame}

\begin{frame}
	\begin{figure}
	\includegraphics[scale=0.5]{./lecture_includes/minwage_whore}
	\end{figure}
\end{frame}


\begin{frame}{Reaction to the paper}

\begin{itemize}

\item Was it or was it not controversial?  Yes and no

\item Orley's opinion about the paper's controversy at the time.  \\ \url{https://youtu.be/MOtbuRX4eyQ?t=1882}

\item What Card and Krueger \emph{experienced} 

\url{https://youtu.be/1soLdywFb_Q?t=3076}



\end{itemize}

\end{frame}

\begin{frame}{Card on that study}

\begin{quote}
``I’ve subsequently stayed away from the minimum wage literature for a number of reasons. First, it cost me a lot of friends. People that I had known for many years, for instance, some of the ones I met at my first job at the University of Chicago, became very angry or disappointed. They thought that in publishing our work we were being traitors to the cause of economics as a whole.''
\end{quote}


\end{frame}



\begin{frame}{OLS specification of the DiD equation}
	
	\begin{itemize}
	\item The correctly specified OLS regression is an interaction with time and group fixed effects:$$Y_{its} = \alpha + \gamma NJ_s + \lambda d_t + \delta (NJ \times d)_{st} + \varepsilon_{its}$$
		\begin{itemize}
		\item NJ is a dummy equal to 1 if the observation is from NJ
		\item d is a dummy equal to 1 if the observation is from November (the post period)
		\end{itemize}
	\item This equation takes the following values
		\begin{itemize}
		\item PA Pre: $\alpha$
		\item PA Post: $\alpha + \lambda$
		\item NJ Pre: $\alpha + \gamma$
		\item NJ Post: $\alpha + \gamma + \lambda + \delta$
		\end{itemize}
	\item DiD equation: (NJ Post - NJ Pre) - (PA Post - PA Pre) $= \delta$
	\end{itemize}
\end{frame}




\begin{frame}[plain]
	$$Y_{ist} = \alpha + \gamma NJ_s + \lambda d_t + \delta(NJ\times d)_{st} + \varepsilon_{ist}$$
	\begin{figure}
	\includegraphics[scale=0.90]{./lecture_includes/waldinger_dd_5.pdf}
	\end{figure}
\end{frame}


\begin{frame}[plain]
	$$Y_{ist} = \alpha + \gamma NJ_s + \lambda d_t + \delta(NJ\times d)_{st} + \varepsilon_{ist}$$
	\begin{figure}
	\includegraphics[scale=0.90]{./lecture_includes/waldinger_dd_5.pdf}
	\end{figure}

Notice how OLS is ``imputing'' $E[Y^0|D=1,Post]$ for the treatment group in the post period? It is only ``correct'', though, if parallel trends is a good approximation

\end{frame}

\subsection{Inference}

\begin{frame}{Introduction to Inference in DID}
  When dealing with clustered data, a crucial concept is the difference between correlated observations and correlated errors. While they may seem similar, they are distinct, and it's essential to focus on the errors when clustering standard errors.
\end{frame}

\begin{frame}{Correlated Observations}
  \begin{itemize}
    \item Correlated observations occur when the observed variables themselves are correlated within a cluster.
    \item For instance, incomes within a specific region might be positively correlated.
    \item Correlated observations do not necessarily violate OLS assumptions.
  \end{itemize}
\end{frame}

\begin{frame}{Correlated Errors}
  \begin{itemize}
    \item Correlated errors occur when the unobserved errors are correlated within a cluster.
    \item This violates the classical OLS assumption of independent errors, leading to inefficient and possibly biased standard errors.
    \item Clustering standard errors accounts for this within-cluster correlation.
  \end{itemize}
\end{frame}

\begin{frame}{Why Focus on Correlated Errors?}
  \begin{itemize}
    \item OLS assumptions are about errors, not the observed variables.
    \item Failing to account for correlated errors can lead to misleading inference.
    \item Clustering standard errors at the appropriate level corrects for this, giving more reliable hypothesis tests and confidence intervals.
  \end{itemize}
\end{frame}


\begin{frame}{Serial correlation creates problems}
  \begin{itemize}
    \item So errors within a cluster (e.g., same treatment group) might be correlated.
    \item Even if between clusters, they are assumed to be independent, the within correlation in errors biases standard errors downward
	\item  Bertrand, Duflo and Mullainathan (2004) show that conventional standard errors will often severely understate the standard deviation of the estimators
	\item They proposed three solutions, but most only use one of them (clustering)

  \end{itemize}
\end{frame}




\begin{frame}{Inference}
	
		\begin{enumerate}
		\item[1 ] Block bootstrapping standard errors (if you analyze states the block should be the states and you would sample whole states with replacement for bootstrapping)
		\item[2 ] Clustering standard errors at the group level (in Stata one would simply add \texttt{, cluster(state)} to the regression equation if one analyzes state level variation)
		\end{enumerate}

\bigskip

Most people will simply cluster, but there are issues if you have too few clusters. They mention a third way but it's only a curiosity.
		
\end{frame}


\begin{frame}{Computing Cluster-Robust Standard Errors}
  \begin{enumerate}
    \item Run your regression to get \( u \) and \( \hat{Y} \).
    \item Calculate cluster-level residuals \( \hat{u}_c \).
    \item Calculate the "meat" (as in bread-meat-bread sandwich) \( M \) as \( \sum_c X_c^\prime \hat{u}_c \hat{u}_c^\prime X_c \).
    \item Your standard error covariance matrix is \( \hat{V} = (X^\prime X)^{-1} M (X^\prime X)^{-1} \).
  \end{enumerate}
  
  \bigskip
  
  Mixtape reference (chapter 2): \url{https://mixtape.scunning.com/02-probability\_and\_regression\#cluster-robust-standard-errors}
\end{frame}

\begin{frame}{Compressing Into a Single Number}
  \begin{itemize}
    \item Diagonal elements of \( \hat{V} \) contain variances for each coefficient.
    \item Standard errors are the square root of these diagonal elements.
    \item Non-diagonal elements are used for hypothesis tests and confidence intervals for combinations of coefficients.
  \end{itemize}
\end{frame}





\section{Parallel trends violations}

\subsection{How parallel trends can get violated}


\begin{frame}{Violating parallel trends exercise}

\begin{itemize}
\item Parallel trends are needed so we can impute the missing $E[Y^0|D=1]$ with $E[Y^0|D=0]$ either explicitly or implicitly
\item Which means if parallel trends isn't true, then the imputation isn't correct and therefore estimates are biased
\item To illustrate this, let's go through the document again
\end{itemize}

\url{https://docs.google.com/spreadsheets/d/1onabpc14JdrGo6NFv0zCWo-nuWDLLV2L1qNogDT9SBw/edit?usp=sharing}

\end{frame}


\begin{frame}{Violating parallel trends}

\begin{itemize}
\item Parallel trends are in expectation only -- we don't rely everybody to follow the same trend, just that the group average for $Y^0$ be approximately the same for treated and control 
\item Violations are a form of selection bias and there are two straightforward ways that parallel trends will be violated
	\begin{enumerate}
	\item Compositional differences in samples associated with repeated cross-sections
	\item Policy endogeneity
	\end{enumerate}
\end{itemize}

\end{frame}



\begin{frame}{Repeated cross-sections and compositional change}
	
	\begin{itemize}
	\item One of the risks of a repeated cross-section is that the composition of the sample may have changed between the pre and post period in ways that are correlated with treatment
	\item Hong (2013) uses repeated cross-sectional data from the Consumer Expenditure Survey (CEX) containing music expenditure and internet use for a random sample of households
	\item Study exploits the emergence of Napster (first file sharing software widely used by Internet users) in June 1999 as a natural experiment
	\item Study compares internet users and internet non-users before and after emergence of Napster
	\end{itemize}

\end{frame}

\begin{frame}{Introduction of Napster and spending on music}
	\begin{figure}
	\includegraphics[scale=0.5]{./lecture_includes/hong_napster}
	\end{figure}
	
\end{frame}


\begin{frame}[plain]
	\begin{figure}
	\includegraphics{./lecture_includes/Hong_1.pdf}
	\end{figure}
	
\end{frame}





\begin{frame}[shrink=20,plain]
	\begin{figure}
	\includegraphics{./lecture_includes/Hong_2.pdf}
	\end{figure}
	
	Diffusion of the Internet changes samples (e.g., younger music fans are early adopters)
	
\end{frame}

\begin{frame}{Repeated cross-sections}

\begin{itemize}
\item Surprisingly underappreciated problem with almost no literature around it
\item So what can you do?  Check covariate balance by regressing the time-varying covariates instead of the outcome onto the treatment using your OLS specification
\item They should be exogenous remember, so this covariate regression can be a helpful test of whether this is a problem
\item ``Difference-in-differences with Compositional Changes'' by Pedro Sant'anna and Qi Xu (not yet released) is the only paper I've ever seen to look into it
\end{itemize}

\end{frame}


\subsection{DiD in Court}

\begin{frame}{Types of evidence}

\begin{itemize}
\item You are building a case, the prosecutor before a judge and jury, always in battle with the defense attorney
\item Evidence has particular broadly defined forms that can help you on the front end
\item Your goal in my humble opinion should be mixing tight logic based falsifications with particular kinds of data visualization, starting with the event study
\end{itemize}

\end{frame}


\begin{frame}{Event studies have become mandatory in DiD}

	\begin{figure}
	\includegraphics[scale=0.5]{./lecture_includes/currie_eventstudy.png}
	\end{figure}

\end{frame}

\begin{frame}{Intuition behind event studies}

\begin{itemize}
	\item Princeton Industrial Relations Section seems to be behind this -- this intense focus on research design but also verifying assumptions
	\item The identifying assumption for all DD designs is parallel trends , but since we cannot verify parallel trends, we often look at pre-trends
	\item It's a type of check for selection bias, but you must understand what it is and what it isn't to see its value but not be naive about it (it is not a silver bullet)
	\item Even if pre-trends are the same one still has to worry about other policies changing at the same time (omitted variable bias is a parallel trends violation)

\end{itemize}

\end{frame}

\begin{frame}{Ashenfelter's dip}

Orley: \url{https://youtu.be/MOtbuRX4eyQ?t=960}

Card: \url{https://youtu.be/1soLdywFb_Q?t=3333}

\end{frame}




\begin{frame}{Plot the raw data when there's only two groups}

	\begin{figure}
	\includegraphics[scale=2.5]{./lecture_includes/waldinger_dd_6.pdf}
	\end{figure}

\end{frame}

\begin{frame}{Evidence for parallel trends: pre-trends}

Let's do the bonus questions on first and second tab now 

\url{https://docs.google.com/spreadsheets/d/1onabpc14JdrGo6NFv0zCWo-nuWDLLV2L1qNogDT9SBw/edit?usp=sharing}

\end{frame}

\begin{frame}{Beware of naive reasoning}

\begin{itemize}
\item Parallel pre-trends $\neq$ parallel trends -- these are often thought to be the same thing, and they aren't
\item Equating them is a kind of \emph{post hoc ergo propter hoc} fallacy
\item Parallel pre-trends is more like a smoking gun based on things ``looking the same'' before
\item Similar to checking for covariate balance in RCTs
\item But \textbf{cannot} substitute for domain knowledge!
\end{itemize}

\end{frame}


\begin{frame}{Event study regression}
	
	\begin{itemize}
	\item Event studies have a simple OLS specification with only one treatment group and one never-treated group $$Y_{its} = \alpha +  \sum_{\tau=-2}^{-q}\mu_{\tau}D_{s\tau} + \sum_{\tau=0}^m\delta_{\tau}D_{s\tau}+\varepsilon_{ist}$$
		\begin{itemize}
		\item where $D$ is an interaction of the treatment group $s$ with the calendar year $\tau$
		\item Treatment occurs in year 0, no anticipation, drop baseline $t-1$
		\item Includes $q$ leads or anticipatory effects and $m$ lags or post treatment effects
		\end{itemize}
	\end{itemize}
\end{frame}

\begin{frame}{Event study regression}


$$Y_{its} = \alpha + \sum_{\tau=-2}^{-q}\mu_{\tau}D_{s\tau} + \sum_{\tau=0}^m\delta_{\tau}D_{s\tau}+\varepsilon_{ist}$$

\bigskip

Typically you'll plot the coefficients and 95\% CI on all leads and lags (binned or not, trimmed or not) 

\bigskip

Under no anticipation, then you expect $\widehat{\mu}$ coefficients to be zero, which gives you confidence that parallel trends holds (but is not a guarantee, and there are still specification issues -- see Jon Roth's work)

\bigskip

Under parallel trends, $\widehat{\delta}$ are estimates of the ATT at points in time

\end{frame}



\begin{frame}{Medicaid and Affordable Care Act example}

\begin{figure}
\includegraphics[scale=0.25]{./lecture_includes/medicaid_qje}
\end{figure}

\end{frame}
\begin{frame}{Types of evidence}

\begin{itemize}
\item \textbf{Bite} -- show that the expansion shifted people into Medicaid and out of uninsured status
\item \textbf{Main Results} -- Show your main results (the point of the paper)
\item \textcolor{red}{\textbf{Placebos}} -- Show that there's no effect on mortality for groups it shouldn't be affecting (people 65+)
\item \textcolor{blue}{\textbf{Mechanisms}} -- Find some reason explaining why the treatment affects the outcome via some ``mechanism''
\item \textbf{Event study} -- Show leads and lags on mortality
\end{itemize}

\end{frame}


\imageframe{./lecture_includes/Miller_Medicaid1.png}

\imageframe{./lecture_includes/Miller_Medicaid2.png}

\imageframe{./lecture_includes/Miller_Medicaid3.png}


\begin{frame}{Falsifications on elderly}

	\begin{figure}
\includegraphics[scale=0.425]{./lecture_includes/placebo_medicaid}
	\end{figure}

\end{frame}



\begin{frame}[plain]

	\begin{figure}
	\includegraphics[scale=0.3]{./lecture_includes/Miller_Medicaid4.png}
	\caption{Miller, et al. (2019) estimates of Medicaid expansion's effects on on annual mortality}
	\end{figure}

\end{frame}

\begin{frame}{Mechanism}

\begin{itemize}
\item Bite: Increases in enrollment and reductions in uninsured support that there is adoption of the treatment
\item Main Results: 9.2\% reduction in mortality among the near-elderly
\item Falsifications: no effect on a similar group who isn't eligible
\item Mechanism: ``The effect is driven by a reduction in disease-related deaths and grows over time.''
\item Event studies: Compelling once the others are established
\end{itemize}

\end{frame}

\begin{frame}{Making event study}

\begin{itemize}
\item All the simple event study is is an interaction between the treatment group dummy and the calendar year dummies
\item You must drop a $t-\tau$ as the baseline (e.g., $t-1$) must be $Y^0$ untreated comparisons recall
\item I have included in a do file that will do it for you either manually or using coefplot in \texttt{simple\_eventstudy.do} at github labs
\end{itemize}

\end{frame}


\begin{frame}{Manually creating the event study}

	\begin{figure}
	\includegraphics[scale=0.20]{./lecture_includes/simple_eventstudy_manual}
	\end{figure}

\end{frame}




\begin{frame}{Creating the event study with Ben Jann's \texttt{coefplot}}

	\begin{figure}
	\includegraphics[scale=0.20]{./lecture_includes/simple_eventstudy.png}
	\end{figure}

\end{frame}





\subsection{Falsifications}


\begin{frame}{Falsifications on more comparison groups}
	
	\begin{itemize}
	\item Very common for readers and others to request a variety of ``robustness checks'' from a DD design
	\item We saw some of these just now (e.g., falsification test using data for alternative control group, the Medicare population)
	\item Triple differences is another way to do that, but it will have a somewhat different parallel trends assumption
	\end{itemize}
\end{frame}





\begin{frame}{Biased diff-in-diff \#1}

\begin{table}\centering
\scriptsize
		\caption{Biased diff-in-diff \#1: comparing states}
		\begin{center}
		\begin{tabular}{lll|lc}
		\toprule
		\multicolumn{1}{l}{\textbf{States}}&
		\multicolumn{1}{c}{\textbf{Period}}&
		\multicolumn{1}{c}{\textbf{Outcomes}}&
		\multicolumn{1}{c}{$D_1$}&
		\multicolumn{1}{c}{$D_2$}\\
		\midrule
		Experimental states & Before & $Y=NJ$ \\
		& After & $Y=NJ + NJ_t + D$ & $\textcolor{red}{NJ_t}+D$\\
		\midrule
		& & & & $D + (\textcolor{red}{NJ_t}- PA_t)$ \\
		\midrule
		Non-experimental  & Before & $Y=PA$ \\
		states& After & $Y=PA + PA_t$ & $PA_t$\\
		\bottomrule
		\end{tabular}
		\end{center}
	\end{table}

\begin{eqnarray*}
\widehat{\delta}_{did} = D + (\textcolor{red}{NJ_t}- PA_t)
\end{eqnarray*}The ATT is D. Diff-in-diff is an unbiased estimate of D if $\textcolor{red}{NJ_t} = PA_t$. Call parallel trends.

\end{frame}


\begin{frame}{Biased diff-in-diff \#2}

\begin{table}\centering
\scriptsize
		\caption{Biased diff-in-diff \#2: comparing groups}
		\begin{center}
		\begin{tabular}{lll|lc}
		\toprule
		\multicolumn{1}{l}{\textbf{Groups}}&
		\multicolumn{1}{c}{\textbf{Period}}&
		\multicolumn{1}{c}{\textbf{Outcomes}}&
		\multicolumn{1}{c}{$D_1$}&
		\multicolumn{1}{c}{$D_2$}\\
		\midrule
		Married women & Before & $Y=MW$ \\
		& After & $Y=MW + MW_t + D$ & $\textcolor{red}{MW_t}+D$\\
		\midrule
		& & & & $D + (\textcolor{red}{MW_t}- SO_t)$ \\
		\midrule
		Single men  & Before & $Y=SO$ \\
		and older women& After & $Y=SO + SO_t$ & $SO_t$\\
		\bottomrule
		\end{tabular}
		\end{center}
	\end{table}

\begin{eqnarray*}
\widehat{\delta}_{did} = D + (\textcolor{red}{MW_t}- SO_t)
\end{eqnarray*}Also unbiased estimate of D if $\textcolor{red}{MW_t} = SO_t$. A different parallel trends assumption.

\end{frame}


\begin{frame}{Two biased diff-in-diffs}

\begin{itemize}

\item If parallel trends of one doesn't hold, use the other, because both identify the ATT under their respective parallel trends
\item But what if both of them are biased -- then you can't use diff-in-diff because diff-in-diff requires at least one parallel trends
\item Okay, but what if both biases are the same?
	\begin{eqnarray*}
	(\textcolor{red}{NJ_t} -PA_t) = (\textcolor{red}{MW_t} - SO_t)
	\end{eqnarray*}
\item Then you can use triple differences, first developed by Gruber (1994)

\end{itemize}

\end{frame}

\begin{frame}{Triple differences by Gruber (1995)}
	
	\begin{figure}
	\includegraphics{./lecture_includes/gruber_ddd_3.pdf}
	\end{figure}
	
\end{frame}

\begin{frame}{Triple differences commentary}

\begin{itemize}

\item In Gruber's 1994 article, it isn't clear why he needed triple differences in the first place -- his triple differences yielded -0.054 which is almost the same as what he found with his first diff-in-diff (-0.062)
\item The main value of triple differences is that you use it when you believe the parallel trends assumption doesn't hold

\end{itemize}

\end{frame}


\begin{frame}[shrink=20]

\begin{table}\centering
		\caption{Difference-in-Difference-in-Differences (Gruber version)}
		\tiny
		\begin{center}
		\begin{tabular}{lll|l|lll}
		\toprule
		\multicolumn{1}{l}{\textbf{Groups}}&
		\multicolumn{1}{c}{\textbf{States}}&
		\multicolumn{1}{c}{\textbf{Period}}&
		\multicolumn{1}{c}{\textbf{Outcomes}}&
		\multicolumn{1}{c}{$D_1$}&
		\multicolumn{1}{c}{$D_2$}&
		\multicolumn{1}{c}{$D_3$}\\
		\midrule
		&&After	&$NJ+MW+\textcolor{blue}{NJ_t}+\textcolor{red}{MW_t}+D$					\\
	&Experimental			&&&$\textcolor{blue}{NJ_t}+\textcolor{red}{MW_t}+D$			\\
		&states&Before	&$NJ+MW$					\\
Married women 					&&&&&$D+\textcolor{blue}{NJ_t} + \textcolor{red}{MW}_t$			\\
20-40							&&&&&$-(PA_t + MW_t)$\\
    		&&After	&$PA+MW+PA_t+MW_t$					\\
	&Non-experimental		&&	&$PA_t+MW_t$				\\
		&states&Before	&$PA+MW$					\\
								\\
&&&&&&$\textcolor{black}{D}$
\\
		&&After	&$NJ+SO+NJ_t+SO_t$				\\
	&Experimental 			&&&$NJ_t+SO_t$ \\				
		&states&Before	&$NJ+SO$					\\
Single men 					&&&&&$NJ_t + SO_t$		\\
Older women					&&&&&$-(PA_t + SO_t) $		\\
	     	&&After	&$PA+SO+PA_t+SO_t$					\\
	&Non-experimental		&&&	$PA_t+SO_t$				\\
		&states&Before	&$PA+SO$					\\
		\bottomrule
		\end{tabular}
		\end{center}
	\end{table}

\begin{center}
\textbf{Triple diff assumption}
\end{center}

$\widehat{\delta}_{DDD}= D + \bigg ((\textcolor{blue}{NJ_t} + \textcolor{red}{MW_t}) - (PA_t + MW_t) \bigg )- \bigg ( (NJ_t + SO_t) - (PA_t + SO_t) \bigg) $

Recall that our first DiD was biased by $NJ_t - PA_t$ and \\
our second DiD was biased by $MW_t - SO_t$. 

\end{frame}

\begin{frame}{Two biased diff-in-diff}

\scriptsize
\begin{eqnarray*}
&=& [(\textcolor{blue}{NJ^{MW}_t} + \textcolor{red}{MW^{NJ}_t}) - (PA^{MW}_t + MW^{PA}_t) ]- [ (NJ^{SO}_t + SO^{NJ}_t) - (PA^{SO}_t + SO^{PA}_t) ]  \\
&=& (\textcolor{blue}{NJ^{MW}_t} + \textcolor{red}{MW^{NJ}_t} - PA^{MW}_t - MW^{PA}_t )- (NJ^{SO}_t + SO^{NJ}_t - PA^{SO}_t - SO^{PA}_t ) \\
&=& \textcolor{blue}{NJ^{MW}_t} + \textcolor{red}{MW^{NJ}_t} - PA^{MW}_t - MW^{PA}_t - NJ^{SO}_t - SO^{NJ}_t + PA^{SO}_t + SO^{PA}_t  \\
&=& \textcolor{blue}{NJ^{MW}_t} - PA^{MW}_t - NJ^{SO}_t + PA^{SO}_t + \textcolor{red}{MW^{NJ}_t} - MW^{PA}_t - SO^{NJ}_t + SO^{PA}_t  \\
&=& (\textcolor{blue}{NJ^{MW}_t} - PA^{MW}_t - NJ^{SO}_t + PA^{SO}_t )+ (\textcolor{red}{MW^{NJ}_t} - SO^{NJ}_t- MW^{PA}_t  + SO^{PA}_t ) \\
&=& [(\textcolor{blue}{NJ^{MW}_t} - PA^{MW}_t) -( NJ^{SO}_t - PA^{SO}_t )]- [(SO^{NJ}_t-\textcolor{red}{MW^{NJ}_t}) - ( SO^{PA}_t  -  MW^{PA}_t )] \\
&=& \underbrace{[(\textcolor{blue}{NJ^{MW}_t} - PA^{MW}_t) -( NJ^{SO}_t - PA^{SO}_t )]}_{\mathclap{\text{Equally biased  DiD \#1}}} - \underbrace{[(SO^{NJ}_t-\textcolor{red}{MW^{NJ}_t}) - ( SO^{PA}_t  -  MW^{PA}_t )]}_{\mathclap{\text{Equally biased DiD \#2}}} \\
\end{eqnarray*}

Triple differences requires two diff-in-diff, each with the same bias. Parallel bias

\end{frame}






\begin{frame}{DDD in Regression}
	
	\begin{eqnarray*}
	Y_{ijt} &=&\alpha +  \beta_2 \tau_t + \beta_3 \delta_j  + \beta_4 D_i + \beta_5(\delta \times \tau)_{jt} \\
	&& +\ \beta_6(\tau \times D)_{ti} +  \beta_7(\delta \times D)_{ij} +  \textcolor{red}{\beta_8(\delta \times \tau \times  D)_{ijt}}+  \varepsilon_{ijt}
	\end{eqnarray*}
	
	\begin{itemize}
	\item Your panel is now a group $j$ state $i$ (e.g., AR high wage worker 1991, AR high wage worker 1992, etc.)
	\item Assume we drop $\tau_t$ but I just want to show it to you for now.
	\item If the placebo DD is non-zero, it might be difficult to convince the reviewer that the DDD removed all the bias 
	\end{itemize}
	
\end{frame}

\begin{frame}{Great new paper to learn more}

\begin{figure}
\includegraphics[scale=0.25]{./lecture_includes/olden_moen_2022_ddd.png}
\end{figure}

\end{frame}




\begin{frame}{Falsification on outcomes}
	
	\begin{itemize}
	\item The within-group control group (DDD) is a form of placebo analysis using the same \emph{outcome}
	\item But there are also placebos using a \emph{different} outcome -- but you need a hypothesis of mechanisms to figure out what is in fact a \emph{different outcome}
	\item Figure out what those are, and test them -- finding no effect on placebo outcomes tends to help people your other results interestingly enough
	\item Cheng and Hoekstra (2013) examine the effect of castle doctrine gun laws on non-gun related offenses like grand theft auto and find no evidence of an effect 
	\end{itemize}
\end{frame}



\begin{frame}{Rational addiction as a placebo critique}


Sometimes, an empirical literature may be criticized using nothing more than placebo analysis

\begin{quote}``A majority of [our] respondents believe the literature is a success story that demonstrates the power of economic reasoning.  At the same time, they also believe the empirical evidence is weak, and they disagree both on the type of evidence that would validate the theory and the policy implications. Taken together, this points to an interesting gap.  On the one hand, most of the respondents claim that the theory has valuable real world implications.  On the other hand, they do not believe the theory has received empirical support.''
\end{quote}

\end{frame}

\begin{frame}{Placebo as critique of empirical rational addiction}

\begin{itemize}
	\item Auld and Grootendorst (2004) estimated standard ``rational addiction'' models (Becker and Murphy 1988) on data with milk, eggs, oranges and apples.  
	\item They find these plausibly non-addictive goods are addictive, which casts doubt on the empirical rational addiction models.
\end{itemize}

\end{frame}

\begin{frame}{Placebo as critique of peer effects}

\begin{itemize}
	\item Several studies found evidence for ``peer effects'' involving inter-peer transmission of smoking, alcohol use and happiness tendencies
	\item Christakis and Fowler (2007) found significant network effects on outcomes like obesity
	\item Cohen-Cole and Fletcher (2008) use similar models and data and find similar network ``effects'' for things that \emph{aren't} contagious like acne, height and headaches
	\item Ockham's razor - given social interaction endogeneity (Manski 1993), homophily more likely explanation
\end{itemize}

\end{frame}


\end{document}


