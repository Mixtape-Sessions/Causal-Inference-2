\documentclass{beamer}

\input{preamble.tex}
\usepackage{breqn} % Breaks lines

\usepackage{amsmath}
\usepackage{mathtools}

\usepackage{pdfpages} % \includepdf

\usepackage{listings} % R code
\usepackage{verbatim} % verbatim

% Video stuff
\usepackage{media9}

% packages for bibs and cites
\usepackage{natbib}
\usepackage{har2nat}
\newcommand{\possessivecite}[1]{\citeauthor{#1}'s \citeyearpar{#1}}
\usepackage{breakcites}
\usepackage{alltt}

% Setup math operators
\DeclareMathOperator{\E}{E} \DeclareMathOperator{\tr}{tr} \DeclareMathOperator{\se}{se} \DeclareMathOperator{\I}{I} \DeclareMathOperator{\sign}{sign} \DeclareMathOperator{\supp}{supp} \DeclareMathOperator{\plim}{plim}
\DeclareMathOperator*{\dlim}{\mathnormal{d}\mkern2mu-lim}
\newcommand\independent{\protect\mathpalette{\protect\independenT}{\perp}}
   \def\independenT#1#2{\mathrel{\rlap{$#1#2$}\mkern2mu{#1#2}}}
\newcommand*\colvec[1]{\begin{pmatrix}#1\end{pmatrix}}

\newcommand{\myurlshort}[2]{\href{#1}{\textcolor{gray}{\textsf{#2}}}}


\begin{document}

\imageframe{./lecture_includes/mixtape_did_cover.png}


% ---- Content ----


\begin{frame}{Introduction}

\begin{itemize}
\item Welcome to Mixtape Sessions workshop on difference-in-differences and synthetic control
\item 8:00am to 5:00pm CST, 15 min breaks every hour, 1 hour lunch
\item Lecture, discussion and exercises
\end{itemize}

\end{frame}


\begin{frame}{Workshop outline}

Introduction to DiD basics 
	\begin{itemize}
	\item Potential outcomes review
	\item DiD equation and OLS
	\item Triple differences, event studies
	\item Including covariates
	\end{itemize}

\end{frame}


\begin{frame}{Workshop outline}

 Differential timing
	\begin{itemize}
	\item Decomposing constant treatment effect model (TWFE)
	\item Aggregating group-time ATTs
	\item Issues and solutions with event studies
	\item Turning treatments on and off
	\item Stacked regression
	\item Explicit Imputation
	\item Continuous treatments
	\item Spatial DiD
	\end{itemize}

\end{frame}

\begin{frame}{Workshop outline}

Synthetic control

\begin{itemize}
\item Canonical synth
\item Matrix completion with nuclear norm regularization
\item Augmented synthetic control 
\end{itemize}

\end{frame}

\begin{frame}{Natural experiments}

\begin{quote}
``A good way to do econometrics is to look for good natural experiments and use statistical methods that can tidy up the confounding factors that nature has not controlled for us.'' -- Daniel McFadden (Nobel Laureate recipient with Heckman 1992)
\end{quote}

\end{frame}


\imageframe{./lecture_includes/bumpersticker.jpeg}


\begin{frame}{What is difference-in-differences (DiD)}

\begin{itemize}
\item DiD is a very old, relatively straightforward, intuitive research design
\item A group of units are assigned some treatment and then compared to a group of units that weren't
\item Closely associated with 1980s Princeton Industrial Relations Section (Orley Ashenfelter, David Card, Alan Krueger, Bob LaLonde)
\item Unclear when key identifying assumptions like parallel trends are worked out (as originally there is no potential outcomes in play), but Angrist says he invented the term with Pischke
\item Most widely used quasi-experimental method
\end{itemize}

\end{frame}


\begin{frame}

	\begin{figure}
	\caption{Currie, et al. (2020)}
	\includegraphics[scale=0.25]{./lecture_includes/currie_did.png}
	\end{figure}


\end{frame}






\begin{frame}{Why an entire workshop on DiD?}

\begin{itemize}
\item \textbf{Research advantages}: DiD is sometimes the only way we have to study large social policies 
\item \textbf{Time to retool}: Recent wave of scholarship suggest model misspecification has been pronounced
\item \textbf{Good news}: Better understanding of our models, new tools, new programs
\end{itemize}

\end{frame}







\begin{frame}{Let's begin with DiD}

\begin{itemize}
\item With all this out of the way, let's dig into the DiD material
\item We will start with the simplest situation using simple difference in means without covariates
\item We will then move into OLS with covariates
\item And then move into alternatives to OLS when we have covariates
\item Later we go into the more advanced material (e.g., differential timing, continuous treatments)
\end{itemize}

\end{frame}







\end{document}
