\documentclass{beamer}

% xcolor and define colors -------------------------
\usepackage{xcolor}

% https://www.viget.com/articles/color-contrast/
\definecolor{purple}{HTML}{5601A4}
\definecolor{navy}{HTML}{0D3D56}
\definecolor{ruby}{HTML}{9a2515}
\definecolor{alice}{HTML}{107895}
\definecolor{daisy}{HTML}{EBC944}
\definecolor{coral}{HTML}{F26D21}
\definecolor{kelly}{HTML}{829356}
\definecolor{cranberry}{HTML}{E64173}
\definecolor{jet}{HTML}{131516}
\definecolor{asher}{HTML}{555F61}
\definecolor{slate}{HTML}{314F4F}

% Mixtape Sessions
\definecolor{picton-blue}{HTML}{00b7ff}
\definecolor{violet-red}{HTML}{ff3881}
\definecolor{sun}{HTML}{ffaf18}
\definecolor{electric-violet}{HTML}{871EFF}

% Main theme colors
\definecolor{accent}{HTML}{00b7ff}
\definecolor{accent2}{HTML}{871EFF}
\definecolor{gray100}{HTML}{f3f4f6}
\definecolor{gray800}{HTML}{1F292D}


% Beamer Options -------------------------------------

% Background
\setbeamercolor{background canvas}{bg = white}

% Change text margins
\setbeamersize{text margin left = 15pt, text margin right = 15pt} 

% \alert
\setbeamercolor{alerted text}{fg = accent2}

% Frame title
\setbeamercolor{frametitle}{bg = white, fg = jet}
\setbeamercolor{framesubtitle}{bg = white, fg = accent}
\setbeamerfont{framesubtitle}{size = \small, shape = \itshape}

% Block
\setbeamercolor{block title}{fg = white, bg = accent2}
\setbeamercolor{block body}{fg = gray800, bg = gray100}

% Title page
\setbeamercolor{title}{fg = gray800}
\setbeamercolor{subtitle}{fg = accent}

%% Custom \maketitle and \titlepage
\setbeamertemplate{title page}
{
    %\begin{centering}
        \vspace{20mm}
        {\Large \usebeamerfont{title}\usebeamercolor[fg]{title}\inserttitle}\\
        {\large \itshape \usebeamerfont{subtitle}\usebeamercolor[fg]{subtitle}\insertsubtitle}\\ \vspace{10mm}
        {\insertauthor}\\
        {\color{asher}\small{\insertdate}}\\
    %\end{centering}
}

% Table of Contents
\setbeamercolor{section in toc}{fg = accent!70!jet}
\setbeamercolor{subsection in toc}{fg = jet}

% Button 
\setbeamercolor{button}{bg = accent}

% Remove navigation symbols
\setbeamertemplate{navigation symbols}{}

% Table and Figure captions
\setbeamercolor{caption}{fg=jet!70!white}
\setbeamercolor{caption name}{fg=jet}
\setbeamerfont{caption name}{shape = \itshape}

% Bullet points

%% Fix left-margins
\settowidth{\leftmargini}{\usebeamertemplate{itemize item}}
\addtolength{\leftmargini}{\labelsep}

%% enumerate item color
\setbeamercolor{enumerate item}{fg = accent}
\setbeamerfont{enumerate item}{size = \small}
\setbeamertemplate{enumerate item}{\insertenumlabel.}

%% itemize
\setbeamercolor{itemize item}{fg = accent!70!white}
\setbeamerfont{itemize item}{size = \small}
\setbeamertemplate{itemize item}[circle]

%% right arrow for subitems
\setbeamercolor{itemize subitem}{fg = accent!60!white}
\setbeamerfont{itemize subitem}{size = \small}
\setbeamertemplate{itemize subitem}{$\rightarrow$}

\setbeamertemplate{itemize subsubitem}[square]
\setbeamercolor{itemize subsubitem}{fg = jet}
\setbeamerfont{itemize subsubitem}{size = \small}


% Special characters

\usepackage{collectbox}

\makeatletter
\newcommand{\mybox}{%
    \collectbox{%
        \setlength{\fboxsep}{1pt}%
        \fbox{\BOXCONTENT}%
    }%
}
\makeatother





% Links ----------------------------------------------

\usepackage{hyperref}
\hypersetup{
  colorlinks = true,
  linkcolor = accent2,
  filecolor = accent2,
  urlcolor = accent2,
  citecolor = accent2,
}


% Line spacing --------------------------------------
\usepackage{setspace}
\setstretch{1.1}


% \begin{columns} -----------------------------------
\usepackage{multicol}


% Fonts ---------------------------------------------
% Beamer Option to use custom fonts
\usefonttheme{professionalfonts}

% \usepackage[utopia, smallerops, varg]{newtxmath}
% \usepackage{utopia}
\usepackage[sfdefault,light]{roboto}

% Small adjustments to text kerning
\usepackage{microtype}



% Remove annoying over-full box warnings -----------
\vfuzz2pt 
\hfuzz2pt


% Table of Contents with Sections
\setbeamerfont{myTOC}{series=\bfseries, size=\Large}
\AtBeginSection[]{
        \frame{
            \frametitle{Roadmap}
            \tableofcontents[current]   
        }
    }


% Tables -------------------------------------------
% Tables too big
% \begin{adjustbox}{width = 1.2\textwidth, center}
\usepackage{adjustbox}
\usepackage{array}
\usepackage{threeparttable, booktabs, adjustbox}
    
% Fix \input with tables
% \input fails when \\ is at end of external .tex file
\makeatletter
\let\input\@@input
\makeatother

% Tables too narrow
% \begin{tabularx}{\linewidth}{cols}
% col-types: X - center, L - left, R -right
% Relative scale: >{\hsize=.8\hsize}X/L/R
\usepackage{tabularx}
\newcolumntype{L}{>{\raggedright\arraybackslash}X}
\newcolumntype{R}{>{\raggedleft\arraybackslash}X}
\newcolumntype{C}{>{\centering\arraybackslash}X}

% Figures

% \imageframe{img_name} -----------------------------
% from https://github.com/mattjetwell/cousteau
\newcommand{\imageframe}[1]{%
    \begin{frame}[plain]
        \begin{tikzpicture}[remember picture, overlay]
            \node[at = (current page.center), xshift = 0cm] (cover) {%
                \includegraphics[keepaspectratio, width=\paperwidth, height=\paperheight]{#1}
            };
        \end{tikzpicture}
    \end{frame}%
}

% subfigures
\usepackage{subfigure}


% Highlight slide -----------------------------------
% \begin{transitionframe} Text \end{transitionframe}
% from paulgp's beamer tips
\newenvironment{transitionframe}{
    \setbeamercolor{background canvas}{bg=accent!40!black}
    \begin{frame}\color{accent!10!white}\LARGE\centering
}{
    \end{frame}
}


% Table Highlighting --------------------------------
% Create top-left and bottom-right markets in tabular cells with a unique matching id and these commands will outline those cells
\usepackage[beamer,customcolors]{hf-tikz}
\usetikzlibrary{calc}
\usetikzlibrary{fit,shapes.misc}

% To set the hypothesis highlighting boxes red.
\newcommand\marktopleft[1]{%
    \tikz[overlay,remember picture] 
        \node (marker-#1-a) at (0,1.5ex) {};%
}
\newcommand\markbottomright[1]{%
    \tikz[overlay,remember picture] 
        \node (marker-#1-b) at (0,0) {};%
    \tikz[accent!80!jet, ultra thick, overlay, remember picture, inner sep=4pt]
        \node[draw, rectangle, fit=(marker-#1-a.center) (marker-#1-b.center)] {};%
}

\usepackage{breqn} % Breaks lines

\usepackage{tikz}
\usetikzlibrary{positioning}

\usepackage{amsmath}
\usepackage{mathtools}

\usepackage{pdfpages} % \includepdf

\usepackage{listings} % R code
\usepackage{verbatim} % verbatim

% Video stuff
\usepackage{media9}

% packages for bibs and cites
\usepackage{natbib}
\usepackage{har2nat}
\newcommand{\possessivecite}[1]{\citeauthor{#1}'s \citeyearpar{#1}}
\usepackage{breakcites}
\usepackage{alltt}

% Setup math operators
\DeclareMathOperator{\E}{E} \DeclareMathOperator{\tr}{tr} \DeclareMathOperator{\se}{se} \DeclareMathOperator{\I}{I} \DeclareMathOperator{\sign}{sign} \DeclareMathOperator{\supp}{supp} \DeclareMathOperator{\plim}{plim}
\DeclareMathOperator*{\dlim}{\mathnormal{d}\mkern2mu-lim}
\newcommand\independent{\protect\mathpalette{\protect\independenT}{\perp}}
   \def\independenT#1#2{\mathrel{\rlap{$#1#2$}\mkern2mu{#1#2}}}
\newcommand*\colvec[1]{\begin{pmatrix}#1\end{pmatrix}}

\newcommand{\myurlshort}[2]{\href{#1}{\textcolor{gray}{\textsf{#2}}}}


\begin{document}

\imageframe{./lecture_includes/mixtape_did_cover.png}


% ---- Content ----

\section{Jagged Frontier Paper}

\begin{frame}{Overview}

\begin{itemize}
\item Discuss the "jagged frontier" paper and the "gen AI at work" paper
\item Goals
	\begin{enumerate}
	\item What is the methodologies of each paper
	\item What is the data of each paper
	\item What is the population of each paper
	\item What is the AI intervention in each paper
	\item What is the core "headline" findings of each paper
	\item What tasks can be automated and what tasks could not at the time of the study not be
	\end{enumerate}
\item We will discuss, not just lecture
\end{itemize}

\end{frame}

\begin{frame}{Dell'Aqua et al 2023}

\begin{itemize}
\item Dell'Aqua et al 2023, or "jagged frontier" is a randomized experiment involving Boston Consulting Group
	\begin{enumerate}
	\item Describe the experiment -- what exactly was the experiment?
	\item Who is Boston Consulting Group?
	\end{enumerate}
\item Randomized experiments are considered the gold standard in causal inference because they eliminate something called selection bias
\item Comparing two groups with one another is the sum of two factors
	\begin{itemize}
	\item Average effect of some intervention or choice like "using ChatGPT for work"
	\item Fundamental differences that were always different between the two groups of people (those who did and didn't use it)
	\end{itemize}
\item Randomization, though, "deletes" the selection bias -- let me illustrate using ChatGPT and python
\end{itemize}

\end{frame}

\begin{frame}{Jagged Frontier experiment}

\begin{itemize}
\item Boston Consulting Group is a major consulting firm -- a McKenzie equivalent with billions in annual revenue employing around 30,000 people globally
\item Their experiment recruited 758 consultants which was 7\% of the consultants employed out of around 11,000 workers
\item Consultants performed around 18 tasks carefully chosen in order to represent things they thought AI could do well and things thought to not do well
	\begin{enumerate}
	\item "Inside the frontier" are things thought to be within AI's reach  back in March 2023
	\item "Outside the frontier" are things thought be "outside" AI's reach back then
	\end{enumerate}
\item Important to remember the dates and language models when you read these papers -- March 2023 was literally when ChatGPT-4 came out; not 4o, not o1, not o3
\item We have no idea if these results still hold up
\end{itemize}

\end{frame}

\begin{frame}{Jagged Frontier experiment}

\begin{itemize}
\item They note the limitations and strengths of AI in an interesting, compact way:
	\begin{enumerate}
	\item Large language models have surprising abilities, but do not come with an instruction manual, largely because even the creators don't know what it can do
	\item But it can clearly perform "real work" by humans possessing less skill than what has been traditionally required to perform those tasks
	\item But it's opaque, as a technology, and has unclear failure points
	\end{enumerate}
\item Given a lack of an instructional manual, they note that it likely will require trial and error by users -- consider Andrew's and Hayes' questions as to whether we can expect AI to have much impact. What do you think if it's opaque, and unclear where its failures are?
\end{itemize}

\end{frame}


\section{Generative AI At Work}


\begin{frame}{Working Paper on ChatBot}
\begin{center}
\includegraphics[scale=0.35]{./lecture_includes/brynn}
\end{center}
\end{frame}



\begin{frame}{Chatbots and Workers}

\begin{itemize}

\item An Fortune 500 firm released gradually a generative AI-based conversational assistant (GPT-3) chatbot to its  5,179 customer support agents 
\item These chatbots provided assistance in handling complaints
\item Very stressful job as the only time customers reached out was when they were very upset
\item It isn't a randomized experiment so they're going to estimate the effect of the adoption of the chatbot using a method called difference-in-differences

\end{itemize}

\end{frame}

\begin{frame}{Outcomes and Pictures}

\begin{itemize}

\item Main focus is on various measures of customer support agents handling of calls, which is the proxy for their productivity -- similar to the "jagged frontier" paper
\item But they also focus on high and low skill workers (heterogeneity like before)
\item Authors are going to present evidence almost entirely using event study graphs
\item They also present regression tables, but the event study graphs are very powerful

\end{itemize}

\end{frame}




\begin{frame}{Example of ChatBot}
\begin{center}
\includegraphics[scale=0.35]{./lecture_includes/brynn1}
\end{center}
\end{frame}


\begin{frame}{Rollout}
\begin{center}
\includegraphics[scale=0.35]{./lecture_includes/brynn2}
\end{center}
\end{frame}


\begin{frame}{Resolutions of Customer Problems}
\begin{center}
\includegraphics[scale=0.25]{./lecture_includes/brynn3}
\end{center}
\end{frame}


\begin{frame}{Additional Outcomes}
\begin{center}
\includegraphics[scale=0.35]{./lecture_includes/brynn4}
\end{center}
\end{frame}

\begin{frame}{Heterogeneity by Skill}
\begin{center}
\includegraphics[scale=0.25]{./lecture_includes/brynn5}
\end{center}
\end{frame}


\begin{frame}{Heterogeneity by Skill}
\begin{center}
\includegraphics[scale=0.25]{./lecture_includes/brynn6}
\end{center}
\end{frame}

\begin{frame}{Heterogeneity by Worker Tenure}
\begin{center}
\includegraphics[scale=0.35]{./lecture_includes/brynn7}
\end{center}
\end{frame}



\begin{frame}{Sentiment}
\begin{center}
\includegraphics[scale=0.35]{./lecture_includes/brynn8}
\end{center}
\end{frame}


\begin{frame}{Sentiment}
\begin{center}
\includegraphics[scale=0.3]{./lecture_includes/brynn11}
\end{center}
\end{frame}


\begin{frame}{Manager Assistance}
\begin{center}
\includegraphics[scale=0.25]{./lecture_includes/brynn9}
\end{center}
\end{frame}


\begin{frame}{Outcomes}

\begin{itemize}

\item Across many dimensions, worker productivity rose
\item And the productivity increases were higher for the least skilled workers -- just like we had seen in the experiment
\item They suggest that generative AI ``reallocated experience'' to the least experienced workers making them essentially appear as though they had been their awhile
\item Findings suggest that it improves customer sentiment, reduces requests for managerial intervention, and improves employee retention
\item Still unclear how generalizeable this is, and what impact we should see on overall aggregate employment as this was AI assisted, not AI alone

\end{itemize}

\end{frame}


\section{Concluding remarks}


\begin{frame}{Discussion questions}

\begin{itemize}

\item What do you think "redistributing experience" might mean?
\item What do you think this might mean for "the returns to experience and skill?"  Is experience and skill as rewarded?
\end{itemize}

\end{frame}


\end{document}

