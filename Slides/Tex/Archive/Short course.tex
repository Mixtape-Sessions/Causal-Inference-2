\documentclass{beamer}

% xcolor and define colors -------------------------
\usepackage{xcolor}

% https://www.viget.com/articles/color-contrast/
\definecolor{purple}{HTML}{5601A4}
\definecolor{navy}{HTML}{0D3D56}
\definecolor{ruby}{HTML}{9a2515}
\definecolor{alice}{HTML}{107895}
\definecolor{daisy}{HTML}{EBC944}
\definecolor{coral}{HTML}{F26D21}
\definecolor{kelly}{HTML}{829356}
\definecolor{cranberry}{HTML}{E64173}
\definecolor{jet}{HTML}{131516}
\definecolor{asher}{HTML}{555F61}
\definecolor{slate}{HTML}{314F4F}

% Mixtape Sessions
\definecolor{picton-blue}{HTML}{00b7ff}
\definecolor{violet-red}{HTML}{ff3881}
\definecolor{sun}{HTML}{ffaf18}
\definecolor{electric-violet}{HTML}{871EFF}

% Main theme colors
\definecolor{accent}{HTML}{00b7ff}
\definecolor{accent2}{HTML}{871EFF}
\definecolor{gray100}{HTML}{f3f4f6}
\definecolor{gray800}{HTML}{1F292D}


% Beamer Options -------------------------------------

% Background
\setbeamercolor{background canvas}{bg = white}

% Change text margins
\setbeamersize{text margin left = 15pt, text margin right = 15pt} 

% \alert
\setbeamercolor{alerted text}{fg = accent2}

% Frame title
\setbeamercolor{frametitle}{bg = white, fg = jet}
\setbeamercolor{framesubtitle}{bg = white, fg = accent}
\setbeamerfont{framesubtitle}{size = \small, shape = \itshape}

% Block
\setbeamercolor{block title}{fg = white, bg = accent2}
\setbeamercolor{block body}{fg = gray800, bg = gray100}

% Title page
\setbeamercolor{title}{fg = gray800}
\setbeamercolor{subtitle}{fg = accent}

%% Custom \maketitle and \titlepage
\setbeamertemplate{title page}
{
    %\begin{centering}
        \vspace{20mm}
        {\Large \usebeamerfont{title}\usebeamercolor[fg]{title}\inserttitle}\\
        {\large \itshape \usebeamerfont{subtitle}\usebeamercolor[fg]{subtitle}\insertsubtitle}\\ \vspace{10mm}
        {\insertauthor}\\
        {\color{asher}\small{\insertdate}}\\
    %\end{centering}
}

% Table of Contents
\setbeamercolor{section in toc}{fg = accent!70!jet}
\setbeamercolor{subsection in toc}{fg = jet}

% Button 
\setbeamercolor{button}{bg = accent}

% Remove navigation symbols
\setbeamertemplate{navigation symbols}{}

% Table and Figure captions
\setbeamercolor{caption}{fg=jet!70!white}
\setbeamercolor{caption name}{fg=jet}
\setbeamerfont{caption name}{shape = \itshape}

% Bullet points

%% Fix left-margins
\settowidth{\leftmargini}{\usebeamertemplate{itemize item}}
\addtolength{\leftmargini}{\labelsep}

%% enumerate item color
\setbeamercolor{enumerate item}{fg = accent}
\setbeamerfont{enumerate item}{size = \small}
\setbeamertemplate{enumerate item}{\insertenumlabel.}

%% itemize
\setbeamercolor{itemize item}{fg = accent!70!white}
\setbeamerfont{itemize item}{size = \small}
\setbeamertemplate{itemize item}[circle]

%% right arrow for subitems
\setbeamercolor{itemize subitem}{fg = accent!60!white}
\setbeamerfont{itemize subitem}{size = \small}
\setbeamertemplate{itemize subitem}{$\rightarrow$}

\setbeamertemplate{itemize subsubitem}[square]
\setbeamercolor{itemize subsubitem}{fg = jet}
\setbeamerfont{itemize subsubitem}{size = \small}


% Special characters

\usepackage{collectbox}

\makeatletter
\newcommand{\mybox}{%
    \collectbox{%
        \setlength{\fboxsep}{1pt}%
        \fbox{\BOXCONTENT}%
    }%
}
\makeatother





% Links ----------------------------------------------

\usepackage{hyperref}
\hypersetup{
  colorlinks = true,
  linkcolor = accent2,
  filecolor = accent2,
  urlcolor = accent2,
  citecolor = accent2,
}


% Line spacing --------------------------------------
\usepackage{setspace}
\setstretch{1.1}


% \begin{columns} -----------------------------------
\usepackage{multicol}


% Fonts ---------------------------------------------
% Beamer Option to use custom fonts
\usefonttheme{professionalfonts}

% \usepackage[utopia, smallerops, varg]{newtxmath}
% \usepackage{utopia}
\usepackage[sfdefault,light]{roboto}

% Small adjustments to text kerning
\usepackage{microtype}



% Remove annoying over-full box warnings -----------
\vfuzz2pt 
\hfuzz2pt


% Table of Contents with Sections
\setbeamerfont{myTOC}{series=\bfseries, size=\Large}
\AtBeginSection[]{
        \frame{
            \frametitle{Roadmap}
            \tableofcontents[current]   
        }
    }


% Tables -------------------------------------------
% Tables too big
% \begin{adjustbox}{width = 1.2\textwidth, center}
\usepackage{adjustbox}
\usepackage{array}
\usepackage{threeparttable, booktabs, adjustbox}
    
% Fix \input with tables
% \input fails when \\ is at end of external .tex file
\makeatletter
\let\input\@@input
\makeatother

% Tables too narrow
% \begin{tabularx}{\linewidth}{cols}
% col-types: X - center, L - left, R -right
% Relative scale: >{\hsize=.8\hsize}X/L/R
\usepackage{tabularx}
\newcolumntype{L}{>{\raggedright\arraybackslash}X}
\newcolumntype{R}{>{\raggedleft\arraybackslash}X}
\newcolumntype{C}{>{\centering\arraybackslash}X}

% Figures

% \imageframe{img_name} -----------------------------
% from https://github.com/mattjetwell/cousteau
\newcommand{\imageframe}[1]{%
    \begin{frame}[plain]
        \begin{tikzpicture}[remember picture, overlay]
            \node[at = (current page.center), xshift = 0cm] (cover) {%
                \includegraphics[keepaspectratio, width=\paperwidth, height=\paperheight]{#1}
            };
        \end{tikzpicture}
    \end{frame}%
}

% subfigures
\usepackage{subfigure}


% Highlight slide -----------------------------------
% \begin{transitionframe} Text \end{transitionframe}
% from paulgp's beamer tips
\newenvironment{transitionframe}{
    \setbeamercolor{background canvas}{bg=accent!40!black}
    \begin{frame}\color{accent!10!white}\LARGE\centering
}{
    \end{frame}
}


% Table Highlighting --------------------------------
% Create top-left and bottom-right markets in tabular cells with a unique matching id and these commands will outline those cells
\usepackage[beamer,customcolors]{hf-tikz}
\usetikzlibrary{calc}
\usetikzlibrary{fit,shapes.misc}

% To set the hypothesis highlighting boxes red.
\newcommand\marktopleft[1]{%
    \tikz[overlay,remember picture] 
        \node (marker-#1-a) at (0,1.5ex) {};%
}
\newcommand\markbottomright[1]{%
    \tikz[overlay,remember picture] 
        \node (marker-#1-b) at (0,0) {};%
    \tikz[accent!80!jet, ultra thick, overlay, remember picture, inner sep=4pt]
        \node[draw, rectangle, fit=(marker-#1-a.center) (marker-#1-b.center)] {};%
}

\usepackage{breqn} % Breaks lines

\usepackage{amsmath}
\usepackage{mathtools}

\usepackage{pdfpages} % \includepdf

\usepackage{listings} % R code
\usepackage{verbatim} % verbatim

% Video stuff
\usepackage{media9}

% packages for bibs and cites
\usepackage{natbib}
\usepackage{har2nat}
\newcommand{\possessivecite}[1]{\citeauthor{#1}'s \citeyearpar{#1}}
\usepackage{breakcites}
\usepackage{alltt}

% Setup math operators
\DeclareMathOperator{\E}{E} \DeclareMathOperator{\tr}{tr} \DeclareMathOperator{\se}{se} \DeclareMathOperator{\I}{I} \DeclareMathOperator{\sign}{sign} \DeclareMathOperator{\supp}{supp} \DeclareMathOperator{\plim}{plim}
\DeclareMathOperator*{\dlim}{\mathnormal{d}\mkern2mu-lim}
\newcommand\independent{\protect\mathpalette{\protect\independenT}{\perp}}
   \def\independenT#1#2{\mathrel{\rlap{$#1#2$}\mkern2mu{#1#2}}}
\newcommand*\colvec[1]{\begin{pmatrix}#1\end{pmatrix}}

\newcommand{\myurlshort}[2]{\href{#1}{\textcolor{gray}{\textsf{#2}}}}


\begin{document}

\imageframe{./lecture_includes/mixtape_did_cover.png}


% ---- Content ----

\section{Background}

\subsection{Introduction}

\begin{frame}{Introduction}

\begin{itemize}
\item Welcome to a short introductory workshop on difference-in-differences 
\item 3 hour lecture, 10:00am to 1:00pm
\item Lecture, discussion, exercises, application
\end{itemize}

\end{frame}


\begin{frame}{Workshop outline}

Introduction to DiD basics 
	\begin{itemize}
	\item Potential outcomes review
	\item DiD equation and estimation with OLS
	\item Evaluating parallel trends with falsifications, event studies 
	\item Triple differences
	\item Including covariates (probably won't have time)
	\end{itemize}

\end{frame}

\begin{frame}{Important caveat}

\begin{itemize}
\item Causal inference can feel overwhelming for even the most seasoned person
\item Don't let econometrics ever become a substitute for hard work -- they aren't the same
\item Learn everything you can about the subject, not just the econometrics
\item Talk to people, read books, read articles, learn the history -- don't just read econometrics as there's no magic inside an econometrics estimator as we saw
\item Econometrics is a tool, but it's only a tool; it doesn't replace your brain and heart
\end{itemize}

\end{frame}




\begin{frame}{Natural experiments}

\begin{quote}
``A good way to do econometrics is to look for good natural experiments and use statistical methods that can tidy up the confounding factors that nature has not controlled for us.'' -- Daniel McFadden (Nobel Laureate recipient with Heckman 1992)
\end{quote}

\end{frame}


\imageframe{./lecture_includes/bumpersticker.jpeg}






\begin{frame}{What is difference-in-differences (DiD)}

\begin{itemize}
\item DiD is a very old, relatively straightforward, intuitive research design
\item A group of units are assigned some treatment and then compared to a group of units that weren't
\item One of the most widely used quasi-experimental methods in economics and even used in industry
\item Historically used with ``big shocks'' happening in space over time but there are exceptions
\end{itemize}

\end{frame}


\begin{frame}

	\begin{figure}
	\caption{Currie, et al. (2020)}
	\includegraphics[scale=0.25]{./lecture_includes/currie_did.png}
	\end{figure}


\end{frame}








\begin{frame}{Origins in Economics and Public Health}

\begin{itemize}
\item In economics, David Card and Orley Ashenfelter are often associated with its origin -- used in the 1970s and 1980s, with mixed success, to study job training programs
\item Their dissatisfaction with it led to a call for more randomized controlled trials because, as we will see, it is not able to solve every kind of problem
\item But it predates economics by 150 years when two health scientists (separately) used it to prove disease transmission mechanisms

\end{itemize}

\end{frame}

\begin{frame}{Case I: Ignaz Semmelweis and washing hands}

\begin{itemize}
\item 1840s, Vienna maternity wards had high postpartum infections in one wing compared to other wings
\item One division had doctors and trainee doctors, but another had midwives and trainee midwives
\end{itemize}

\end{frame}

\begin{frame}{Case I: Ignaz Semmelweis and washing hands}

\begin{itemize}
\item Training hospitals of students had earlier moved to ``anatomical'' training involving cadavers for classes
\item Semmelweis thinks the mortality is caused by working with cadavers
\item Proposes washing hands with chlorine in 1847 in the midwives' wing (but not the physician wing) 
\item Mortality converged to the same levels in the two wings
\end{itemize}

\end{frame}



\begin{frame}{Case II: John Snow and cholera}

\begin{itemize}
\item Three major waves of cholera in the early to mid 1800s in London and people mistakenly thought the cause was ``smelly air'' (or \emph{miasma})
\item John Snow believed cholera was spread through water and food which entered the Thames river through host's evacuations
\item Provides a variety of evidence for this, including a beautiful map, but also takes advantage of a natural experiment
\item Lambeth water company moves its pipe between 1849 and 1854 but Southwark and Vauxhall delays
\end{itemize}

\end{frame}


\begin{frame}

	\begin{figure}
	\caption{Two water utility companies in London 1854}
	\includegraphics[scale=0.225]{./lecture_includes/lambeth.png}
	\end{figure}


\end{frame}


\begin{frame}{Three ways to study this}

\begin{enumerate}
\item Simple cross-section: compare mortality in 1854 for the two neighborhoods
\item Before and after: also called interrupted time series. Compare Lambeth in 1854 to 1849
\item Difference-in-differences: do both
\end{enumerate}

\end{frame}



\begin{frame}{3) Difference-in-differences}

\begin{table}\centering
		\caption{Lambeth and Southwark and Vauxhall, 1849 and 1854}
		\begin{center}
		\begin{tabular}{lll|lc}
		\toprule
		\multicolumn{1}{l}{\textbf{Companies}}&
		\multicolumn{1}{c}{\textbf{Time}}&
		\multicolumn{1}{c}{\textbf{Outcome}}&
		\multicolumn{1}{c}{$D_1$}&
		\multicolumn{1}{c}{$D_2$}\\
		\midrule
		Lambeth & Before & $Y=L$ \\
		& After & $Y=L + \textcolor{red}{T_L} + \textcolor{blue}{D}$ & $\textcolor{red}{T_L}+\textcolor{blue}{D}$\\
		\midrule
		& & & & $\textcolor{blue}{D}$ \\
		\midrule
		Southwark and Vauxhall & Before & $Y=SV$ \\
		& After & $Y=SV + T_{SV}$ & $T_{SV}$\\
		\bottomrule
		\end{tabular}
		\end{center}
	\end{table}

\begin{eqnarray*}
\widehat{\delta}_{did} = \textcolor{blue}{D} + (\textcolor{red}{T_L} - T_{SV})
\end{eqnarray*}\textcolor{blue}{D} is the ``treatment effect''.  It's the effect of moving the water on Lambeth mortality and it's in blue because we can't see it 

\end{frame}

\begin{frame}{3) Difference-in-differences}

\begin{table}\centering
		\caption{Lambeth and Southwark and Vauxhall, 1849 and 1854}
		\begin{center}
		\begin{tabular}{lll|lc}
		\toprule
		\multicolumn{1}{l}{\textbf{Companies}}&
		\multicolumn{1}{c}{\textbf{Time}}&
		\multicolumn{1}{c}{\textbf{Outcome}}&
		\multicolumn{1}{c}{$D_1$}&
		\multicolumn{1}{c}{$D_2$}\\
		\midrule
		Lambeth & Before & $Y=L$ \\
		& After & $Y=L + \textcolor{red}{T_L} + \textcolor{blue}{D}$ & $\textcolor{red}{T_L}+\textcolor{blue}{D}$\\
		\midrule
		& & & & $\textcolor{blue}{D}$ \\
		\midrule
		Southwark and Vauxhall & Before & $Y=SV$ \\
		& After & $Y=SV + T_{SV}$ & $T_{SV}$\\
		\bottomrule
		\end{tabular}
		\end{center}
	\end{table}

\begin{eqnarray*}
\widehat{\delta}_{did} = \textcolor{blue}{D} + (\textcolor{red}{T_L} - T_{SV})
\end{eqnarray*}\textcolor{red}{$T_L$} is a natural change in mortality that would have happened had they not moved the pipe upstream.  It creates problems for us

\end{frame}



\begin{frame}{3) Difference-in-differences}

\begin{table}\centering
		\caption{Lambeth and Southwark and Vauxhall, 1849 and 1854}
		\begin{center}
		\begin{tabular}{lll|lc}
		\toprule
		\multicolumn{1}{l}{\textbf{Companies}}&
		\multicolumn{1}{c}{\textbf{Time}}&
		\multicolumn{1}{c}{\textbf{Outcome}}&
		\multicolumn{1}{c}{$D_1$}&
		\multicolumn{1}{c}{$D_2$}\\
		\midrule
		Lambeth & Before & $Y=L$ \\
		& After & $Y=L + \textcolor{red}{T_L} + \textcolor{blue}{D}$ & $\textcolor{red}{T_L}+\textcolor{blue}{D}$\\
		\midrule
		& & & & $D$ \\
		\midrule
		Southwark and Vauxhall & Before & $Y=SV$ \\
		& After & $Y=SV + T_{SV}$ & $T_{SV}$\\
		\bottomrule
		\end{tabular}
		\end{center}
	\end{table}

\begin{eqnarray*}
\widehat{\delta}_{did} = \textcolor{blue}{D} + (\textcolor{red}{T_L} - T_{SV})
\end{eqnarray*}But, if $\textcolor{red}{T_L}=\textcolor{black}{T_{SV}}$, which is called ``parallel trends'', then we can identify $\textcolor{blue}{D}$ using DiD.  And that's DiD in a nutshell. 

\end{frame}



\subsection{Potential outcomes}



\begin{frame}{Potential outcomes review}

\begin{itemize}
\item Previous table was how I learned DiD the first time, but now we need to become more formalized
\item The following notation is not often taught in our introductory econometrics courses which tend to focus on regressions first and causality second
	\begin{itemize}
	\item We will focus on causality first, regressions second
	\end{itemize}
\item This is a simple review of the potential outcomes model by Jerzy Neyman (1923) and Don Rubin (1973)
\item Potential outcomes notation is the dominant language of causality, though there are others too (e.g., Pearl)
\end{itemize}

\end{frame}
\begin{frame}{Potential outcomes notation}
	
	\begin{itemize}
	\item Let the treatment be a binary variable: $$D_{i,t} =\begin{cases} 1 \text{ if pipe inlet is upstream at time $t$} \\ 0 \text{ if pipe inlet is downstream at time $t$} \end{cases}$$where $i$ indexes an individual observation, such as a person

	\end{itemize}
\end{frame}

\begin{frame}{Potential outcomes notation}
	
	\begin{itemize}

	\item Potential outcomes: $$Y_{i,t}^j =\begin{cases} 1 \text{: health if drank from upstream at time $t$} \\ 0 \text{: health if drank from downstream at time $t$} \end{cases}$$where $j$ indexes a counterfactual state of the world

	\end{itemize}
\end{frame}


\begin{frame}{Potential vs realized}

\begin{itemize}
\item Distinction between the potential outcome $Y^1$ and the realized outcome $Y$ -- one is hypothetical and the other is real
\item Potential outcomes are ``selected'' to become real when people choose their treatments represented here with a ``switching equation'' $$Y_{it}=D_{it}Y_{it}^1 + (1-D_{it})Y_{it}^0$$
\item Example: My wages if I go to college are $Y^1$ and my wages if I don't go to college are $Y^0$, but since I went to college ($D=1$), my wages are $Y=Y^1$.
\item Point here is we define causality using potential outcomes, but data is realized outcomes, which creates problems

\end{itemize}
\end{frame}



\begin{frame}{Treatment effect definitions}


	\begin{block}{Individual treatment effect}
	    The individual treatment effect,  $\delta_i$, equals $Y_i^1-Y_i^0$
	\end{block}

Core building block of causal inference is the individual treatment effect. 
	
\end{frame}


\begin{frame}{Conditional Average Treatment Effects}	
	\begin{block}{Average Treatment Effect on the Treated (ATT)}
	The average treatment effect on the treatment group is equal to the average treatment effect conditional on being a treatment group member:
		\begin{eqnarray*}
		E[\delta|D=1]&=&E[Y^1-Y^0|D=1] \nonumber \\
		&=&E[Y^1|D=1]-\textcolor{red}{E[Y^0|D=1]}
		\end{eqnarray*}
	\end{block}
	
	\bigskip
	
Since each person has an individual treatment effect, we can summarize them any number of ways -- average treatment effect for girls, for old people, for people who like trivia night. Or for people who live in the Lambeth neighborhood.  

	
\end{frame}

\begin{frame}{Conditional Average Treatment Effects}	
	\begin{block}{Average Treatment Effect on the Treated (ATT)}
	The average treatment effect on the treatment group is equal to the average treatment effect conditional on being a treatment group member:
		\begin{eqnarray*}
		E[\delta|D=1]&=&E[Y^1-Y^0|D=1] \nonumber \\
		&=&E[Y^1|D=1]-\textcolor{red}{E[Y^0|D=1]}
		\end{eqnarray*}
	\end{block}
	
	\bigskip
	
Why is the first term black but the second term \textcolor{red}{red}?

	
\end{frame}

\section{Identification and Estimation}

\subsection{Parallel trends}

\begin{frame}{Fundamental problem of causal inference}

\begin{itemize}
\item Potential outcomes reference a causality concept called the counterfactual
\item Individual treatment effect is defined by comparing your outcome today with a college degree to your outcome today without a college degree
	\begin{itemize}
	\item Obviously those can't both be true -- either you have a degree today or you don't
	\end{itemize}
\item Parameters (i.e., the causal effects, $\delta$) are different from estimates of those parameters (i.e., $\widehat{\delta}$)
\end{itemize}

\end{frame}

\begin{frame}{Steps of your causal projects}

\begin{enumerate}
\item Define the parameter we want (``ATT''), 
\item Ask what what beliefs do you need (``identification''), and 
\item Build cranks that produce the correct numbers (``estimator'')
\end{enumerate}

\bigskip

People often skip 1 and 2 and go straight to 3 and run regressions then go back and assume exogeneity (step 2), and hope that the estimates are weighted averages of individual treatment effects (1), but that is not guaranteed

\end{frame}



\begin{frame}{DiD is four averages and three differences}

I call this the DiD equation, but Goodman-Bacon calls it the ``2x2''; I'll use his $k$ and $U$ notation for treated and untreated groups

\begin{eqnarray*}
\widehat{\delta}^{2x2}_{kU} = \bigg ( E[Y_k|Post] - E[Y_k|Pre] \bigg ) - \bigg ( E[Y_U | Post ] - E[ Y_U | Pre] \bigg) \\
\end{eqnarray*}$k$ index people with Lambeth, $U$ index people with Southwark and Vauxhall, $Post$ is after Lambeth moved pipe upstream, $Pre$ before Lambeth moved its pipe (baseline), and $E[y]$ mean cholera mortality. 

\end{frame}

\begin{frame}{DiD is four averages and three differences}

``Pre'' (1849) and ``Post'' (1854) refer to when Lambeth, $k$, was treated which is why it is the same for both $k$ and $U$ groups

\begin{eqnarray*}
\widehat{\delta}^{2x2}_{kU} = \bigg ( E[Y_k|Post] - E[Y_k|Pre] \bigg ) - \bigg ( E[Y_U | Post ] - E[ Y_U | Pre] \bigg) \\
\end{eqnarray*}

Since we have one treatment group, then ``Pre'' and ``Post'' reference Lambeth's treatment date

\end{frame}


\begin{frame}{Potential outcomes and the switching equation}

\begin{eqnarray*}
\widehat{\delta}^{2x2}_{kU} &=& \bigg ( \underbrace{E[Y^1_k|Post] - E[Y^0_k|Pre] \bigg ) - \bigg ( E[Y^0_U | Post ] - E[ Y^0_U | Pre]}_{\mathclap{\text{Replace potential outcomes with realized outcomes using switching equation}}} \bigg)  \\
&&+ \underbrace{\textcolor{red}{E[Y_k^0 |Post] - E[Y^0_k | Post]}}_{\mathclap{\text{Adding zero}}} 
\end{eqnarray*}

\end{frame}

\begin{frame}{Parallel trends bias}

Rearrange and we get this:

\begin{eqnarray*}
\widehat{\delta}^{2x2}_{kU} &=& \underbrace{E[Y^1_k | Post] - \textcolor{red}{E[Y^0_k | Post]}}_{\mathclap{\text{ATT}}} \\
&& + \bigg [  \underbrace{\textcolor{red}{E[Y^0_k | Post]} - E[Y^0_k | Pre] \bigg ] - \bigg [ E[Y^0_U | Post] - E[Y_U^0 | Pre] }_{\mathclap{\text{Non-parallel trends bias in 2x2 case}}} \bigg ]
\end{eqnarray*}


\end{frame}

\begin{frame}{Parallel trends bias}

\begin{eqnarray*}
\widehat{\delta}^{2x2}_{kU} &=& \underbrace{E[Y^1_k | Post] - \textcolor{red}{E[Y^0_k | Post]}}_{\mathclap{\text{ATT}}} \\
&& + \bigg [  \underbrace{\textcolor{red}{E[Y^0_k | Post]} - E[Y^0_k | Pre] \bigg ] - \bigg [ E[Y^0_U | Post] - E[Y_U^0 | Pre] }_{\mathclap{\text{Non-parallel trends bias in 2x2 case}}} \bigg ]
\end{eqnarray*}

\bigskip

The left hand side is our DiD estimator (i..e, four averages, three differences); the right hand side has our parameter (top) and assumption (parallel trends, bottom).  Recall from the earlier table how DiD was equal to $D+(\textcolor{red}{T_L} - T_{SV})$.  That's this.

\end{frame}


\begin{frame}{Identification through parallel trends}
	

	\begin{block}{Parallel trends}
	Assume two groups, treated and comparison group, then we define parallel trends as:	 $$\textcolor{red}{E(}\textcolor{red}{\Delta Y^0_k)} = E(\Delta Y^0_U)$$
	\end{block}

\textbf{In words}: ``The \textcolor{red}{evolution of cholera mortality for Lambeth \emph{had it kept its pipe downstream}} is the same as the evolution of cholera mortality for Southwark and Vauxhall''.  

\bigskip

It's in \textcolor{red}{red} so you know it's a nontrivial assumption.  But why?  Can't we just check?

	

	
\end{frame}


\begin{frame}{Group work}

\begin{itemize}
\item Before we move into regression, let's go through a simple exercise to really pin down these core ideas with simple calculations
\item I've passed around a worksheet.  We will spend some time together doing this worksheet; use the online or use the worksheet
\end{itemize}

\bigskip 

\url{https://docs.google.com/spreadsheets/d/1onabpc14JdrGo6NFv0zCWo-nuWDLLV2L1qNogDT9SBw/edit?usp=sharing}

\end{frame}



\subsection{Estimation with OLS specification}

\begin{frame}{OLS Specification}
	
	\begin{itemize}
	\item Simple DiD equation (four averages, three differences) estimates ATT under parallel trends; don't need regression
	\item But there is an OLS specification that is numerically identical to four averages and three differences
	\item OLS was historically preferred because
		\begin{itemize}
		\item OLS estimates the ATT under parallel trends so it is valid
		\item Easy to calculate the standard errors
		\item Easy to include multiple periods which increases power and makes estimates more precise
		\end{itemize}
	\item This specification is not appropriate under differential timing or with the inclusion of covariates
	\end{itemize}
\end{frame}

\begin{frame}{Minimum wages}

\begin{itemize}
\item Card and Krueger (1994) have a famous study estimating causal effect (ATT) of minimum wages on employment
\item Exploited a policy change in New Jersey between February and November in mid-1990s where minimum wage was increased, but neighbor PA did not
\item Using DiD, they do not find a negative effect of the minimum wage on employment which is part of its legacy today, but I mainly present it to illustrate the history and the design principles
\end{itemize}

\end{frame}

\begin{frame}
	\begin{figure}
	\includegraphics[scale=0.5]{./lecture_includes/minwage_whore}
	\end{figure}
\end{frame}

\begin{frame}{Quick comment}

\begin{itemize}
\item Buchanan's comment gets taken out of historical context to a degree
\item Empirical labor and empirical macroeconomics (e.g., Lucas Critique) had been going back to the 1970s in a bit of a ``empirical crisis'' much like we see sometimes today with debates about p-hacking, but theirs was more basic confusion of causality and correlation
\item Consequently, the dominant paradigm in ``knowing facts in economics'' was theory, not empiricism
\item So Buchanan's dismissiveness probably had traces of that; quality of empirical work was sub standard so people tended to not take it very seriously
\end{itemize}

\end{frame}


\begin{frame}{Card on that study}

\begin{quote}
``I’ve subsequently stayed away from the minimum wage literature for a number of reasons. First, it cost me a lot of friends. People that I had known for many years, for instance, some of the ones I met at my first job at the University of Chicago, became very angry or disappointed. They thought that in publishing our work we were being traitors to the cause of economics as a whole.''
\end{quote}

\bigskip

But let's listen to Orley's opinion about the paper's controversy at the time.  \url{https://youtu.be/MOtbuRX4eyQ?t=1882}

\end{frame}



\begin{frame}{OLS specification of the DiD equation}
	
	\begin{itemize}
	\item The correctly specified OLS regression is an interaction with time and group fixed effects:$$Y_{its} = \alpha + \gamma NJ_s + \lambda d_t + \delta (NJ \times d)_{st} + \varepsilon_{its}$$
		\begin{itemize}
		\item NJ is a dummy equal to 1 if the observation is from NJ
		\item d is a dummy equal to 1 if the observation is from November (the post period)
		\end{itemize}
	\item This equation takes the following values
		\begin{itemize}
		\item PA Pre: $\alpha$
		\item PA Post: $\alpha + \lambda$
		\item NJ Pre: $\alpha + \gamma$
		\item NJ Post: $\alpha + \gamma + \lambda + \delta$
		\end{itemize}
	\item DiD equation: (NJ Post - NJ Pre) - (PA Post - PA Pre) $= \delta$
	\end{itemize}
\end{frame}




\begin{frame}[plain]
	$$Y_{ist} = \alpha + \gamma NJ_s + \lambda d_t + \delta(NJ\times d)_{st} + \varepsilon_{ist}$$
	\begin{figure}
	\includegraphics[scale=0.90]{./lecture_includes/waldinger_dd_5.pdf}
	\end{figure}
\end{frame}


\begin{frame}[plain]
	$$Y_{ist} = \alpha + \gamma NJ_s + \lambda d_t + \delta(NJ\times d)_{st} + \varepsilon_{ist}$$
	\begin{figure}
	\includegraphics[scale=0.90]{./lecture_includes/waldinger_dd_5.pdf}
	\end{figure}

Notice how OLS is ``imputing'' $E[Y^0|D=1,Post]$ for the treatment group in the post period? It is only ``correct'', though, if parallel trends is a good approximation

\end{frame}

\subsection{Inference}

\begin{frame}{Inference}
	
	\begin{itemize}
	\item  Bertrand, Duflo and Mullainathan (2004) show that conventional standard errors will often severely understate the standard deviation of the estimators
	\item Standard errors are biased downward (i.e., too small, over reject)
	\item They proposed three solutions, but most only use one of them (clustering)
	\end{itemize}
\end{frame}


\begin{frame}{Inference}
	
		\begin{enumerate}
		\item[1 ] Block bootstrapping standard errors (if you analyze states the block should be the states and you would sample whole states with replacement for bootstrapping)
		\item[2 ] Clustering standard errors at the group level (in Stata one would simply add \texttt{, cluster(state)} to the regression equation if one analyzes state level variation)
		\end{enumerate}

\bigskip

Most people will simply cluster, but there are issues if you have too few clusters. They mention a third way but it's only a curiosity.
		
\end{frame}




\section{Parallel trends violations}

\subsection{How parallel trends can get violated}


\begin{frame}{Violating parallel trends exercise}

\begin{itemize}
\item Parallel trends enable regressions to correct impute counterfactual $E[Y^0|D=1]$ using $E[Y^0|D=0]$ 
\item OLS \emph{always} imputes but it only is right if parallel trends is true
\item Which means if parallel trends isn't true, then the imputation isn't correct and therefore estimates are biased
\item To illustrate this, let's go through the document again (tab is ``DID 2'')
\end{itemize}

\url{https://docs.google.com/spreadsheets/d/1onabpc14JdrGo6NFv0zCWo-nuWDLLV2L1qNogDT9SBw/edit?usp=sharing}

\end{frame}


\begin{frame}{Violating parallel trends}

\begin{itemize}
\item Parallel trends are in expectation only -- we don't rely everybody to follow the same trend, just that the group average for $Y^0$ be approximately the same for treated and control 
\item Violations are a form of selection bias and there are two straightforward ways that parallel trends will be violated
	\begin{enumerate}
	\item Compositional differences in samples associated with repeated cross-sections
	\item Treatment coincided with some other important event that only affected treatment group
	\end{enumerate}
\end{itemize}

\end{frame}



\begin{frame}{Repeated cross-sections and compositional change}
	
	\begin{itemize}
	\item One of the risks of a repeated cross-section is that the composition of the sample may have changed between the pre and post period in ways that are correlated with treatment
	\item Hong (2013) uses repeated cross-sectional data from the Consumer Expenditure Survey (CEX) containing music expenditure and internet use for a random sample of households
	\item Study exploits the emergence of Napster (first file sharing software widely used by Internet users) in June 1999 as a natural experiment
	\item Study compares internet users and internet non-users before and after emergence of Napster
	\end{itemize}

\end{frame}

\begin{frame}[plain]
	\begin{figure}
	\includegraphics{./lecture_includes/Hong_1.pdf}
	\end{figure}
	
\end{frame}

\begin{frame}[shrink=20,plain]
	\begin{figure}
	\includegraphics{./lecture_includes/Hong_2.pdf}
	\end{figure}
	
	Diffusion of the Internet changes samples (e.g., younger music fans are early adopters)
	
\end{frame}

\begin{frame}{Repeated cross-sections}

\begin{itemize}
\item Surprisingly underappreciated problem with almost no literature around it
\item So what can you do?  Check covariate balance by regressing the time-varying covariates instead of the outcome onto the treatment using your OLS specification
\item They should be exogenous remember, so this covariate regression can be a helpful test of whether this is a problem
\item Hong (2013) is the only paper at the moment that addresses this and uses a modification called ``propensity scores'' to adjust 
\end{itemize}

\end{frame}


\subsection{Types of evidence}

\begin{frame}{Think like a prosecutor}

\begin{itemize}
\item You are building a case, the prosecutor before a judge and jury, always in battle with the defense attorney
\item Evidence has particular broadly defined forms that can help you on the front end
\item Your goal in my humble opinion should be mixing tight logic based falsifications with particular kinds of data visualization, starting with the event study
\end{itemize}

\end{frame}

\begin{frame}{Three types of evidence}

\begin{enumerate}
\item \textbf{Bite}: Show that the treatment impacted first order behavior before showing how it affected second order behavior
\item \textbf{Event studies}: A particular kind of data visualization focused on pre- and post-treatment DiD coefficients in a regression equation
\item \textbf{Placebos}: Ruling out reasonable competing theories using the same regression model on different outcomes; can include triple differences
\end{enumerate}

\end{frame}


\begin{frame}{Event studies have become mandatory in DiD}

	\begin{figure}
	\includegraphics[scale=0.5]{./lecture_includes/currie_eventstudy.png}
	\end{figure}

\end{frame}

\begin{frame}{Intuition behind event studies}

\begin{itemize}
	\item Princeton Industrial Relations Section seems to be behind this -- this intense focus on research design but also verifying assumptions
	\item The identifying assumption for all DD designs is parallel trends , but since we cannot verify parallel trends, we often look at pre-trends
	\item It's a type of check for selection bias, but you must understand what it is and what it isn't to see its value but not be naive about it (it is not a silver bullet)
	\item Even if pre-trends are the same one still has to worry about other policies changing at the same time (omitted variable bias is a parallel trends violation)

\end{itemize}

\end{frame}




\begin{frame}{Plot the raw data when there's only two groups}

	\begin{figure}
	\includegraphics[scale=2.5]{./lecture_includes/waldinger_dd_6.pdf}
	\end{figure}

\end{frame}

\begin{frame}{Evidence for parallel trends: pre-trends}

Let's do the bonus questions on first and second tab now 

\url{https://docs.google.com/spreadsheets/d/1onabpc14JdrGo6NFv0zCWo-nuWDLLV2L1qNogDT9SBw/edit?usp=sharing}

\end{frame}


\begin{frame}{Pre trends as a type of evidence}

\begin{itemize}
\item Parallel pre-trends $\neq$ parallel trends -- these are often thought to be the same thing, and they aren't
\item Equating them is a kind of \emph{post hoc ergo propter hoc} fallacy
\item Parallel pre-trends is more like a smoking gun based on things ``looking the same'' before
\item Checking pre-trends is just a form of falsification, as well as giving us some assurances our groups are comparable 
\end{itemize}

\end{frame}


\begin{frame}{Event study regression}
	
	\begin{itemize}
	\item Event studies have a simple OLS specification with only one treatment group and one never-treated group $$Y_{its} = \alpha +  \sum_{\tau=-2}^{-q}\mu_{\tau}D_{s\tau} + \sum_{\tau=0}^m\delta_{\tau}D_{s\tau}+\varepsilon_{ist}$$
		\begin{itemize}
		\item where $D$ is an interaction of the treatment group $s$ with the calendar year $\tau$
		\item Treatment occurs in year 0, no anticipation, drop baseline $t-1$
		\item Includes $q$ leads or anticipatory effects and $m$ lags or post treatment effects
		\end{itemize}
	\end{itemize}
\end{frame}

\begin{frame}{Event study regression}


$$Y_{its} = \alpha + \sum_{\tau=-2}^{-q}\mu_{\tau}D_{s\tau} + \sum_{\tau=0}^m\delta_{\tau}D_{s\tau}+\varepsilon_{ist}$$

\bigskip

Typically you'll plot the coefficients and 95\% CI on all leads and lags (binned or not, trimmed or not) 

\bigskip

Under no anticipation, then you expect $\widehat{\mu}$ coefficients to be zero, which gives you confidence that parallel trends holds (but is not a guarantee, and there are still specification issues -- see Jon Roth's work)

\bigskip

Under parallel trends, $\widehat{\delta}$ are estimates of the ATT at points in time

\end{frame}



\begin{frame}{Medicaid and Affordable Care Act example}

\begin{figure}
\includegraphics[scale=0.25]{./lecture_includes/medicaid_qje}
\end{figure}

\end{frame}

\begin{frame}{Their three types of evidence}

\begin{itemize}
\item \textbf{Bite} -- show that the expansion shifted people into Medicaid and out of uninsured status
\item \textbf{Placebos} -- Show that there's no effect on mortality for groups it shouldn't be affecting (people 65+)
\item \textbf{Event study} -- Show leads and lags on mortality
\end{itemize}

\end{frame}


\imageframe{./lecture_includes/Miller_Medicaid1.png}

\imageframe{./lecture_includes/Miller_Medicaid2.png}

\imageframe{./lecture_includes/Miller_Medicaid3.png}

\begin{frame}{Quick review}

\begin{itemize}

\item \textbf{Bite}: Did the expansion of Medicaid put more people on Medicaid?
	\begin{enumerate}
	\item 40pp increase in people eligible (but this was mechanical)
	\item 6-10pp increase in people on Medicaid (but maybe it was crowding out private insurance?)
	\item 4-6pp decrease in uninsured (at least some of the marginal Medicaid enrollees had been uninsured)
	\end{enumerate}
\item \textbf{Placebo}:  their main result is about near-elderly mortality (i.e., around age 64), but first they look at the effect on actual elderly (65+)
\end{itemize}

\end{frame}


\begin{frame}{65 and older mortality placebo}

	\includegraphics[scale=0.3]{./lecture_includes/medicaid_qje_placebo}

\textbf{Discussion}: Why do they do this?  Explain to me like I'm 5 the value of a picture like this.

\end{frame}
\begin{frame}[plain]

	\begin{figure}
	\includegraphics[scale=0.3]{./lecture_includes/Miller_Medicaid4.png}
	\caption{Miller, et al. (2019) estimates of Medicaid expansion's effects on on annual mortality}
	\end{figure}

\end{frame}

\begin{frame}{Lab}

\begin{itemize}

\item We will now use Stata and R code to implement simple DiD equations manually, with OLS, and for event studies (both manually and using OLS)
\item Goal of this exercise is simply to deepen your understanding of the mechanics
\item Please go to this link \url{https://github.com/Mixtape-Sessions/Causal-Inference-2/tree/main/Lab/Lalonde}
\item Q1:a-c vs. Q2a.  We won't do the covariate part yet. 

\end{itemize}

\end{frame}

\subsection{Triple difference}

\begin{frame}{Triple differences as alternative strategy}
	
	\begin{itemize}
	\item Very common for readers and others to request a variety of ``robustness checks'' from a DD design
	\item We saw some of these just now (e.g., falsification test using data for alternative control group, the Medicare population)
	\item Triple differences uses a within-state untreated group; little trickier, so let's use the table again
	\end{itemize}
\end{frame}

\begin{frame}{DDD Example by Gruber}
	
	\begin{figure}
	\includegraphics{./lecture_includes/gruber_ddd_3.pdf}
	\end{figure}
	
\end{frame}



\begin{frame}[shrink=20]

\begin{table}\centering
		\caption{Difference-in-Difference-in-Differences numerical example}
		\tiny
		\begin{center}
		\begin{tabular}{lll|l|lll}
		\hline \hline
		\multicolumn{1}{l}{\textbf{States}}&
		\multicolumn{1}{c}{\textbf{Group}}&
		\multicolumn{1}{c}{\textbf{Period}}&
		\multicolumn{1}{c}{\textbf{Outcomes}}&
		\multicolumn{1}{c}{$D_1$}&
		\multicolumn{1}{c}{$D_2$}&
		\multicolumn{1}{c}{$D_3$}\\
		\hline
		&&After	&$NJ+T+NJ_t+l_t+D$					\\
	&Married women, 20-40yo			&&&$T+NJ_t+l_t+D$			\\
		&&Before	&$NJ$					\\
Experimental states					&&&&&$D+l_t-s_t$			\\
		&&After	&$NJ+T+NJ_t+s_t$					\\
	&Older 40, Single men 20-40yo		&&	&$T+NJ_t+s_t$				\\
		&&Before	&$NJ$					\\
								\\
&&&&&&$D$
\\
		&&After	&$PA+T+PA_t+l_t$				\\
	&Married women, 20-40yo			&&&$T+PA_t+l_t$ \\				
		&&Before	&$PA$					\\
Non-experimental states					&&&&&$l_t-s_t$		\\
		&&After	&$PA+T+PA_t+s_t$					\\
	&Older 40, Single men 20-40yo		&&&	$T+PA_t+s_t$				\\
		&&Before	&$PA$					\\
		\hline \hline
		\end{tabular}
		\end{center}
	\end{table}
	
\textbf{What is our identifying assumption?} Answer: just another parallel trend assumption but with a slightly different interpretation!

\bigskip

$l_t-s_t$ is the same for the experimental and non-experimental states. This is ``change in inequality between two groups hourly wages'' from pre to post.  It's a new parallel trend assumption.



\end{frame}


\begin{frame}{DDD in Regression}
	
	\begin{eqnarray*}
	Y_{ijt} &=&\alpha +  \beta_2 \tau_t + \beta_3 \delta_j  + \beta_4 D_i + \beta_5(\delta \times \tau)_{jt} \\
	&& +\ \beta_6(\tau \times D)_{ti} +  \beta_7(\delta \times D)_{ij} +  \textcolor{red}{\beta_8(\delta \times \tau \times  D)_{ijt}}+  \varepsilon_{ijt}
	\end{eqnarray*}
	
	\begin{itemize}
	\item Your panel is now a group $j$ state $i$ (e.g., AR high wage worker 1991, AR high wage worker 1992, etc.)
	\item Assume we drop $\tau_t$ but I just want to show it to you for now.
	\item If the placebo DD is non-zero, it might be difficult to convince the reviewer that the DDD removed all the bias 
	\end{itemize}
	
\end{frame}

\begin{frame}{Great new paper to learn more}

\begin{figure}
\includegraphics[scale=0.25]{./lecture_includes/olden_moen_2022_ddd.png}
\end{figure}

\end{frame}



\subsection{Placebo outcomes}

\begin{frame}{Falsification test with alternative outcome}
	
	\begin{itemize}
	\item The within-group control group (DDD) is a form of placebo analysis using the same \emph{outcome}
	\item But there are also placebos using a \emph{different} outcome -- but you need a hypothesis of mechanisms to figure out what is in fact a \emph{different outcome}
	\item Figure out what those are, and test them -- finding no effect on placebo outcomes tends to help people your other results interestingly enough
	\item Cheng and Hoekstra (2013) examine the effect of castle doctrine gun laws on non-gun related offenses like grand theft auto and find no evidence of an effect 
	\end{itemize}
\end{frame}



\begin{frame}{Rational addiction as a placebo critique}


Sometimes, an empirical literature may be criticized using nothing more than placebo analysis

\begin{quote}``A majority of [our] respondents believe the literature is a success story that demonstrates the power of economic reasoning.  At the same time, they also believe the empirical evidence is weak, and they disagree both on the type of evidence that would validate the theory and the policy implications. Taken together, this points to an interesting gap.  On the one hand, most of the respondents claim that the theory has valuable real world implications.  On the other hand, they do not believe the theory has received empirical support.''
\end{quote}

\end{frame}

\begin{frame}{Placebo as critique of empirical rational addiction}

\begin{itemize}
	\item Auld and Grootendorst (2004) estimated standard ``rational addiction'' models (Becker and Murphy 1988) on data with milk, eggs, oranges and apples.  
	\item They find these plausibly non-addictive goods are addictive, which casts doubt on the empirical rational addiction models.
\end{itemize}

\end{frame}

\begin{frame}{Placebo as critique of peer effects}

\begin{itemize}
	\item Several studies found evidence for ``peer effects'' involving inter-peer transmission of smoking, alcohol use and happiness tendencies
	\item Christakis and Fowler (2007) found significant network effects on outcomes like obesity
	\item Cohen-Cole and Fletcher (2008) use similar models and data and find similar network ``effects'' for things that \emph{aren't} contagious like acne, height and headaches
	\item Ockham's razor - given social interaction endogeneity (Manski 1993), homophily more likely explanation
\end{itemize}

\end{frame}



\section{Including Covariates}

\subsection{Inverse probability weighting}

\begin{frame}{Controls}

\begin{itemize}
\item Controls can address omitted variable bias (backdoor criterion), and they can improve precision
\item OLS can accommodate controls, and so we tend to include them so long as they are time varying 
\item But unfortunately, time varying covariates can create problems, especially if the treatment causes the covariates (bad controls, colliders)
\end{itemize}

\end{frame}






\begin{frame}{Inverse probability weighting DiD}

 Abadie (2005) incorporates baseline covariates into the propensity score which are then used as weights to estimate the ATT in a simple 3-step process
	\begin{enumerate}
	\item Calculate each unit's ``after minus before'' (DiD equation)
	\item Estimate the conditional probability of treatment based on baseline covariates (propensity score estimation)
	\item Weight the comparison group's DiD equation with the IPW
	\end{enumerate}

\end{frame}

\begin{frame}{Terms}

\begin{itemize}
\item $t$ is year of treatment which doesn't vary across units (so no differential timing)
\item $Y^1$ and $Y^0$ are potential outcomes (counterfactual versus actual)
\item $D$ is 1 or 0 based on group and time
\item $X_b$ are ``baseline'' covariates \textbf{only} -- they do not vary over time, which means propensity scores are estimated off the $b$ period \textbf{only}
\end{itemize}

\end{frame}

\begin{frame}{Assumptions}

Kind of common for this propensity score literature to only have two assumptions.  But usually the first conditional independence.  Now it is parallel trends because this is DD

\begin{enumerate}
\item Conditional parallel trends $$E[Y^0_t - Y^0_b|D=1,X_b] - E[Y^0_t - Y^0_t | D=0, X_b]$$ (Notice the $b$ subscript.  What is that you think?)
\item Common support $$Pr(D=1)>0; Pr(D=1|X)<1$$ Let's see a picture of common support that I drew.  Apologies it's horrible
\end{enumerate}

\end{frame}

\begin{frame}{Common support}

As we are identifying the ATT, we only need common support with respect to treated units

\bigskip

Your identify assumptions are always with respect to the missing covariates in other words and for the ATT, you are missing $Y^0$ for the treatment group

\bigskip

If we were estimating ATU, we'd be missing $Y^1$ for controls and need common support ($Y$ in treatment for all ranges of control), and for ATE we'd need both

\end{frame}

\begin{frame}{Visualizing propensity score to get common support}

	\begin{figure}
	\includegraphics[scale=0.05]{./lecture_includes/common_support_abadie.png}
	\end{figure}

\end{frame}

\begin{frame}{Definition and estimation}

Defining the ATT parameter of interest
\begin{equation}
ATT=E[Y^1_t - Y^0_t |D_t=1]
\end{equation}

\bigskip
Abadie's estimator
\begin{equation}
E\bigg [ \frac{Y_t - Y_b}{Pr(D_t=1)} \times \frac{D_t - Pr(D=1|X_b)}{1-Pr(D=1|X_b)} \bigg ]
\end{equation}


\end{frame}


\begin{frame}{Propensity scores}

\begin{itemize}
\item It's common to hear people say that we don't know the propensity score; we can only estimate it. Same here -- we approximate it with regressions
\item Paper is titled ``Semi-parametric DiD'' because Abadie imposes structure on the polynomials used to construct the propensity score (``series logit'')
\end{itemize}

\end{frame}



\begin{frame}{Abadie 2005 influence}

	\begin{figure}
	\includegraphics[scale=0.25]{./lecture_includes/abadie_restud_ipw}
	\end{figure}Abadie (2005) is his fourth most cited paper

\end{frame}




\subsection{Double Robust DiD}

\begin{frame}{Doubly Robust Difference-in-differences}

\begin{itemize}
\item DR models control for covariates twice -- once using the propensity score, once using outcomes adjusted by regression -- and are unbiased so long as:
	\begin{itemize}
	\item The regression specification for the outcome is correctly specified
	\item The propensity score specification is correctly specified
	\end{itemize}
\item Sant'Anna and Zhao (2020) incorporated DR into DiD by combining inverse probability weighting and outcome regression into a single DiD model
\item It's in the engine of Callaway and Sant'Anna (2020) that we discuss later so it merits close study
\item One of my favorite lesser known of the new DiD papers
\end{itemize}

\end{frame}

\begin{frame}{Patterns in econometrician reasoning}

\begin{enumerate}
\item Define the target parameter first (as opposed to writing down a regression specification first)
\item Identification (e.g., parallel trends)
\item Estimation
\item Aggregation
\item Inference
\end{enumerate}

\end{frame}


\begin{frame}{Defining the target parameter}

Major part of the new econometrics is to always start with the target parameter and build to it using estimation and identification that ``works''

\bigskip

\begin{eqnarray*}
\delta = E[Y^1_{it} - Y^0_{it} | D_i=1]
\end{eqnarray*}

\end{frame}

\begin{frame}{Identification assumptions I: Data}

Assumption 1: Assume panel data or repeated cross-sectional data

\bigskip

Handling repeated cross-sectional data is possible but assumes stationarity which is a kind of stability assumption, but I'll use panel representation. 

\bigskip

Cross-sections will be potentially violated with changing sample compositions (e.g., the Napster example). 

\end{frame}

\begin{frame}{Identification assumptions II: Modification to parallel trends}

Assumption 2: Conditional parallel trends

\bigskip

Counterfactual trends for the treatment group are the same as the control group for all values of $X$

\begin{eqnarray*}
E[Y_1^0 - Y_0^0 | X, D=1] = E[Y^0_1 - Y^0_0 | X, D=0]
\end{eqnarray*}

\end{frame}

\begin{frame}{Identification assumptions III: Common support}

Assumption 3: Common support

\bigskip

For some $e>0$, the probability of being in the treatment group is greater than $e$ and the probability of being in the treatment group conditional on $X$ is $\leq1-e$. 

\bigskip

Intuition of assumption 3: Called overlap or common support. Means there is at least a small fraction of the population that is treated and that for every value of the covariates $X$ there is at least a small chance that the unit is not treated. It's called common support when it's a propensity score but it's just about the distribution of treatment and control across values of $X$. Very common when dealing with covariate comparisons as otherwise you're extrapolating (curse of dimensionality)

\end{frame}

\begin{frame}{Estimating DD with Assumptions 1-3}

\begin{itemize}
\item Assumptions 1-3 gives us a couple of options of estimating the DiD
\item We can either use the outcome regression (OR) approach of Heckman, et al 1997
\item Or we can use the inverse probability weighting (IPW) approach of Abadie (2005)
\end{itemize}

\end{frame}


\begin{frame}{Heckman, et al. 1997}

	\begin{figure}
	\includegraphics[scale=0.35]{./lecture_includes/petra_restud_1997}
	\end{figure}

\end{frame}



\begin{frame}{Outcome regression}

This is the Heckman, et al. (1997) approach where the outcome evolution is modeled with a regression

\bigskip

\begin{eqnarray*}
\widehat{\delta}^{OR} = \overline{Y}_{1,1} - \bigg [ \overline{Y}_{1,0} + \frac{1}{n^T} \sum_{i|D_i=1} ( \widehat{\mu}_{0,1}(X_i) - \widehat{\mu}_{0,0}(X_i)) \bigg ]
\end{eqnarray*}

where $\overline{Y}$ is the sample average of $Y$ among units in the treatment group at time $t$ and $\widehat{\mu}(X)$ is an estimator of the true, but unknown, $m_{d,t}(X)$ which is by definition equal to $E[Y_t|D=d,X=x]$.

\end{frame}




\begin{frame}{Outcome regression}

\begin{eqnarray*}
\widehat{\delta}^{OR} = \overline{Y}_{1,1} - \bigg [ \overline{Y}_{1,0} + \frac{1}{n^T} \sum_{i|D_i=1} ( \widehat{\mu}_{0,1}(X_i) - \widehat{\mu}_{0,0}(X_i)) \bigg ]
\end{eqnarray*}

\begin{enumerate}
\item Regress changes $\Delta Y$ on $X$ among untreated groups using baseline covariates only
\item Get fitted values of the regression using all $X$ from $D=1$ only.  Average those
\item Calculate change in this fitted $Y$ among treated with the average fitted values
\end{enumerate}

\end{frame}

\begin{frame}{Inverse probability weighting}

This is the Abadie (2005) approach where we use weighting

\begin{eqnarray*}
\widehat{\delta}^{ipw} = \frac{1}{E_N[D]} E \bigg [ \frac{D-\widehat{p}(X)}{1-\widehat{p}(X)} (Y_1-Y_0) \bigg ]
\end{eqnarray*}

where $\widehat{p}(X)$ is an estimator for the true propensity score. Reduces the dimensionality of $X$ into a single scalar.

\end{frame}

\begin{frame}{These models cannot be ranked}

\begin{itemize}
\item Outcome regression needs $\widehat{\mu}(X)$ to be correctly specified, whereas
\item Inverse probability weighting needs $\widehat{p}(X)$ to be correctly specified
\item It's hard to ``rank'' these two in practice with regards to model misspecification because each is inconsistent when their own models are misspecified
\end{itemize}

\end{frame}


\begin{frame}{TWFE}

Consider our earlier TWFE specification:

\begin{eqnarray*}
Y_{it} = \alpha_1  + \alpha_2 T_t + \alpha_3 D_i +  \delta (T_i \times D_t)  + \varepsilon_{it}
\end{eqnarray*}

\bigskip

Just add in covariates then right?

\begin{eqnarray*}
Y_{it} = \alpha_1  + \alpha_2 T_t + \alpha_3 D_i  + \delta (T_i \times D_t) + \theta \cdot X_{it} + \varepsilon_{it}
\end{eqnarray*}

Sure! If you're willing to impose three \emph{more} assumptions

\end{frame}




\begin{frame}{Decomposing TWFE with covariates}

TWFE places restrictions on the DGP. Previous TWFE regression under assumptions 1-3 implies the following:

\bigskip

\begin{eqnarray*}
E[Y^1_1|D=1,X] = \alpha_1 + \alpha_2 + \alpha_3 + \delta + \theta X
\end{eqnarray*}

\bigskip

Conditional parallel trends implies

\small
\begin{eqnarray*}
&&E[Y^0_{1} - Y^0_{0}|D=1,X]= E[Y^0_{1} - Y^0_{0}|D=0,X] \\
&&E[Y^0_{1}|D=1,X] - E[Y^0_{0}|D=1,X]= E[Y^0_{1}|D=0,X] - E[Y^0_{0}|D=0,X] \\
&&E[Y^0_{1}|D=1,X] = E[Y^0_{0}|D=1,X] + E[Y^0_{1}|D=0,X] - E[Y^0_{0}|D=0,X] \\
&&E[Y^0_{1}|D=1,X] = E[Y_{0}|D=1,X] + E[Y_{1}|D=0,X] - E[Y_{0}|D=0,X] \\
\end{eqnarray*}


\end{frame}

\begin{frame}{Switching equation substitution}

Last line from the switching equation. This gives us:

\begin{eqnarray*}
E[Y^0_{1}|D=1,X] = \alpha_1  + \alpha_2 + \alpha_3 + \theta X
\end{eqnarray*}

Now compare this with our earlier $Y^1$ expression

\begin{eqnarray*}
E[Y^1_1|D=1,X] = \alpha_1 + \alpha_2 + \alpha_3 + \delta + \theta X
\end{eqnarray*}

We can define our target parameter, the ATT, now in terms of the fixed effects representation

\end{frame}


\begin{frame}{Collecting terms}

TWFE representation of our conditional expectations of the potential outcomes
\begin{eqnarray*}
&&E[Y^1_1|D=1,X] = \alpha_1 + \alpha_2 + \alpha_3 + \delta + \theta_1 X \\
&&E[Y^0_{1}|D=1,X] = \alpha_1  + \alpha_2 + \alpha_3 + \theta_2 X \\
\end{eqnarray*}

Substitute these into our target parameter

\begin{eqnarray*}
ATT &=& E[Y^1_1|D=1,X]  - E[Y^0_{1}|D=1,X]   \\
&&=(\alpha_1 + \alpha_2 + \alpha_3 + \delta + \theta_1 X) - ( \alpha_1  + \alpha_2 + \alpha_3 + \theta_2 X )\\
&&=\delta + (\theta_1 X - \theta_2 X)
\end{eqnarray*}

\bigskip

What if $\theta_1 X \neq \theta_2 X$?

\end{frame}

\begin{frame}{Assumption 4: Homogeneous treatment effects in X}


TWFE requires homogenous treatment effects in $X$ (i.e., the treatment effect is the same for all $X$)

\bigskip

If $X$ is sex, then effects are the same for males and females.

\bigskip

  If $X$ is continuous, like income, then the effect is the same whether someone makes \$1 or \$1 million.

\end{frame}

\begin{frame}{X-specific trends}

TWFE also places restrictions on covariate trends for the two groups too.  Take conditional expectations of our TWFE equation. 

\begin{eqnarray*}
E[Y_1|D=1] &=& \alpha_1 + \alpha_2 + \alpha_3 + \delta + \theta X_{11} \\
E[Y_0|D=1] &=& \alpha_1 + \alpha_3 + \theta X_{10} \\
E[Y_1|D=0] &=& \alpha_1 + \alpha_2 + \theta X_{01} \\
E[Y_0|D=0] &=& \alpha_1 + \theta X_{00}
\end{eqnarray*}


\end{frame}


\begin{frame}{X-specific trends}

Now take the DiD formula:

\begin{eqnarray*}
\delta^{DD} = &&\bigg ( (\alpha_1 + \alpha_2 + \alpha_3 + \delta + \theta X_{11} ) - (\alpha_1 + \alpha_3 + \theta X_{10} ) \bigg )- \\
&& \bigg ( (\alpha_1 + \alpha_2 + \theta X_{01}) - (\alpha_1 + \theta X_{00}) \bigg )
\end{eqnarray*}

\bigskip

Eliminating terms, we get:

\begin{eqnarray*}
\delta^{DD} = &&\delta + \\
&& (\theta X_{11} - \theta X_{10} ) - (\theta X_{01} - \theta X_{00} )
\end{eqnarray*}

\bigskip

Second line requires that trends in X for treatment group equal trends in X for control group.

\end{frame}


\begin{frame}{Assumption 5 and 6}

We need ``no X-specific trends'' for the treatment group (assumption 5) and comparison group (assumption 6)

\bigskip

\textbf{Intuition}: No X-specific trends means the evolution of potential outcome $Y^0$ is the same regardless of $X$. This would mean you cannot allow rich people to be on a different trend than poor people, for instance.

\bigskip

Without these six, in general TWFE will not identify ATT. 

\end{frame}

\begin{frame}{Why not both?}

\begin{itemize}
\item Let's review the problem.  What if you claim you need $X$ for conditional parallel trends?
\item You have three options:
	\begin{enumerate}
	\item Outcome regression (Heckman, et al. 1997) -- needs Assumptions 1-3
	\item Inverse probability weighting (Abadie 2005) -- needs Assumptions 1-3
	\item TWFE (everybody everywhere all the time) -- needs Assumptions 1-6
	\end{enumerate}
\item Problem is 1 and 2 need the models to be correctly specified
\item Doubly robust combines them to give us insurance; we now get two chances to be wrong, as opposed to just one
\end{itemize}

\end{frame}


\begin{frame}{Double Robust DiD}
\begin{eqnarray*}
\delta^{dr} = E \bigg [ \bigg ( \frac{D}{E[D]} -\frac{ \frac{p(X)(1-D)}{(1-p(X))} }{E \bigg [\frac{p(X)(1-D)}{(1-p(X))} \bigg ]} \bigg  )( \Delta Y - \mu_{0,\Delta}(X)) \bigg ]
\end{eqnarray*}

\begin{eqnarray*}
&&p(x): \text{propensity score model} \\
&& \Delta Y = Y_1 - Y_0 = Y_{post} - Y_{pre} \\
&& \mu_{d,\Delta} = \mu_{d,1}(X) - \mu_{d,0}(X), \text{ where } \mu(X) \text{ is a model for} \\
&& m_{d,t} = E[Y_t|D=d,X=x]
\end{eqnarray*}So that means $\mu_{0,\Delta}$ is just the control group's change in average $Y$ for each $X=x$

\end{frame}

\begin{frame}{Double Robust DiD}

\begin{eqnarray*}
\delta^{dr} = E \bigg [ \bigg ( \frac{D}{E[D]} -\frac{ \frac{p(X)(1-D)}{(1-p(X))} }{E \bigg [\frac{p(X)(1-D)}{(1-p(X))} \bigg ]} \bigg  )( \Delta Y - \mu_{0,\Delta}(X)) \bigg ]
\end{eqnarray*}

Notice how the model controls for $X$: you're weighting the adjusted outcomes using the propensity score

\bigskip

The reason you control for $X$ twice is because you don't know which model is right.  DR DiD frees you from making a choice without making you pay too much for it


\end{frame}

\begin{frame}{Efficiency}

\begin{itemize}
\item Authors exploit all the restrictions implied by the assumptions to construct semiparametric bounds
\item This is where the influence function comes in, which those who have studied the DID code closely may have noticed
\item One of the main results of the paper is that the DR DiD estimator is also DR for inference
\item Let's skip to Monte Carlos
\end{itemize}

\end{frame}

\begin{frame}{Monte Carlo details}

\begin{itemize}
\item Compare DR with TWFE, OR and IPW
\item Sample size is 1,000
\item 10,000 Monte Carlo experiments
\item Propensity score estimated with logit; OR estimated using linear specification
\end{itemize}

\end{frame}



\begin{frame}[plain]

\begin{table}[htbp]\centering
\scriptsize
\caption{Monte Carlo Simulations, DGP1, Both OR and Propensity score correct}
\centering
\begin{threeparttable}
\begin{tabular}{l*{5}{c}}
\toprule
\multicolumn{1}{l}{\textbf{}}&
\multicolumn{1}{c}{\textbf{Bias}}&
\multicolumn{1}{c}{\textbf{RMSE}}&
\multicolumn{1}{c}{\textbf{SE}}&
\multicolumn{1}{c}{\textbf{Coverage}}&
\multicolumn{1}{c}{\textbf{CI length}}\\
\midrule
TWFE & -20.9518 & 21.1227 & 2.5271 & 0.000 & 9.9061 \\
OR & -0.0012 & 0.1005 & 0.1010 & 0.9500 & 0.3960 \\
IPW & 0.0257 & 2.7743 & 2.6636 & 0.9518 & 10.4412 \\
DR & -0.0014 & 0.1059 & 0.1052 & 0.9473 & 0.4124 \\
\bottomrule
\end{tabular}
\end{threeparttable}
\end{table}

\end{frame}


\begin{frame}[plain]
	\begin{figure}
	\includegraphics[scale=0.25]{./lecture_includes/mc_dr_1.png}
	\end{figure}

\end{frame}


\begin{frame}[plain]

\begin{table}[htbp]\centering
\scriptsize
\caption{Monte Carlo Simulations, DGP4, Neither OR and Propensity score correct}
\centering
\begin{threeparttable}
\begin{tabular}{l*{5}{c}}
\toprule
\multicolumn{1}{l}{\textbf{}}&
\multicolumn{1}{c}{\textbf{Bias}}&
\multicolumn{1}{c}{\textbf{RMSE}}&
\multicolumn{1}{c}{\textbf{SE}}&
\multicolumn{1}{c}{\textbf{Coverage}}&
\multicolumn{1}{c}{\textbf{CI length}}\\
\midrule
TWFE & -16.3846 & 16.5383 & 3.6268 & 0.000 & 14.2169 \\
OR & -5.2045 & 5.3641 & 1.2890 & 0.0145 & 5.0531 \\
IPW & -1.0846 & 2.6557 & 2.3746 & 0.9487 & 9.3084 \\
DR & -3.1878 & 3.4544 & 1.2946 & 0.3076 & 5.0749 \\
\bottomrule
\end{tabular}
\end{threeparttable}
\end{table}

\end{frame}

\begin{frame}[plain]
	\begin{figure}
	\includegraphics[scale=0.12]{./lecture_includes/mc_dr_2.png}
	\end{figure}


\end{frame}

\subsection{Lalonde lab}

\begin{frame}{R and Stata Code}

There is code in R and Stata (all DiD estimators are now beautifully arranged at a website hosted by Asjad Naqvi)
\begin{itemize}
\item Stata: \textbf{drdid}
\item R: \textbf{drdid}
\end{itemize}

\bigskip

\url{https://asjadnaqvi.github.io/DiD/docs/01_stata/}

\bigskip

Remember -- it's for 2x2 with covariates (i.e., one treatment group). 

\end{frame}

\begin{frame}{Application using real data}

\begin{itemize}
\item Let's now use a real example with real data and see how well this does
\item Famous paper in AER by Lalonde (1986), an Orley and Card student at Princeton
\item Found that most program evaluation did badly, but let's revisit it with diff-in-diff
\end{itemize}

\end{frame}

\begin{frame}{Description of NSW Job Trainings Program}
	
The National Supported Work Demonstration (NSW), operated by Manpower Demonstration Research Corp in the mid-1970s:
	\begin{itemize}
	\item was a temporary employment program designed to help disadvantaged workers lacking basic job skills move into the labor market by giving them work experience and counseling in a sheltered environment
	\item was also unique in that it \textbf{randomly assigned} qualified applicants to training positions:
		\begin{itemize}
		\item \textbf{Treatment group}: received all the benefits of NSW program
		\item \textbf{Control group}: left to fend for themselves
		\end{itemize}
	\item admitted AFDC females, ex-drug addicts, ex-criminal offenders, and high school dropouts of both sexes
	\end{itemize}
\end{frame}

\begin{frame}{NSW Program}
	
	\begin{itemize}
	\item Treatment group members were:
		\begin{itemize}
		\item guaranteed a job for 9-18 months depending on the target group and site
		\item divided into crews of 3-5 participants who worked together and met frequently with an NSW counselor to discuss grievances and performance
		\item paid for their work
		\end{itemize}
	\item Control group members were randomized so the same
	\item Note: the randomization balanced observables and unobservables across the two arms, thus enabling the estimation of an ATE for the people who self-selected into the program
	\end{itemize}
\end{frame}

\begin{frame}{NSW Program}

\begin{itemize}
	\item Other details about the NSW program:
		\begin{itemize}
		\item \underline{Wages}:  NSW offered the trainees lower wage rates than they would've received on a regular job, but allowed their earnings to increase for satisfactory performance and attendance
		\item \underline{Post-treatment}: after their term expired, they were forced to find regular employment
		\item \underline{Job types}:  varied within sites -- gas station attendant, working at a printer shop -- and males and females were frequently performing different kinds of work
		\end{itemize}
\end{itemize}

\end{frame}
	
\begin{frame}{NSW Data}
	
	\begin{itemize}
	\item \underline{NSW data collection}:
		\begin{itemize}
		\item MDRC collected earnings and demographic information from both treatment and control at baseline and every 9 months thereafter
		\item Conducted up to 4 post-baseline interviews
		\item Different sample sizes from study to study can be confusing, but has simple explanations
		\end{itemize}
	\end{itemize}
\end{frame}
	

\begin{frame}{NSW Data}

\begin{itemize}
	\item \underline{Estimation}:
		\begin{itemize}
		\item NSW was a randomized job trainings program; therefore estimating the average treatment effect is straightforward:
			\begin{eqnarray*}
			\frac{1}{N_t}\sum_{D_i=1}Y_i - \frac{1}{N_c}\sum_{D_i=0}Y_i \approx E[Y^1-Y^0] 
			\end{eqnarray*}in large samples assuming treatment selection is independent of potential outcomes (randomization) -- i.e., $(Y^0,Y^1)\independent{D}$. 
		\end{itemize}
	\item \underline{NSW worked}: Treatment group participants' real earnings post-treatment (1978) was positive and economically meaningful -- $\approx$ \$900 (LaLonde 1986) to \$1,800 (Dehejia and Wahba 2002) depending on the sample used
\end{itemize}

\end{frame}
	
\begin{frame}[plain]
	\begin{center}
	LaLonde, Robert J. (1986). \myurlshort{http://business.baylor.edu/scott_cunningham/teaching/lalonde-1986.pdf}{``Evaluating the Econometric Evaluations of Training Programs with Experimental Data''}. \emph{American Economic Review}. 
	\end{center}
	
\underline{LaLonde's study} was \textbf{not} an evaluation of the NSW program, as that had been done, but rather an evaluation of econometric models done by:
		\begin{itemize}
		\item replacing the experimental NSW control group with non-experimental control group drawn from two nationally representative survey datasets: Current Population Survey (CPS) and Panel Study of Income Dynamics (PSID)
		\item estimating the average effect using non-experimental workers as controls for the NSW trainees 
		\item comparing his non-experimental estimates to the experimental estimates of \$900
		\end{itemize}
\end{frame}

\begin{frame}{LaLonde (1986)}

\begin{itemize}

	\item \underline{LaLonde's conclusion}: available econometric approaches were biased and inconsistent
		\begin{itemize}
		\item His estimates were way off and usually the wrong sign
		\item Conclusion was influential in policy circles and led to greater push for more experimental evaluations
		\end{itemize}

\end{itemize}

\end{frame}

\imageframe{./lecture_includes/lalonde_table5a.png}
\imageframe{./lecture_includes/lalonde_table5b.png}

\begin{frame}[plain,shrink=10]{Imbalanced covariates for experimental and non-experimental samples}

    \begin{center}
		\begin{table}
		\begin{tabular}{lcccccc}
		\hline \hline
		\multicolumn{3}{c}{}&
		\multicolumn{1}{c}{CPS}&
		\multicolumn{1}{c}{NSW}\\
		
		\multicolumn{1}{c}{}&
		\multicolumn{2}{c}{All} &
		\multicolumn{1}{c}{Controls} &
		\multicolumn{1}{c}{Trainees} \\

		\multicolumn{3}{c}{}&
		\multicolumn{1}{c}{$N_c=15,992$}&
		\multicolumn{1}{c}{$N_t=297$}&
		\multicolumn{1}{c}{}&
		\multicolumn{1}{c}{}\\

		\multicolumn{1}{l}{covariate}&
		\multicolumn{1}{c}{mean}&
		\multicolumn{1}{c}{(s.d.)}&
		\multicolumn{1}{c}{mean}&
		\multicolumn{1}{c}{mean}&
		\multicolumn{1}{c}{t-stat}&
		\multicolumn{1}{c}{diff}\\
		\hline
Black    & 0.09 & 0.28 & 0.07 & 0.80 & 47.04 & -0.73\\
Hispanic & 0.07 & 0.26 & 0.07 & 0.94 & 1.47 & -0.02\\
Age & 33.07 & 11.04 & 33.2 & 24.63 & 13.37  & 8.6\\
Married & 0.70 & 0.46 & 0.71 & 0.17 & 20.54 & 0.54\\
No degree & 0.30 & 0.46 & 0.30 & 0.73 & 16.27 & -0.43\\
Education & 12.0 & 2.86 & 12.03 & 10.38 & 9.85 & 1.65 \\
1975 Earnings   & 13.51 & 9.31 & 13.65 & 3.1 & 19.63 & 10.6\\
1975 Unemp  & 0.11 & 0.32 & 0.11 & 0.37 & 14.29 & -0.26\\
		\hline 
		\end{tabular}
		\end{table}
    \end{center}

\end{frame}


\begin{frame}{Lab}

\url{https://github.com/Mixtape-Sessions/Causal-Inference-2/tree/main/Lab/Lalonde}

\bigskip

Together let's do questions 1 and 2a-c

\end{frame}





\begin{frame}{Concluding remarks}

\begin{itemize}
\item So we hopefully see a few of the key elements of DiD
	\begin{itemize}
	\item Remember: the DiD equation and ATT equation are distinct concepts and definitions
	\item DiD designs can be implemented with OLS specifications that calculate differences in means
	\item Parallel pre-trends and parallel trends are not the same thing -- the first is testable, the latter is not testable
	\item Event studies are mandatory but pre-trends are smoking guns, but can mislead nonetheless
	\end{itemize}
\item Including \emph{time-varying} covariates in the canonical OLS specification requires additional assumptions
\item Doubly robust and IPW incorporate covariates through propensity scores and outcome regressions (or both) using baseline covariate means only
\end{itemize}

\end{frame}


\end{document}

\end{document}
