\documentclass{beamer}

% xcolor and define colors -------------------------
\usepackage{xcolor}

% https://www.viget.com/articles/color-contrast/
\definecolor{purple}{HTML}{5601A4}
\definecolor{navy}{HTML}{0D3D56}
\definecolor{ruby}{HTML}{9a2515}
\definecolor{alice}{HTML}{107895}
\definecolor{daisy}{HTML}{EBC944}
\definecolor{coral}{HTML}{F26D21}
\definecolor{kelly}{HTML}{829356}
\definecolor{cranberry}{HTML}{E64173}
\definecolor{jet}{HTML}{131516}
\definecolor{asher}{HTML}{555F61}
\definecolor{slate}{HTML}{314F4F}

% Mixtape Sessions
\definecolor{picton-blue}{HTML}{00b7ff}
\definecolor{violet-red}{HTML}{ff3881}
\definecolor{sun}{HTML}{ffaf18}
\definecolor{electric-violet}{HTML}{871EFF}

% Main theme colors
\definecolor{accent}{HTML}{00b7ff}
\definecolor{accent2}{HTML}{871EFF}
\definecolor{gray100}{HTML}{f3f4f6}
\definecolor{gray800}{HTML}{1F292D}


% Beamer Options -------------------------------------

% Background
\setbeamercolor{background canvas}{bg = white}

% Change text margins
\setbeamersize{text margin left = 15pt, text margin right = 15pt} 

% \alert
\setbeamercolor{alerted text}{fg = accent2}

% Frame title
\setbeamercolor{frametitle}{bg = white, fg = jet}
\setbeamercolor{framesubtitle}{bg = white, fg = accent}
\setbeamerfont{framesubtitle}{size = \small, shape = \itshape}

% Block
\setbeamercolor{block title}{fg = white, bg = accent2}
\setbeamercolor{block body}{fg = gray800, bg = gray100}

% Title page
\setbeamercolor{title}{fg = gray800}
\setbeamercolor{subtitle}{fg = accent}

%% Custom \maketitle and \titlepage
\setbeamertemplate{title page}
{
    %\begin{centering}
        \vspace{20mm}
        {\Large \usebeamerfont{title}\usebeamercolor[fg]{title}\inserttitle}\\
        {\large \itshape \usebeamerfont{subtitle}\usebeamercolor[fg]{subtitle}\insertsubtitle}\\ \vspace{10mm}
        {\insertauthor}\\
        {\color{asher}\small{\insertdate}}\\
    %\end{centering}
}

% Table of Contents
\setbeamercolor{section in toc}{fg = accent!70!jet}
\setbeamercolor{subsection in toc}{fg = jet}

% Button 
\setbeamercolor{button}{bg = accent}

% Remove navigation symbols
\setbeamertemplate{navigation symbols}{}

% Table and Figure captions
\setbeamercolor{caption}{fg=jet!70!white}
\setbeamercolor{caption name}{fg=jet}
\setbeamerfont{caption name}{shape = \itshape}

% Bullet points

%% Fix left-margins
\settowidth{\leftmargini}{\usebeamertemplate{itemize item}}
\addtolength{\leftmargini}{\labelsep}

%% enumerate item color
\setbeamercolor{enumerate item}{fg = accent}
\setbeamerfont{enumerate item}{size = \small}
\setbeamertemplate{enumerate item}{\insertenumlabel.}

%% itemize
\setbeamercolor{itemize item}{fg = accent!70!white}
\setbeamerfont{itemize item}{size = \small}
\setbeamertemplate{itemize item}[circle]

%% right arrow for subitems
\setbeamercolor{itemize subitem}{fg = accent!60!white}
\setbeamerfont{itemize subitem}{size = \small}
\setbeamertemplate{itemize subitem}{$\rightarrow$}

\setbeamertemplate{itemize subsubitem}[square]
\setbeamercolor{itemize subsubitem}{fg = jet}
\setbeamerfont{itemize subsubitem}{size = \small}


% Special characters

\usepackage{collectbox}

\makeatletter
\newcommand{\mybox}{%
    \collectbox{%
        \setlength{\fboxsep}{1pt}%
        \fbox{\BOXCONTENT}%
    }%
}
\makeatother





% Links ----------------------------------------------

\usepackage{hyperref}
\hypersetup{
  colorlinks = true,
  linkcolor = accent2,
  filecolor = accent2,
  urlcolor = accent2,
  citecolor = accent2,
}


% Line spacing --------------------------------------
\usepackage{setspace}
\setstretch{1.1}


% \begin{columns} -----------------------------------
\usepackage{multicol}


% Fonts ---------------------------------------------
% Beamer Option to use custom fonts
\usefonttheme{professionalfonts}

% \usepackage[utopia, smallerops, varg]{newtxmath}
% \usepackage{utopia}
\usepackage[sfdefault,light]{roboto}

% Small adjustments to text kerning
\usepackage{microtype}



% Remove annoying over-full box warnings -----------
\vfuzz2pt 
\hfuzz2pt


% Table of Contents with Sections
\setbeamerfont{myTOC}{series=\bfseries, size=\Large}
\AtBeginSection[]{
        \frame{
            \frametitle{Roadmap}
            \tableofcontents[current]   
        }
    }


% Tables -------------------------------------------
% Tables too big
% \begin{adjustbox}{width = 1.2\textwidth, center}
\usepackage{adjustbox}
\usepackage{array}
\usepackage{threeparttable, booktabs, adjustbox}
    
% Fix \input with tables
% \input fails when \\ is at end of external .tex file
\makeatletter
\let\input\@@input
\makeatother

% Tables too narrow
% \begin{tabularx}{\linewidth}{cols}
% col-types: X - center, L - left, R -right
% Relative scale: >{\hsize=.8\hsize}X/L/R
\usepackage{tabularx}
\newcolumntype{L}{>{\raggedright\arraybackslash}X}
\newcolumntype{R}{>{\raggedleft\arraybackslash}X}
\newcolumntype{C}{>{\centering\arraybackslash}X}

% Figures

% \imageframe{img_name} -----------------------------
% from https://github.com/mattjetwell/cousteau
\newcommand{\imageframe}[1]{%
    \begin{frame}[plain]
        \begin{tikzpicture}[remember picture, overlay]
            \node[at = (current page.center), xshift = 0cm] (cover) {%
                \includegraphics[keepaspectratio, width=\paperwidth, height=\paperheight]{#1}
            };
        \end{tikzpicture}
    \end{frame}%
}

% subfigures
\usepackage{subfigure}


% Highlight slide -----------------------------------
% \begin{transitionframe} Text \end{transitionframe}
% from paulgp's beamer tips
\newenvironment{transitionframe}{
    \setbeamercolor{background canvas}{bg=accent!40!black}
    \begin{frame}\color{accent!10!white}\LARGE\centering
}{
    \end{frame}
}


% Table Highlighting --------------------------------
% Create top-left and bottom-right markets in tabular cells with a unique matching id and these commands will outline those cells
\usepackage[beamer,customcolors]{hf-tikz}
\usetikzlibrary{calc}
\usetikzlibrary{fit,shapes.misc}

% To set the hypothesis highlighting boxes red.
\newcommand\marktopleft[1]{%
    \tikz[overlay,remember picture] 
        \node (marker-#1-a) at (0,1.5ex) {};%
}
\newcommand\markbottomright[1]{%
    \tikz[overlay,remember picture] 
        \node (marker-#1-b) at (0,0) {};%
    \tikz[accent!80!jet, ultra thick, overlay, remember picture, inner sep=4pt]
        \node[draw, rectangle, fit=(marker-#1-a.center) (marker-#1-b.center)] {};%
}

\usepackage{breqn} % Breaks lines

\usepackage{amsmath}
\usepackage{mathtools}

\usepackage{pdfpages} % \includepdf

\usepackage{listings} % R code
\usepackage{verbatim} % verbatim

% Video stuff
\usepackage{media9}

% packages for bibs and cites
\usepackage{natbib}
\usepackage{har2nat}
\newcommand{\possessivecite}[1]{\citeauthor{#1}'s \citeyearpar{#1}}
\usepackage{breakcites}
\usepackage{alltt}

% Setup math operators
\DeclareMathOperator{\E}{E} \DeclareMathOperator{\tr}{tr} \DeclareMathOperator{\se}{se} \DeclareMathOperator{\I}{I} \DeclareMathOperator{\sign}{sign} \DeclareMathOperator{\supp}{supp} \DeclareMathOperator{\plim}{plim}
\DeclareMathOperator*{\dlim}{\mathnormal{d}\mkern2mu-lim}
\newcommand\independent{\protect\mathpalette{\protect\independenT}{\perp}}
   \def\independenT#1#2{\mathrel{\rlap{$#1#2$}\mkern2mu{#1#2}}}
\newcommand*\colvec[1]{\begin{pmatrix}#1\end{pmatrix}}

\newcommand{\myurlshort}[2]{\href{#1}{\textcolor{gray}{\textsf{#2}}}}


\begin{document}

\imageframe{./lecture_includes/mixtape_did_cover.png}


% ---- Content ----

\section{Background material}
\subsection{Introduction}



\begin{frame}{Why diff-in-diff and differential timing?}

\begin{itemize}
\item Lots of new stuff and knowledge about it is unevenly distributed
\item   Madrid and Spain are closely tied to the development of causal panel estimators
	\begin{itemize}
	\item Abadie's 2003 synth paper about Basque country; 
	\item Pedro Sant'anna is from UC3M, 
	\item Sophie Sun is assistant professor at CEMFI, 
	\item Dmitri Arkhangelsky (``synthetic did'') is assistant professor at CEMFI
	\end{itemize}
\item So ironically, Europe is in its own way central to these new papers
\end{itemize}

\end{frame}

\begin{frame}{Differential timing outline}

We will cover some of the properties of twoway fixed effects (TWFE), some solutions and my personal opinions

\bigskip

\begin{enumerate}
	\item Brief review of potential outcomes and the ATT
	\item Difference-in-differences equation (``four averages and three differences'') and the parallel trends assumption
	\item TWFE Pathologies in static specification
		\begin{itemize}
		\item Goodman-Bacon decomposition as diagnosis of the problem
		\item Callaway and Sant'Anna estimator as a cure
		\end{itemize}
	\item TWFE Pathologies in event study specification
		\begin{itemize}
		\item Sun and Abraham as both a diagnosis and a cure 
		\end{itemize}
	\item Application, practical advice and code
\end{enumerate}

\end{frame}	


\begin{frame}{Beaver dam and diff-in-diff credibility crisis}
  \begin{columns}
    \begin{column}{0.5\textwidth}
      \begin{itemize}
\item Differential timing literature is like a stick that struck a beaver's dam
\item Stick made a hole causing a leak
\item Gradually that hole got larger and the leak got bigger
\item Eventually the dam collapsed 
\item That's now
      \end{itemize}
    \end{column}
    \begin{column}{0.5\textwidth}
      \includegraphics[scale=0.25]{./lecture_includes/beaverdam}
    \end{column}
  \end{columns}
\end{frame}





\begin{frame}{Difference-in-differences credibility crisis}

\begin{itemize}
\item I'll start with circa 2016 onward -- several grad students and assistant professors found critical pathologies with TWFE and developed solutions  
\item Many simultaneous discoveries, some redundancies, and \textbf{sudden} awareness of the issues started happening around 2017, eventually became a massive thing
\item Extreme meteoric rise, unusual for econometrics
\end{itemize}

\end{frame}

\begin{frame}{Compare with LATE paper}
  \begin{columns}
    \begin{column}{0.5\textwidth}
      \begin{itemize}
\item Compare with Imbens and Angrist 1995 LATE in \emph{Econometrica}
\item 61 annual cites the year Imbens is denied tenure at Harvard for what would later win him a Nobel Prize
      \end{itemize}
    \end{column}
    \begin{column}{0.5\textwidth}
      \includegraphics[scale=0.25]{./lecture_includes/guido_late}
    \end{column}
  \end{columns}
\end{frame}


\begin{frame}{Compare with synth paper}
  \begin{columns}
    \begin{column}{0.5\textwidth}
      \begin{itemize}
\item Athey and Imbens called synth the most important innovation in causal inference of the last two decades
\item Most econometrics papers, even influential ones, show slow growth
\item Something was different about diff-in-diff even before the econometricians recently shifted their attention to it
      \end{itemize}
    \end{column}
    \begin{column}{0.5\textwidth}
      \includegraphics[scale=0.25]{./lecture_includes/basque_synth}
    \end{column}
  \end{columns}
\end{frame}


\begin{frame}{Diff-in-diff had belonged to the empiricists}

	\begin{figure}
	\caption{Currie, et al. (2020)}
	\includegraphics[scale=0.25]{./lecture_includes/currie_did.png}
	\end{figure}

\bigskip

\footnotesize

With some exception (e.g., Heckman, Ichimura and Todd 1997; Abadie 2005; Bertrand, Duflo and Mullainthan 2004), econometricians had not given it much notice

\end{frame}





\begin{frame}{Borusyak et al}
  \begin{columns}
    \begin{column}{0.5\textwidth}
      \begin{itemize}
\item Starts it all; written as grad students at Harvard
\item Goes through many revisions, posted as working paper
\item Returned to a few years ago with a third coauthor, Jahn Spiess, now R\&R at Restud
      \end{itemize}
    \end{column}
    \begin{column}{0.5\textwidth}
      \includegraphics[scale=0.25]{./lecture_includes/kirill_cites}
    \end{column}
  \end{columns}
\end{frame}


\begin{frame}{``dCdH''}
  \begin{columns}
    \begin{column}{0.5\textwidth}
      \begin{itemize}
\item First major hit (in AER), may have been in working paper in 2017 (at least 2018)
\item Very thorough decomposition of the TWFE pathology, very general solution, included Stata code
\item Very active and talented young team (assistant profs when this was done)
      \end{itemize}
    \end{column}
    \begin{column}{0.5\textwidth}
      \includegraphics[scale=0.25]{./lecture_includes/dcdh_cites}
    \end{column}
  \end{columns}
\end{frame}

\begin{frame}{Goodman-Bacon}
  \begin{columns}
    \begin{column}{0.5\textwidth}
      \begin{itemize}
\item Arguably the most influential in terms of bringing attention to the problem (but no solution)
\item Begun while grad student at Michigan, published last of the crop
\item Probably Twitter network had a role as he was very active, also not an econometrician
      \end{itemize}
    \end{column}
    \begin{column}{0.5\textwidth}
      \includegraphics[scale=0.25]{./lecture_includes/andrew_cites}
    \end{column}
  \end{columns}
\end{frame}


\begin{frame}{``CS''}
  \begin{columns}
    \begin{column}{0.5\textwidth}
      \begin{itemize}
\item Second published solution to the problem, written while assistant professors at Vanderbilt and Ole Miss, 
\item Pedro is a UC3M alum (2015 grad) and Brantly is a Vanderbilt grad
\item Both are now coauthors with Andrew Goodman-Bacon
\item Introduced new terms like group-time ATT, released very tight R code (``did'')
      \end{itemize}
    \end{column}
    \begin{column}{0.5\textwidth}
      \includegraphics[scale=0.25]{./lecture_includes/pedro_cites}
    \end{column}
  \end{columns}
\end{frame}



\begin{frame}{``SA''}
  \begin{columns}
    \begin{column}{0.5\textwidth}
      \begin{itemize}
\item Third published solution to the problem, very similar to CS
\item Focus was on decomposing the event study
\item Written while grad students at MIT but Sophie Sun is now an assistant professor at CEMFI!
      \end{itemize}
    \end{column}
    \begin{column}{0.5\textwidth}
      \includegraphics[scale=0.25]{./lecture_includes/sun_cites}
    \end{column}
  \end{columns}
\end{frame}

\begin{frame}{There's even more and more coming}

\begin{itemize}
\item Gardner, Wooldridge, John Roth, and on and on
\item Too many people to name at this point
\item Given the large cites, we are likely to keep seeing more on this
\item Probably shifting applied practice for the better but there are some growing pains
\end{itemize}

\end{frame}










\section{TWFE Pathologies}

\subsection{Historical links}




\begin{frame}{Twoway fixed effects}

\begin{itemize}
\item When working with panel data, the so-called ``twoway fixed effects'' (TWFE) estimator was the workhorse estimator
\item And from the start, it was used with diff-in-diff
\item But at the start, it wasn't staggered adoption -- it was a much simpler design in which a group was treated in one year, and a comparison group wasn't
\end{itemize}

\end{frame}

\begin{frame}{Two OLS Models}

\begin{eqnarray}
Y_{ist} &=& \alpha_0 + \alpha_1 Treat_{is} + \alpha_2 Post_{t} + \delta (Treat_{is} \times Post_t) + \varepsilon_{ist} \\
Y_{ist} &=& \beta_0 + \delta D_{ist} + \tau_t + \sigma_s + \varepsilon_{ist}
\end{eqnarray}

\bigskip

First equation is used for simple designs when everyone is treated at once; second equation was used when different groups were treated at different times (``differential timing'')

\bigskip

First equation works; second one only sometimes works

\end{frame}


\subsection{Potential outcomes}

\begin{frame}{Equivalence}
$$Y_{ist} = \alpha_0 + \alpha_1 Treat_{is} + \alpha_2 Post_{t} + \textcolor{blue}{\delta} (Treat_{is} \times Post_t) + \varepsilon_{ist} $$

\bigskip

$$\widehat{\textcolor{blue}{\delta}} = \bigg ( \overline{y}_k^{post(k)} - \overline{y}_k^{pre(k)} \bigg ) - \bigg ( \overline{y}_U^{post(k)} - \overline{y}_U^{pre(k)} \bigg ) $$

\begin{itemize}
\item Orley claims that the TWFE estimator of $\delta$ and the ``four averages and three subtractions'' are the same thing numerically
\item And they are -- they are numerically \emph{identical}
\item And under a particular assumption, they are also unbiased estimates of an aggregate causal parameter
\item But to see this we need new notation -- potential outcomes
\end{itemize}

\end{frame}


















\subsection{Bacon decomposition}

\begin{frame}{Discussion of estimate}


$$Y_{ist} = \beta_0 + \delta D_{ist} + \tau_t + \sigma_s + \varepsilon_{ist}$$


\begin{itemize}
\item So that's the simple case; what about the differential timing case?  

\item If you estimate with OLS with differential timing, what does $\widehat{\delta}$ correspond to?
\item It also corresponds to the previous ``four averages and three subtractions'' -- but it's numerous of them, not just one

\end{itemize}

\end{frame}



\begin{frame}{Decomposition Preview}


\begin{itemize}
\item Andrew Goodman-Bacon decomposed $\widehat{\delta}$ and showed it is numerically identical to a weighted average of all ``four averages and three subtractions''
\item But, even before we get to causality there are unusual features
\item TWFE model assigns its own weights which are a function of the size of a ``group'' and the variance of group treatment dummies
\end{itemize}

\end{frame}



\begin{frame}{$K^2$ distinct DDs}

Let's look at 3 timing groups (a, b and c) and one untreated group (U).  With 3 timing groups, there are 9 2x2 DDs.  Here they are:


\begin{center}
\begin{tabular}{c|c|c}
\multicolumn{1}{l}{} &
\multicolumn{1}{l}{} &
\multicolumn{1}{l}{} \\
\midrule
a to b & b to a & c to a \\
a to c & b to c & c to b \\
a to U & b to U & c to U \\
\midrule
\end{tabular}
\end{center}

\bigskip

Let's return to a simpler example with only two groups -- a $k$ group treated at $t_k^*$ and an $l$ treated at $t_l^*$ plus an never-treated group called the $U$ untreated group
\end{frame} 


\begin{frame}{Terms and notation}

\begin{itemize}
\item Let there be two treatment groups (k,l) and one untreated group (U)
\item k,l define the groups based on when they receive treatment (differently in time) with k receiving it earlier than l
\item Denote $\overline{D}_k$ as the share of time each group spends in treatment status
\item Denote $\widehat{\delta}_{jb}^{2x2}$ as the canonical $2\times 2$ DD estimator for groups $j$ and b where $j$ is the treatment group and $b$ is the comparison group
\end{itemize}

\end{frame}


\imageframe{./lecture_includes/bacon_goodman_2.png}



\begin{frame}[plain]
$$\widehat{\delta}^{2x2}_{kU} = \bigg ( \overline{y}_k^{post(k)} - \overline{y}_k^{pre(k)} \bigg ) - \bigg ( \overline{y}_U^{post(k)} - \overline{y}_U^{pre(k)} \bigg ) $$
	\begin{figure}
	\includegraphics[scale=0.45]{./lecture_includes/bacon_goodman_3.png}
	\end{figure}

\end{frame}

\begin{frame}[plain]
$$\widehat{\delta}^{2x2}_{lU} = \bigg ( \overline{y}_l^{post(l)} - \overline{y}_l^{pre(l)} \bigg ) - \bigg ( \overline{y}_U^{post(l)} - \overline{y}_U^{pre(l)} \bigg ) $$
	\begin{figure}
	\includegraphics[scale=0.45]{./lecture_includes/bacon_goodman_4.png}
	\end{figure}

\end{frame}


\begin{frame}[plain]

$$\delta_{kl}^{2x2,k} = \bigg ( \overline{y}_k^{MID(k,l)} - \overline{y}_k^{Pre(k,l)} \bigg ) - \bigg ( \overline{y}_l^{MID(k,l)} - \overline{y}_l^{PRE(k,l)} \bigg ) $$

	\begin{figure}
	\includegraphics[scale=0.45]{./lecture_includes/bacon_goodman_6.png}
	\end{figure}

\end{frame}

\begin{frame}[plain]
$$\delta_{lk}^{2x2,l} = \bigg ( \overline{y}_l^{POST(k,l)} - \overline{y}_l^{MID(k,l)} \bigg ) - \bigg ( \overline{y}_k^{POST(k,l)} - \overline{y}_k^{MID(k,l)} \bigg ) $$

	\begin{figure}
	\includegraphics[scale=0.4]{./lecture_includes/bacon_goodman_7.png}
	\end{figure}

\end{frame}


	

\begin{frame}{Bacon decomposition}

$$Y_{ist} = \beta_0 + \delta D_{ist} + \tau_t + \sigma_s + \varepsilon_{ist}$$


TWFE estimate of $\widehat{\delta}$ is equal to a weighted average over all group 2x2 (of which there are 4 in this example)
\begin{eqnarray*}	
\widehat{\delta}^{TWFE} = \sum_{k \neq U} s_{kU}\widehat{\delta}_{kU}^{2x2} + \sum_{k \neq U} \sum_{l>k} s_{kl}  \bigg [ \mu_{kl}\widehat{\delta}_{kl}^{2x2,k} + (1-\mu_{kl}) \widehat{\delta}_{lk}^{2x2,l} \bigg]
\end{eqnarray*}where that first 2x2 combines the k compared to U and the l to U (combined to make the equation shorter)

\end{frame}
	


\begin{frame}{Third, the Weights}

 \begin{eqnarray*} s_{ku} &=& \frac{ n_k n_u \overline{D}_k (1- \overline{D}_k ) }{ \widehat{Var} ( \tilde{D}_{it} )} \\
s_{kl} &=& \frac{ n_k n_l (\overline{D}_k - \overline{D}_{l} ) ( 1- ( \overline{D}_k - \overline{D}_{l} )) }{\widehat{Var}(\tilde{D}_{it})} \\
\mu_{kl} &=& \frac{1 - \overline{D}_k }{1 - ( \overline{D}_k - \overline{D}_{l} )}
\end{eqnarray*}where $n$ refer to sample sizes, $\overline{D}_k (1- \overline{D}_k )$ $(\overline{D}_k - \overline{D}_{l} ) ( 1- ( \overline{D}_k - \overline{D}_{l} ))$ expressions refer to variance of treatment, and the final equation is the same for two timing groups.

\end{frame}

\begin{frame}{Weights discussion}

\begin{itemize}
\item Two things to note:
	\begin{itemize}
	\item More units in a group, the bigger its 2x2 weight is
	\item Group treatment variance weights up or down a group's 2x2
	\end{itemize}
\item Think about what causes the treatment variance to be as big as possible. Let's think about the $s_{ku}$ weights.
	\begin{itemize}
	\item $\overline{D}=0.1$. Then $0.1 \times 0.9 = 0.09$
	\item $\overline{D}=0.4$. Then $0.4 \times 0.6 =0.24$
	\item $\overline{D}=0.5$. Then $0.5 \times 0.5 = 0.25$
	\item $\overline{D}=0.6$. Then $0.6 \times 0.4 = 0.24$
	\end{itemize}
\item This means the weight on treatment variance is maximized for \emph{groups treated in middle of the panel}
\end{itemize}
\end{frame}

\begin{frame}{More weights discussion}

\begin{itemize}
\item But what about the ``treated on treated'' weights (i.e., $\overline{D}_k - \overline{D}_{l} $)  
\item Same principle as before - when the difference between treatment variance is close to 0.5, those 2x2s are given the greatest weight
\item For instance, say $t^*_k=0.15$ and $t^*_l=0.67$. Then $\overline{D}_k - \overline{D}_{l} = 0.52$.  And thus $0.52 \times 0.48 = 0.2496$.
\end{itemize}

\end{frame}


\begin{frame}{Summarizing TWFE centralities}

\begin{itemize}
\item Groups in the middle of the panel weight up their respective 2x2s via the variance weighting
\item Decomposition highlights the strange role of panel length when using TWFE
\item Different choices about panel length change both the 2x2 and the weights based on variance of treatment
\end{itemize}

\end{frame}




\begin{frame}{Back to TWFE}


$$Y_{ist} = \beta_0 + \delta D_{ist} + \tau_t + \sigma_s + \varepsilon_{ist}$$


\begin{itemize}

\item So we know that the estimate is a weighted average over all ``four averages and three subtractions'' but is that good or bad?
\item It's good if it's unbiased; it's bad if it isn't, and the decomposition doesn't tell us which unless we replace realized outcomes with potential outcomes
\item Bacon shows that TWFE estimate of $\delta$ needs two assumptions for unbiasedness:
	\begin{enumerate}
	\item variance weighted parallel trends are zero and 
	\item no dynamic treatment effects (not the case with 2x2)
	\end{enumerate}
\item Under those assumptions, TWFE estimator estimates the variance weighted ATT as a weighted average of all possible ATTs (not just weighted average of DiDs)

\end{itemize}

\end{frame}


\begin{frame}{Moving from 2x2s to causal effects and bias terms}

Let's start breaking down these estimators into their corresponding estimation objects expressed in causal effects and biases


\begin{eqnarray*}
\widehat{\delta}^{2x2}_{kU} &=& ATT_k{Post} + \Delta Y^0_k(Post(k),Pre(k)) - \Delta Y^0_U(Post(k),Pre) \\
\widehat{\delta}^{2x2}_{kl} &=& ATT_k(MID) + \Delta Y^0_k(MID,Pre) - \Delta Y^0_l(MID, Pre)
\end{eqnarray*}These look the same because you're always comparing the treated unit with an untreated unit (though in the second case it's just that they haven't been treated \emph{yet}). 

\end{frame}

\begin{frame}{The dangerous 2x2}

But what about the 2x2 that compared the late groups to the already-treated earlier groups? With a lot of substitutions we get:

\begin{eqnarray*}
\widehat{\delta}^{2x2}_{lk} &=& ATT_{l,Post(l)} + \underbrace{\Delta Y^0_l(Post(l),MID) - \Delta Y^0_k ( Post(l), MID)}_{\mathclap{\text{Parallel trends bias}}} \\
&& - \underbrace{(ATT_k(Post) - ATT_k(Mid))}_{\mathclap{\text{Heterogeneity bias!}}}
\end{eqnarray*}


\end{frame}

\begin{frame}{Substitute all this stuff into the decomposition formula}

\begin{eqnarray*}	
\widehat{\delta}^{DD} = \sum_{k \neq U} s_{kU}\widehat{\delta}_{kU}^{2x2} + \sum_{k \neq U} \sum_{l>k} s_{kl}  \bigg [ \mu_{kl}\widehat{\delta}_{kl}^{2x2,k} + (1-\mu_{kl}) \widehat{\delta}_{kl}^{2x2,l} \bigg]
\end{eqnarray*}where we will make these substitutions\begin{eqnarray*}
\widehat{\delta}_{kU}^{2x2} &=& ATT_k(Post) + \Delta Y_l^0(Post,Pre) - \Delta Y_U^0(Post, Pre) \\
\widehat{\delta}_{kl}^{2x2,k} &=& ATT_k(Mid) + \Delta Y_l^0(Mid,Pre) - \Delta Y_l^0(Mid, Pre) \\
\widehat{\delta}^{2x2,l}_{lk} &=& ATT_{l}Post(l) + \Delta Y^0_l(Post(l),MID) - \Delta Y^0_k ( Post(l), MID) \\
&&- (ATT_k(Post) - ATT_k(Mid))
\end{eqnarray*}Notice all those potential sources of biases! 

\end{frame}


\begin{frame}{Potential Outcome Notation}

\begin{eqnarray*}
p\text{ }lim\text{ } \widehat{\delta}^{TWFE}_{n\to\infty} &=& VWATT + VWPT - \Delta ATT
\end{eqnarray*}

\begin{itemize}
\item Notice the number of assumptions needed \emph{even} to estimate this very strange weighted ATT (which is a function of how you drew the panel in the first place). 
\item With dynamics, it attenuates the estimate (bias) and can even reverse sign depending on the magnitudes of what is otherwise effects in the sign in a reinforcing direction! 
\item Model can flip signs (does not satisfy a ``no sign flip property'')
\end{itemize}

\end{frame}



\subsection{Simulation}



\begin{frame}{Simulated data}

\begin{itemize}
\item 1000 firms, 40 states, 25 firms per states, 1980 to 2009 or 30 years, 30,000 observations, four groups

\item I'll impose ``unit level parallel trends'', which is much stronger than we need (we only need average parallel trends)

\item Also no anticipation of treatment effects until treatment occurs but does \emph{not} guarantee homogenous treatment effects

\item Two types of situations: constant versus dynamic treatment effects
\end{itemize}
\end{frame}



\begin{frame}{Constant vs Dynamic Treatment Effects}
    \begin{columns}
        \column{0.5\linewidth}
        \centering
        \includegraphics[height=6.5cm, width=5.5cm]{./lecture_includes/constant_te}

        \column{0.5\linewidth}
        \centering
        \includegraphics[height=6.5cm, width=5.5cm]{./lecture_includes/dynamic_te}
    \end{columns} 
\end{frame}




\begin{frame}{Group-time ATT}
       \begin{columns}
          \column{0.38\linewidth}
             \centering
             \includegraphics[height=6.5cm, width=5.5cm]{./lecture_includes/baker_attgt}
           \column{0.48\linewidth}
		\begin{itemize}
\item Heterogenous treatment effects across time and across groups
\item Cells are called ``group-time ATT'' (Callaway and Sant'anna 2020) or ``cohort ATT'' (Sun and Abraham 2020)
\item ATT is weighted average of all cells and $+82$ with uniform weights $1/60$
		\end{itemize}
         \end{columns} 
    \end{frame}

\begin{frame}{Estimation}

\bigskip

Estimate the following equation using OLS:

$$Y_{ist} = \alpha_i + \gamma_t +\delta D_{it} + \varepsilon_{ist}$$


\begin{table}[htbp]\centering
\small
\caption{Estimating ATT with different models}
\begin{center}
\begin{tabular}{l*{5}{c}}
\hline
\multicolumn{1}{l}{\textbf{}}&
\multicolumn{1}{c}{\textbf{Truth}}&
\multicolumn{1}{c}{\textbf{(TWFE)}}&
\multicolumn{1}{c}{\textbf{(CS)}}&
\multicolumn{1}{c}{\textbf{(SA)}}&
\multicolumn{1}{c}{\textbf{(BJS)}}\\
\hline
$\widehat{ATT}$  & 82    & -6.69*** &&&\\
\hline
\end{tabular}
\end{center}
\end{table}

The sign flipped.  Why?  Because of \emph{extreme} dynamics (i.e., $- \Delta ATT$)

\end{frame}

\begin{frame}{Bacon decomposition}
\begin{table}[htbp]\centering
\small
\caption{Bacon Decomposition (TWFE $= -6.69$)}
\begin{center}
\begin{tabular}{l*{5}{c}}
\hline
\multicolumn{1}{l}{\textbf{DD Comparison}}&
\multicolumn{1}{l}{\textbf{Weight}}&
\multicolumn{1}{l}{\textbf{Avg DD Est}}\\
\hline
Earlier T vs. Later C  &     0.500   &       51.800 \\
Later T vs. Earlier C   &    0.500    &     -65.180 \\
\midrule
T $=$ Treatment; C$ =$ Comparison \\
$(0.5*51.8) + (0.5*-65.180) = -6.69$ \\
\hline
\end{tabular}
\end{center}
\end{table}

\bigskip

While large weight on the ``late to early 2x2'' is \emph{suggestive} of an issue, these would appear even if we had constant treatment effects

\end{frame}






\section{Two solutions and a new decomposition}

\subsection{CS}



\begin{frame}{Callaway and Sant'Anna 2020}

CS is a DiD estimator used for estimating and then summarizing smaller ATT parameters under differential timing and conditional parallel trends into more policy relevant ATT parameters (either dynamic or static)


\begin{figure}
\includegraphics[scale=0.25]{./lecture_includes/pedro_cites}
\end{figure}

\end{frame}


\begin{frame}{When is CS used}

Just some examples of when you'd want to consider it:
\begin{enumerate}
\item When treatment effects differ depending on when it was adopted
\item When treatment effects change over time
\item When shortrun treatment effects are different than longrun effects
\item When treatment effect dynamics differ if people are first treated in a recession relative to expansion years
\end{enumerate}

\bigskip

CS estimates the ATT by identifying smaller causal effects and aggregating them using non-negative weights

\end{frame}






\begin{frame}{Group-time ATT}
       \begin{columns}
          \column{0.38\linewidth}
             \centering
             \includegraphics[height=6.5cm, width=5.5cm]{./lecture_includes/baker_attgt}
           \column{0.38\linewidth}
            Each cell contains that group's ATT(g,t)
\begin{eqnarray*}
ATT(g,t) = E[Y_t^1 - Y_t^0 | G_g=1]
\end{eqnarray*}CS identifies all feasible ATT(g,t)
         \end{columns} 
    \end{frame}




\begin{frame}{Group-time ATT}

Group-time ATT is the ATT for a specific group and time
\begin{itemize}
\item Groups are basically cohorts of units treated at the same time
\item Group-time ATT estimates are simple (weighted) differences in means
\item Does not directly restrict heterogeneity with respect to observed covariates, timing or the evolution of treatment effects over time
\item Allows us ways to choose our aggregations
\item Inference is the bootstrap
\end{itemize}

\end{frame}



\begin{frame}{Notation}

\begin{itemize}
\item $T$ periods going from $t=1, \dots, T$
\item Units are either treated ($D_t=1$) or untreated ($D_t=0$) but once treated cannot revert to untreated state
\item $G_g$ signifies a group and is binary.  Equals one if individual units are treated at time period $t$.
\item $C$ is also binary and indicates a control group unit equalling one if ``never treated'' (can be relaxed though to ``not yet treated'')
	\begin{itemize}
	\item Recall the problem with TWFE on using treatment units as controls
	\end{itemize}
\item Generalized propensity score enters into the estimator as a weight: $$\widehat{p(X)} = Pr(G_g=1 | X,G_g+C=1)$$
\end{itemize}

\end{frame}

\begin{frame}{Assumptions}

Assumption 1: Sampling is iid (panel data, but repeated cross-sections are possible) \\
\bigskip
Assumption 2: Conditional parallel trends (for either never treated or not yet treated) \\
\begin{eqnarray*}
E[Y_t^0 - Y_{t-1}^0 | X,G_g=1] = [Y_t^0 - Y_{t-1}^0 | X,C=1] 
\end{eqnarray*}
\bigskip
Assumption 3: Irreversible treatment \\
Assumption 4: Common support (propensity score) \\
\bigskip
Assumption 5: Limited treatment anticipation (i.e., treatment effects are zero pre-treatment)

\end{frame}

\begin{frame}{CS Estimator (the IPW version)}

\begin{eqnarray*}
ATT(g,t) = E \bigg [ \bigg ( \frac{G_g}{E[G_g]} - \frac{ \frac{\hat{p}(X)C}{1-\hat{p}(X)}}{E \bigg [ \frac{\hat{p}(X)C}{1-\hat{p}(X)} \bigg ]} \bigg ) (Y_t - Y_{g-1} ) \bigg ) \bigg ]
\end{eqnarray*}

This is the inverse probability weighting estimator.  Alternatively, there is an outcome regression approach and a doubly robust. Sant'Anna recommends DR.  CS uses the never-treated or the not-yet-treated as controls but never the already-treated 
\end{frame}




\begin{frame}{Aggregated vs single year/group ATT}

\begin{itemize}
\item The method they propose is really just identifying very narrow ATT per group time.
\item But we are often interested in  more aggregate parameters, like the ATT across all groups and all times
\item They present two alternative methods for building ``interesting parameters'' 
\item Inference from a bootstrap
\end{itemize}


\end{frame}



\begin{frame}{Group-time ATT }
             \includegraphics[scale=0.45]{./lecture_includes/baker_attgt_cs}

Question: Why didn't CS estimate all ATT(g,t)? What is ``feasible ATT''?

\end{frame}

\begin{frame}{Reporting results}
\begin{table}[htbp]\centering
\small
\caption{Estimating ATT using only pre-2004 data}
\begin{center}
\begin{tabular}{l*{5}{c}}
\hline
\multicolumn{1}{l}{\textbf{}}&
\multicolumn{1}{c}{\textbf{(Truth)}}&
\multicolumn{1}{c}{\textbf{(TWFE)}}&
\multicolumn{1}{c}{\textbf{(CS)}}&
\multicolumn{1}{c}{\textbf{(SA)}}&
\multicolumn{1}{c}{\textbf{(BJS)}}\\
\hline
$\widehat{Feasible\ ATT}$  & 68.33    & 26.81 *** & 68.34*** &&\\
\hline
\end{tabular}
\end{center}
\end{table}

TWFE is no longer negative, interestingly, once we eliminate the last group (giving us a never-treated group), but is still suffering from attenuation bias. 

\end{frame}



\subsection{SA}

\begin{frame}{Event study and differential timing}

\begin{itemize}
\item Sometimes we care about a simple summary, and sometimes we care about separating it out in time and sometimes in even more interesting ways
\item Event studies with one treatment group and one untreated group were relatively straightforward
\item Interact treatment group with calendar date to get a series of leads and lags
\item But when there are more than one treatment group, specification challenges emerge
\end{itemize}

\end{frame}



\begin{frame}{Differential timing complicates plotting sample averages}

	\begin{figure}
	\includegraphics[scale=0.25]{./lecture_includes/mml_eventstudy.png}
	\caption{Anderson, et al. (2013) display of raw traffic fatality rates for re-centered treatment states and control states with randomized treatment dates}
	\end{figure}

\end{frame}

\begin{frame}{Replicated from a project of mine}

	\begin{figure}
	\includegraphics[scale=0.5]{./lecture_includes/dd.pdf}
	\caption{Roll out of Craigslist ``personal ads'' for casual intimate encounters and birth rates using the ``randomized treatment assignment'' approach for visualization}
	\end{figure}

\end{frame}

\begin{frame}{Event study specification with TWFE}


\begin{eqnarray*}
Y_{i,t} = \alpha_i + \delta_t + \sum_{g \in G} \mu_g1\{t-E_i \in g \} + \varepsilon_{i,t}
\end{eqnarray*}

\bigskip

Coefficient $\mu_g$ on a dummy measuring the number of years prior to or after that unit was treated.  

\end{frame}

\begin{frame}[plain]
	\begin{figure}
	\includegraphics[scale=0.5]{./lecture_includes/br1544.pdf}
	\end{figure}
	
Same data as a couple slides ago, leads don't look good, so I abandoned the project. 
	
\end{frame}

\begin{frame}{Bias of TWFE Event Study Specificaiton}

\begin{itemize}
\item Bacon only focused on the static specification, and that's where the biases due to dynamics revealed itself
\item He was unable to get into the leads and lags using the FWL method he was using (``it's hard!'' - Bacon)
\item Sophie Sun and Sarah Abraham did though -- prompted by a stray comment by their professor
\item But they also unlike Bacon present a solution (which is like CS, but discovered independently)
\end{itemize}

\end{frame}



\begin{frame}{Sun and Abraham 2020}

	\begin{enumerate}
	\item SA shows a decomposition of the population regression coefficient on event study leads and lags with differential timing estimated with TWFE
	\item They show that the population regression coefficient is ``contaminated'' by information from other leads and lags (which is then later generalized by Goldsmith-Pinkham, Hull and Kolsar 2022)
	\item SA presents an alternative estimator that is a version of CS only using the ``last cohort'' as the treatment group (not the not-yet-treated)
	\item Derives the variance of the estimator instead of bootstrapping, handles covariates differently than CS, but otherwise identical
	\end{enumerate}

\end{frame}

\begin{frame}{Summarizing (cont.)}

\begin{itemize}
\item Under homogenous treatment profiles, weights sum to zero and``cancel out'' the treatment effects from other periods 
\item Under treatment effect heterogeneity, they do not cancel out and leads and lags are biased
\item They present a 3-step TWFE based alternative estimator which addresses the problems that they find
\end{itemize}

\end{frame}


\begin{frame}{Some notation and terms}

\begin{itemize}
\item As people often \textbf{bin} the data, we allow a lead or lag $l$ to appear in bin $g$ so sometimes they use $g$ instead of $l$ or $l \in g$
\item Building block is the ``cohort-specific ATT'' or $CATT_{e,l}$ -- same as ATT(g,t)
\item Our goal is to estimate $CATT_{e,l}$ with population regression coefficient $\mu_l$
\item They focus on irreversible treatment where treatment status is non-decreasing sequence of zeroes and ones
\end{itemize}

\end{frame}



\begin{frame}{Difficult notation (cont.)}

\begin{itemize}
\item The $\infty$ symbol is used to either describe the group ($E_i=\infty$) or the potential outcome ($Y^{\infty}$)
\item $Y^{\infty}_{i,t}$ is is the potential outcome for unit $i$ if it had never received treatment (versus received it later), also called the baseline outcome
\item Other counterfactuals are possible -- maybe unit $i$ isn't ``never treated'' but treated later in counterfactual
\end{itemize}
\end{frame}

\begin{frame}{More difficult notation (cont.)}

\begin{itemize}
\item Treatment effects are the difference between the observed outcome relative to the never-treated counterfactual outcome: $Y_{i,t} - Y^{\infty}_{i,t}$
\item We can take the average of treatment effects at a given relative time period across units first treated at time $E_i=e$ (same cohort) which is what we mean by $CATT_{e,l}$
\item Doesn't use $t$ index time (``calendar time''), rather uses $l$ which is time until or time after treatment date $e$ (``relative time'')
\item Think of it as ${l}=$year - treatment date
\end{itemize}

\end{frame}

\begin{frame}{Relative vs calendar event time}


\begin{figure}
\includegraphics[scale=0.2]{./lecture_includes/timetil}
\end{figure}

\end{frame}


\begin{frame}{Definition 1}

\textbf{Definition 1:} The cohort-specific ATT $l$ periods from initial treatment date $e$ is:

\begin{eqnarray*}
CATT_{e,l} = E[Y_{i,e+l} - Y^{\infty}_{i,e+l} | E_i=e]
\end{eqnarray*}

\bigskip

Fill out the second part of the Group-time ATT exercise together.

\end{frame}

\begin{frame}{TWFE assumptions}

\begin{itemize}
\item For consistent estimates of the coefficient leads and lags using TWFE model, we need three assumptions
\item For SA and CS, we only need two
\item Let's look then at the three
\end{itemize}

\end{frame}


\begin{frame}{Assumption 1: Parallel trends}

\textbf{Assumption 1: Parallel trends in baseline outcomes}: $E[Y^{\infty}_{i,t} - Y^{\infty}_{i,s} | E_i = e ]$ is the same for all $e \in supp(E_i)$ and for all $s$, $t$ and is equal to $E[Y^{\infty}_{i,t} - Y^{\infty}_{i,s} ]$

\bigskip

Lead and lag coefficients are DiD equations but once we invoke parallel trends they can become causal parameters.  This reminds us again how crucial it is to have  appropriate controls

\end{frame}


\begin{frame}{Assumption 2: No anticipation}

\textbf{Assumption 2: No anticipator behavior in pre-treatment periods}: There is a set of pre-treatment periods such that $E[Y_{i,e+l}^e - Y_{i,e+l}^{\infty} | E_i = e]=0$ for all possible leads.

\bigskip

Essentially means that pre-treatment, the causal effect is zero.  Most plausible if no one sees the treatment coming, but even if they see it coming, they may not be able to make adjustments that affect outcomes

\end{frame}


\begin{frame}{Assumption 3: Homogeneity}

\textbf{Assumption 3: Treatment effect profile homogeneity}: For each relative time period $l$, the $CATT_{e,l}$ doesn't depend on the cohort and is equal to $CATT_l$. 


\end{frame}

\begin{frame}{Treatment effect heterogeneity}

\begin{itemize}
\item Assumption 3 is violated when different cohorts experience different paths of treatment effects
\item Cohorts may differ in their covariates which affect how they respond to treatment (e.g., if treatment effects vary with age, and there is variation in age across units first treated at different times, then there will be heterogeneous treatment effects)
\item Doesn't rule out parallel trends
\end{itemize}

\end{frame}

\begin{frame}{Event study model}

Dynamic TWFE model

\begin{eqnarray*}
Y_{i,t} = \alpha_i + \delta_t + \sum_{g \in G} \mu_g1\{t-E_i \in g \} + \varepsilon_{i,t}
\end{eqnarray*}

\bigskip

We are interested in the properties of $\mu_g$ under differential timing as well as whether there are any never-treated units

\end{frame}



\begin{frame}[plain, shrink=20]
\begin{center}
\textbf{Interpreting $\widehat{\mu_g}$ under no to all assumptions}
\end{center}

\textbf{Proposition 1 (no assumptions):} The population regression coefficient on relative period bin $g$ is a linear combination of differences in trends from its own relative period $l \in g$, from relative periods $l \in g'$ of other bins $g' \neq g$, and from relative periods excluded from the specification (e.g., trimming). 

\begin{eqnarray*}
\mu_g &=& \underbrace{\sum_{l \in g} \sum_{e} w^g_{e,l} \big ( E[Y_{i,e+l} - Y^{\infty}_{i,0} | E_i = e] - E[Y^{\infty}_{i,e+l} - Y^{\infty}_{i,0}] \big )}_{\mathclap{\text{Targets}}} \\
&+& \underbrace{\sum_{g' \neq g} \sum_{l \in g'} \sum_e w^g_{e,l} \big ( E[Y_{i,e+l} - Y^{\infty}_{i,0} | E_i=e] - E[Y^{\infty}_{i,e+l} - Y^{\infty}_{i,0}] \big )}_{\mathclap{\text{Contamination from other leads and lags}}} \\
&+&  \underbrace{\sum_{l \in g^{excl}} \sum_{e} w^g_{e,l} \big ( E[Y_{i,e+l} - Y^{\infty}_{i,0} | E_i=e] - E[Y^{\infty}_{i,e+l} - Y^{\infty}_{i,0}] \big )}_{\mathclap{\text{Contamination from dropped periods}}} 
\end{eqnarray*}

\bigskip


\end{frame}

\begin{frame}{Weight ($w^g_{e,l}$) summation cheat sheet}

\begin{enumerate}
\item For relative periods of $\mu_g$ own $l \in g$, $\sum_{l \in g}\sum_ew^g_{e,l}=1$
\item For relative periods belonging to some other bin $l\in g'$ and $g' \neq g$, t $\sum_{l \in g'}\sum_ew^g_{e,l} = 0$
\item For relative periods not included in $G$, $\sum_{l \in g^{excl}} \sum_e w^g_{e,l} = -1$
\end{enumerate}

\end{frame}




\begin{frame}{Estimating the weights}

Regress $D^l_{i,t} \times 1\{E_i=e \}$ on:

\begin{enumerate}
\item all bin indicators included in the main TWFE regression, 
\item $\{ 1\{ t-E_i \in g \} \}_{g \in G}$(i.e., leads and lags) and 
\item the unit and time fixed effects
\end{enumerate}

\end{frame}


\begin{frame}{Still biased under parallel trends}

\textbf{Proposition 2}: Under the parallel trends only, the population regression coefficient on the indicator for relative period bing $g$ is a linear combination of $CATT_{e,l \in g}$ as well as $CATT_{d,l'}$ from other relative periods $l' \notin g$ with the same weights stated in Proposition 1:

\begin{eqnarray*}
\mu_g &=& \underbrace{\sum_{l \in g} \sum_e w^g_{e,l} CATT_{e,l}}_{\mathclap{\text{Desirable}}} \\
&& + \underbrace{\sum_{g' \neq g, g' \in G} \sum_{l' \in g'} \sum_e w^g_{e,l'}  CATT_{e,l'}}_{\mathclap{\text{Bias from other specified bins}}} \\
&&+ \underbrace{\sum_{l' \in g^{excl}} \sum_e w^g_{e,l'} CATT_{e,l'}}_{\mathclap{\text{Bias from dropped relative time indicators}}}
\end{eqnarray*}



\end{frame}


\begin{frame}{Still biased under parallel trends and no anticipation}

\textbf{Proposition 3}: If parallel trends holds and no anticipation holds for all $l<0$ (i.e., no anticipatory behavior pre-treatment), then the population regression coefficient $\mu_g$ for $g$ is a linear combination of post-treatment $CATT_{e,l'}$ for all $l' \geq 0$.

\begin{eqnarray*}
\mu_g &=& \sum_{l' \in g, l' \geq 0} \sum_e w^g_{e,l'} CATT_{e,l'} \\
&&+ \sum_{g' \neq g,g' \in G} \sum_{l' \in g', l' \geq 0} \sum_e w^g_{e,l'} CATT_{e,l'} \\
&&+ \sum_{l' \in g^{excl},l' \geq 0} \sum_e w^g_{w,l'} CATT_{e,l'}
\end{eqnarray*}

\end{frame}

\begin{frame}{Proposition 3 comment}

Notice how once we impose zero pre-treatment treatment effects, those terms are gone (i.e., no $l \in g, l<0$).  But the second term remains unless we impose treatment effect homogeneity (homogeneity causes terms due to weights summing to zero to cancel out). Thus $\mu_g$ may be non-zero for pre-treatment periods \emph{even though parallel trends hold in the pre period.}

\end{frame}

\begin{frame}{Proposition 4}

\textbf{Proposition 4}: If parallel trends and treatment effect homogeneity, then $CATT_{e,l}=ATT_l$ is constant across $e$ for a given $l$, and the population regression coefficient $\mu_g$ is equal to a linear combination of $ATT_{l \in g}$, as well as $ATT_{l' \notin g}$ from other relative periods

\begin{eqnarray*}
\mu_g &=& \sum_{l \in g} w^g_l ATT_l \\
&&+ \sum_{g' \neq g} \sum_{l' \in g'} w^g_{l'} ATT_{l'} \\
&&+ \sum_{l' \in g^{excl}} w^g_{l'}ATT_{l'}
\end{eqnarray*}


\end{frame}

\begin{frame}{Simple example}


Balanced panel $T=2$ with cohorts $E_i \in \{1,2 \}$. For illustrative purposes, we will include bins $\{-2,0\}$ in our calculations but drop $\{-1,1\}$. 


\end{frame}

\begin{frame}{Simple example}

\begin{eqnarray*}
\mu_{-2} &=& \underbrace{CATT_{2,-2}}_{\mathclap{\text{own period}}} + \underbrace{\frac{1}{2}CATT_{1,0} - \frac{1}{2} CATT_{2,0}}_{\mathclap{\text{other included bins}}} \\
&&+ \underbrace{ \frac{1}{2} CATT_{1,1} - CATT_{1,-1} - \frac{1}{2} CATT_{2,-1} }_{\mathclap{\text{Excluded bins}}}
\end{eqnarray*}

\begin{itemize}
\item Parallel trends gets us to all of the $CATT$
\item No anticipation makes $CATT=0$ for all $l<0$ (all $l<0$ cancel out)
\item Homogeneity cancels second and third terms
\item Still leaves $\frac{1}{2} CATT_{1,1}$ -- you chose  to exclude a group with a treatment effect
\end{itemize}Lesson: drop the relative time indicators on the left, not things on the right, bc lagged effects will contaminate through the excluded bins


\end{frame}


\begin{frame}{Robust event study estimation}


\begin{itemize}
\item All the robust estimators under differential timing have solutions and they all skip over forbidden contrasts. 
\item Sun and Abraham (2020) propose a 3-step interacted weighted estimator (IW) using last treated group as control group
\item Callaway and Sant'anna (2020) estimate group-time ATT which can be a weighted average over relative time periods too but uses ``not-yet-treated'' as control
\end{itemize}

\end{frame}




\begin{frame}{Interaction-weighted estimator}

\begin{itemize}
\item \textbf{Step one}: Do this DD regression and hold on to $\widehat{\delta}_{e,l}$
\end{itemize}

\begin{eqnarray*}
Y_{i,t} = \alpha_i + \lambda_t + \sum_{e \notin C} \sum_{l \neq -1} \delta_{e,l} \big (1 \{ E_i = e \} \cdot D_{i,t}^l \big ) + \varepsilon_{i,t}
\end{eqnarray*}


\bigskip

Can use never-treated or last-treated cohort. Drop always treated. The $\delta_{e,l}$ is a DD estimator for $CATT_{e,l}$ with particular choices for pre-period and cohort controls

\end{frame}


\begin{frame}{Interaction-weighted estimator}

\begin{itemize}
\item \textbf{Step two}: Estimate weights using sample shares of each cohort in the relevant periods:
\end{itemize}

\begin{eqnarray*}
Pr(E_i=e|E_i \in [-l,T-l])
\end{eqnarray*}

\end{frame}

\begin{frame}{Interaction-weighted estimator}

\begin{itemize}
\item \textbf{Step three}: Take a weighted average of estimates for $CATT_{e,l}$ from Step 1 with weight estimates from step 2
\end{itemize}


\begin{eqnarray*}
\widehat{v}_g = \frac{1}{|g|} \sum_{l \in g} \sum_e \widehat{\delta}_{e,l} \widehat{Pr} \{ E_i=e | E_i \in [-l,T-l]\}
\end{eqnarray*}


\end{frame}

\begin{frame}{Consistency and Inference}


\begin{itemize}
\item Under parallel trends and no anticipation, $\widehat{\delta}_{e,l}$ is consistent, and sample shares are also consistent estimators for population shares. 
\item Thus IW estimator is consistent for a weighted average of $CATT_{e,l}$ with weights equal to the share of each cohort in the relevant period(s).
\item They show that each IW estimator is asymptotically normal and derive its asymptotic variance. Doesn't rely on bootstrap like CS.
\end{itemize}

\end{frame}

\begin{frame}{DD Estimator of CATT}

\textbf{Definition 2}: DD estimator with pre-period $s$ and control cohorts $C$ estimates $CATT_{e,l}$ as:

\begin{eqnarray*}
\widehat{\delta_{e,l}} = \frac{ E_N \big [ \big ( Y_{i, e+l} - Y_{i,s} \big ) \times 1\{E_i=e\} \big ]}{E_N[1 \{E_i=e\} ]} - \frac{E_N \big [ \big ( Y_{i,e+l} \times 1 \{E_i \in C \} ]}{E_N [1 \{ E_i \in C \}]}
\end{eqnarray*}


\textbf{Proposition 5}: If parallel trends and no anticipation both hold for all pre-periods, then the DD estimator using any pre-period and non-empty control cohorts (never-treated or not-yet-treated) is an unbiased estimate for $CATT_{e,l}$

\end{frame}

\begin{frame}{Software}

\begin{itemize}
\item \textbf{Stata}: eventstudyinteract (can be installed from ssc)
\item \textbf{R}: fixest with subab() option (see \url{https://lrberge.github.io/fixest/reference/sunab.html/})
\end{itemize}


\end{frame}


\begin{frame}{Reporting results}
\begin{table}[htbp]\centering
\small
\caption{Estimating ATT}
\begin{center}
\begin{tabular}{l*{5}{c}}
\hline
\multicolumn{1}{l}{\textbf{}}&
\multicolumn{1}{c}{\textbf{(Truth)}}&
\multicolumn{1}{c}{\textbf{(TWFE)}}&
\multicolumn{1}{c}{\textbf{(CS)}}&
\multicolumn{1}{c}{\textbf{(SA)}}&
\multicolumn{1}{c}{\textbf{(BJS)}}\\
\hline
$\widehat{Feasible\ ATT}$  & 68.33    & 26.81*** & 68.34*** & 68.33***&\\
\hline
\end{tabular}
\end{center}
\end{table}

\end{frame}

\begin{frame}{Computing relative event time leads and lags }
             \includegraphics[scale=0.45]{./lecture_includes/sa_leads}

Two things to notice: (1) there only 17 lags with robust models but will be 24 with TWFE; (2) changing colors mean what?

\end{frame}

\begin{frame}{Comparing TWFE and SA }

\begin{figure}
\begin{center}
             \includegraphics[scale=0.4]{./lecture_includes/twfe_sa_event}
\end{center}
\end{figure}

Question: why is TWFE \emph{falling} pre-treatment?  Why is SA rising, but jagged, post-treatment?

\end{frame}

\section{Some opinions and an application}

\begin{frame}{Advice}

\begin{itemize}
\item DiD will remain popular for a while, and if anything all this new DiD has brought even more attention to it
\item But now things are changing -- how do we write the papers?  Not just how do we estimate parameters
\item Papers are a combination of science and rhetoric -- let's look at a new one
\item Braghieri, Levy and Makarin (2022), ``Social Media and Mental Health'', \emph{American Economic Review}, 112(11): 3660-3693
\end{itemize}

\end{frame}


\begin{frame}{Big picture}

\begin{itemize}
\item Widely cited that social media causes mental health problems in youth
\item Anecdotal, documentaries, but no causal evidence (``slim to none'')
\item Study will use staggered rollout of Facebook platform to college campuses from 2004 to 2006 to estimate the effect on aggregate mental health scores from a survey
\item You be the judge, but they present what in most cases would be strong evidence that Facebook harmed college students mental health

\end{itemize}

\end{frame}

\begin{frame}{Many things to like}

\begin{itemize}
\item Important question: mental health, suicide, review descriptive stats together
\item Strong design: staggered rollout
\item Event study is eye popping
\item Mechanism and main results
\item Very interesting dataset

\end{itemize}

\end{frame}


\begin{frame}{Fives parts of a strong DiD}

\begin{enumerate}

\item \textbf{Bite}: They cannot really show much here.  No data on Facebook usage.  More an ITT
\item \textbf{Main Results}: Estimated effect on mental health measures
\item \textcolor{red}{Mechanism}: Speculative
\item \textbf{Falsifications}: I can't really see very strong falsifications either.
\item \textbf{Event studies}: POW.  Just wait

\end{enumerate}

\end{frame}

\begin{frame}{Mechanism}

\begin{tikzpicture}[
  ->,>=stealth, % defines the style of the edge/arrow
  shorten >=1pt, % shortens the arrow at the target node
  auto,
  node distance=3cm, % defines the space between the nodes
  thick,
  main node/.style={circle,fill=blue!20,draw,font=\sffamily\Large\bfseries}, % style of the main nodes
  dashed node/.style={circle,fill=blue!20,draw,dashed,font=\sffamily\Large\bfseries} % style of the dashed node
  ]

  \node[main node] (D) {D}; 
  \node[main node] (M) [above right of=D] {M}; 
  \node[main node] (X) [below right of=D] {X};
  \node[main node] (Y) [right of=M, below of=M] {Y};
  \node[dashed node] (U) [above of=M] {U};
  
  \path[every node/.style={font=\sffamily\small}]
    (D) edge node {} (M)
    (D) edge node {} (X)
    (M) edge node {} (Y)
    (X) edge node {} (Y)
    (U) edge[dashed] node {} (D)
    (U) edge[dashed] node {} (M)
    (U) edge[dashed] node {} (X)
    (U) edge[dashed] node {} (Y);
\end{tikzpicture}

\end{frame}

\begin{frame}{Mechanism}

\begin{itemize}
\item $D$ is the treatment variable, and the ATT is over all possible channels, but what if you want to think $M$ is the mechanism
\item When you can't rule out competing theories with falsifications, you have to try and build the case that the effect is coming through a channel
\item Rule out $X$ and provide evidence for $M$
\item Goal here is to try and present evidence (not proof) that it's probably the story you're saying

\end{itemize}

\end{frame}



\begin{frame}{Ruling out alternative mechanism}

\begin{tikzpicture}[
  ->,>=stealth, % defines the style of the edge/arrow
  shorten >=1pt, % shortens the arrow at the target node
  auto,
  node distance=3cm, % defines the space between the nodes
  thick,
  main node/.style={circle,fill=blue!20,draw,font=\sffamily\Large\bfseries}, % style of the main nodes
  dashed node/.style={circle,fill=blue!20,draw,dashed,font=\sffamily\Large\bfseries} % style of the dashed node
  ]

  \node[main node] (D) {D}; 
  \node[main node] (M) [right of=D] {M}; 
  \node[main node] (Y) [right of=M] {Y};
  \node[dashed node] (U) [above of=M] {U};
  
  \path[every node/.style={font=\sffamily\small}]
    (D) edge node {} (M)
    (M) edge node {} (Y)
    (U) edge[dashed] node {} (D)
    (U) edge[dashed] node {} (M)
    (U) edge[dashed] node {} (Y);
\end{tikzpicture}


\end{frame}





\begin{frame}{Mechanism}

\begin{itemize}

\item Story is interpersonal comparisons which they try to show 
\item We can discuss how plausible we found it, but ask yourself at the end -- did the event study help you believe it?  Why/why not?

\end{itemize}

\end{frame}

\begin{frame}{Data on Facebook}

\begin{itemize}

\item Ingenious use of the Wayback Time Machine
\item Looked at over 700 schools using Facebook screen shots
\item When Facebook first mentions a school on its front page, that school is marked as having gotten Facebook

\end{itemize}

\end{frame}


\begin{frame}{New schools being adopted}
\begin{center}
\includegraphics[scale=0.35]{./lecture_includes/timemachine_1}
\end{center}
\end{frame}

\begin{frame}{New schools being adopted}
\begin{center}
\includegraphics[scale=0.35]{./lecture_includes/timemachine_2}
\end{center}
\end{frame}


\begin{frame}{Data on college students}

\begin{itemize}
\item NCHA Data is survey administered to college students on a semi-annual basis by American College Health Assoc
\item Inquires about demographics, physical health, mental health, alcohol and drug use, sexual behaviors, and perception of these behaviors by peers
\item ACHA merged a treatment indicator to each respondent based on Facebook dataset provided to them so that privacy could be maintained

\end{itemize}

\end{frame}

\begin{frame}{Mental health}

\begin{itemize}

\item Self-reported symptoms are standard medical practice in mental health -- DSM-5 relies on self-reports such as difficulty sleeping, fatigue, feelings of guilt, suicidal ideation
\item No data on Facebook or social media usage so this is ITT version of the ATT
\item Respondent answers to the questions are aggregated into indices such as \emph{poor mental health} where larger numbers are worse

\end{itemize}

\end{frame}

\begin{frame}{Main TWFE Model}

\begin{equation}
Y_{icgt} = \alpha_g + \delta_t + \beta \times Facebook_{gt} + X_i \times \gamma + X_c \times \psi + \varepsilon_{icgt}
\end{equation}

\bigskip

$Y_{icgt}$ is an outcome for person $i$ in wave $t$ attending college $c$ in expansion group $g$; $\alpha_g$ is expansion-group or college fixed effects; $\delta_t$ are survey-wave fixed effects; $Facebook_{gt}$ indicates the respondents' campus has Facebook by time $t$ at expansion group $g$; $X_i$ and $X_c$ are individual and college-level controls; and standard errors are clustered at college level.

\bigskip

$\widehat{\beta}$ identifies the ATT under parallel trends in the robust models


\end{frame}

\begin{frame}{Robustness}

\begin{itemize}

\item Main static results will all be in TWFE, but appendix shows other methods like CS and SA 
\item Event studies will show all models including some we haven't reviewed
\item Growing popularity to show ``all the robust DiD'' models so that readers can see you aren't cherry picking

\end{itemize}

\end{frame}



\begin{frame}
\begin{center}
\includegraphics[scale=0.35]{./lecture_includes/facebook_1}
\end{center}
\end{frame}

\begin{frame}
\begin{center}
\includegraphics[scale=0.35]{./lecture_includes/facebook_2}
\end{center}
\end{frame}

\begin{frame}
\begin{center}
\includegraphics[scale=0.35]{./lecture_includes/facebook_3}
\end{center}
\end{frame}

\begin{frame}
\begin{center}
\includegraphics[scale=0.35]{./lecture_includes/facebook_4}
\end{center}
\end{frame}

\begin{frame}
\begin{center}
\includegraphics[scale=0.35]{./lecture_includes/facebook_5}
\end{center}
\end{frame}

\begin{frame}
\begin{center}
\includegraphics[scale=0.35]{./lecture_includes/facebook_6}
\end{center}
\end{frame}

\begin{frame}
\begin{center}
\includegraphics[scale=0.35]{./lecture_includes/facebook_7}
\end{center}
\end{frame}


\begin{frame}{Comments}

\begin{itemize}

\item Can't lose sight of the big picture -- you still have to write a great paper, not just pass your exams, and going from estimation to publishing is a different but related skill
\item Some of the old exhibits may not carry forward (TWFE with many columns)
\item Increasingly, people are presenting a single event study graph with ``all the DiD'' against TWFE so as to avoid cherry picking estimators
\item You should use the tool for the job, but these differences are subtle (``which parallel trends?'', ``which comparison group?'')
\end{itemize}

\end{frame}

\begin{frame}{Conclusion}

\begin{itemize}
\item Good question, good data, and you can publish well with DiD
\item Hardly definitive, but the staggered design is a solution to our inability to run the RCT
\item Remember -- many questions can be randomized in theory but not practice (e.g., smoking)
\item Learn as much as you can and don't stop learning

\end{itemize}

\end{frame}






\end{document}

