\documentclass{beamer}

% xcolor and define colors -------------------------
\usepackage{xcolor}

% https://www.viget.com/articles/color-contrast/
\definecolor{purple}{HTML}{5601A4}
\definecolor{navy}{HTML}{0D3D56}
\definecolor{ruby}{HTML}{9a2515}
\definecolor{alice}{HTML}{107895}
\definecolor{daisy}{HTML}{EBC944}
\definecolor{coral}{HTML}{F26D21}
\definecolor{kelly}{HTML}{829356}
\definecolor{cranberry}{HTML}{E64173}
\definecolor{jet}{HTML}{131516}
\definecolor{asher}{HTML}{555F61}
\definecolor{slate}{HTML}{314F4F}

% Mixtape Sessions
\definecolor{picton-blue}{HTML}{00b7ff}
\definecolor{violet-red}{HTML}{ff3881}
\definecolor{sun}{HTML}{ffaf18}
\definecolor{electric-violet}{HTML}{871EFF}

% Main theme colors
\definecolor{accent}{HTML}{00b7ff}
\definecolor{accent2}{HTML}{871EFF}
\definecolor{gray100}{HTML}{f3f4f6}
\definecolor{gray800}{HTML}{1F292D}


% Beamer Options -------------------------------------

% Background
\setbeamercolor{background canvas}{bg = white}

% Change text margins
\setbeamersize{text margin left = 15pt, text margin right = 15pt} 

% \alert
\setbeamercolor{alerted text}{fg = accent2}

% Frame title
\setbeamercolor{frametitle}{bg = white, fg = jet}
\setbeamercolor{framesubtitle}{bg = white, fg = accent}
\setbeamerfont{framesubtitle}{size = \small, shape = \itshape}

% Block
\setbeamercolor{block title}{fg = white, bg = accent2}
\setbeamercolor{block body}{fg = gray800, bg = gray100}

% Title page
\setbeamercolor{title}{fg = gray800}
\setbeamercolor{subtitle}{fg = accent}

%% Custom \maketitle and \titlepage
\setbeamertemplate{title page}
{
    %\begin{centering}
        \vspace{20mm}
        {\Large \usebeamerfont{title}\usebeamercolor[fg]{title}\inserttitle}\\
        {\large \itshape \usebeamerfont{subtitle}\usebeamercolor[fg]{subtitle}\insertsubtitle}\\ \vspace{10mm}
        {\insertauthor}\\
        {\color{asher}\small{\insertdate}}\\
    %\end{centering}
}

% Table of Contents
\setbeamercolor{section in toc}{fg = accent!70!jet}
\setbeamercolor{subsection in toc}{fg = jet}

% Button 
\setbeamercolor{button}{bg = accent}

% Remove navigation symbols
\setbeamertemplate{navigation symbols}{}

% Table and Figure captions
\setbeamercolor{caption}{fg=jet!70!white}
\setbeamercolor{caption name}{fg=jet}
\setbeamerfont{caption name}{shape = \itshape}

% Bullet points

%% Fix left-margins
\settowidth{\leftmargini}{\usebeamertemplate{itemize item}}
\addtolength{\leftmargini}{\labelsep}

%% enumerate item color
\setbeamercolor{enumerate item}{fg = accent}
\setbeamerfont{enumerate item}{size = \small}
\setbeamertemplate{enumerate item}{\insertenumlabel.}

%% itemize
\setbeamercolor{itemize item}{fg = accent!70!white}
\setbeamerfont{itemize item}{size = \small}
\setbeamertemplate{itemize item}[circle]

%% right arrow for subitems
\setbeamercolor{itemize subitem}{fg = accent!60!white}
\setbeamerfont{itemize subitem}{size = \small}
\setbeamertemplate{itemize subitem}{$\rightarrow$}

\setbeamertemplate{itemize subsubitem}[square]
\setbeamercolor{itemize subsubitem}{fg = jet}
\setbeamerfont{itemize subsubitem}{size = \small}


% Special characters

\usepackage{collectbox}

\makeatletter
\newcommand{\mybox}{%
    \collectbox{%
        \setlength{\fboxsep}{1pt}%
        \fbox{\BOXCONTENT}%
    }%
}
\makeatother





% Links ----------------------------------------------

\usepackage{hyperref}
\hypersetup{
  colorlinks = true,
  linkcolor = accent2,
  filecolor = accent2,
  urlcolor = accent2,
  citecolor = accent2,
}


% Line spacing --------------------------------------
\usepackage{setspace}
\setstretch{1.1}


% \begin{columns} -----------------------------------
\usepackage{multicol}


% Fonts ---------------------------------------------
% Beamer Option to use custom fonts
\usefonttheme{professionalfonts}

% \usepackage[utopia, smallerops, varg]{newtxmath}
% \usepackage{utopia}
\usepackage[sfdefault,light]{roboto}

% Small adjustments to text kerning
\usepackage{microtype}



% Remove annoying over-full box warnings -----------
\vfuzz2pt 
\hfuzz2pt


% Table of Contents with Sections
\setbeamerfont{myTOC}{series=\bfseries, size=\Large}
\AtBeginSection[]{
        \frame{
            \frametitle{Roadmap}
            \tableofcontents[current]   
        }
    }


% Tables -------------------------------------------
% Tables too big
% \begin{adjustbox}{width = 1.2\textwidth, center}
\usepackage{adjustbox}
\usepackage{array}
\usepackage{threeparttable, booktabs, adjustbox}
    
% Fix \input with tables
% \input fails when \\ is at end of external .tex file
\makeatletter
\let\input\@@input
\makeatother

% Tables too narrow
% \begin{tabularx}{\linewidth}{cols}
% col-types: X - center, L - left, R -right
% Relative scale: >{\hsize=.8\hsize}X/L/R
\usepackage{tabularx}
\newcolumntype{L}{>{\raggedright\arraybackslash}X}
\newcolumntype{R}{>{\raggedleft\arraybackslash}X}
\newcolumntype{C}{>{\centering\arraybackslash}X}

% Figures

% \imageframe{img_name} -----------------------------
% from https://github.com/mattjetwell/cousteau
\newcommand{\imageframe}[1]{%
    \begin{frame}[plain]
        \begin{tikzpicture}[remember picture, overlay]
            \node[at = (current page.center), xshift = 0cm] (cover) {%
                \includegraphics[keepaspectratio, width=\paperwidth, height=\paperheight]{#1}
            };
        \end{tikzpicture}
    \end{frame}%
}

% subfigures
\usepackage{subfigure}


% Highlight slide -----------------------------------
% \begin{transitionframe} Text \end{transitionframe}
% from paulgp's beamer tips
\newenvironment{transitionframe}{
    \setbeamercolor{background canvas}{bg=accent!40!black}
    \begin{frame}\color{accent!10!white}\LARGE\centering
}{
    \end{frame}
}


% Table Highlighting --------------------------------
% Create top-left and bottom-right markets in tabular cells with a unique matching id and these commands will outline those cells
\usepackage[beamer,customcolors]{hf-tikz}
\usetikzlibrary{calc}
\usetikzlibrary{fit,shapes.misc}

% To set the hypothesis highlighting boxes red.
\newcommand\marktopleft[1]{%
    \tikz[overlay,remember picture] 
        \node (marker-#1-a) at (0,1.5ex) {};%
}
\newcommand\markbottomright[1]{%
    \tikz[overlay,remember picture] 
        \node (marker-#1-b) at (0,0) {};%
    \tikz[accent!80!jet, ultra thick, overlay, remember picture, inner sep=4pt]
        \node[draw, rectangle, fit=(marker-#1-a.center) (marker-#1-b.center)] {};%
}

\usepackage{breqn} % Breaks lines

\usepackage{amsmath}
\usepackage{mathtools}

\usepackage{pdfpages} % \includepdf

\usepackage{listings} % R code
\usepackage{verbatim} % verbatim

% Video stuff
\usepackage{media9}

% packages for bibs and cites
\usepackage{natbib}
\usepackage{har2nat}
\newcommand{\possessivecite}[1]{\citeauthor{#1}'s \citeyearpar{#1}}
\usepackage{breakcites}
\usepackage{alltt}

% Setup math operators
\DeclareMathOperator{\E}{E} \DeclareMathOperator{\tr}{tr} \DeclareMathOperator{\se}{se} \DeclareMathOperator{\I}{I} \DeclareMathOperator{\sign}{sign} \DeclareMathOperator{\supp}{supp} \DeclareMathOperator{\plim}{plim}
\DeclareMathOperator*{\dlim}{\mathnormal{d}\mkern2mu-lim}
\newcommand\independent{\protect\mathpalette{\protect\independenT}{\perp}}
   \def\independenT#1#2{\mathrel{\rlap{$#1#2$}\mkern2mu{#1#2}}}
\newcommand*\colvec[1]{\begin{pmatrix}#1\end{pmatrix}}

\newcommand{\myurlshort}[2]{\href{#1}{\textcolor{gray}{\textsf{#2}}}}


\begin{document}

\imageframe{./lecture_includes/mixtape_did_cover.png}


% ---- Content ----


\section{Synthetic control}

\subsection{Abadie, Diamond and Hainmueller}

\begin{frame}{Day 2: Synthetic Control}

\begin{itemize}
\item Now we move into the realm of how to handle perfect doctor scenarios without randomization
\item Largely going to be a focus on modeling outcomes with panel data today
\item Key characteristics will be small number of units but long panels
\item We will also look at the situation when ``fit'' for single treated unit is good (ADH) versus bad (augmented)
\end{itemize}

\end{frame}

\begin{frame}[plain]
	\begin{figure}
	\includegraphics[scale=0.25]{./lecture_includes/currie_synth.png}
	\end{figure}
\end{frame}

\begin{frame}{What is synthetic control}
	
	\begin{itemize}
	\item Synthetic control has been called the most important innovation in causal inference of the last two decades (Athey and Imbens 2017)
	\item Originally designed for comparative case studies, but newer developments have extended it to multiple treated units as well as differential timing
	\item Continues to also be methodologically a frontier for applied econometrics, so consider this talk a starting point for you
	\end{itemize}
\end{frame}
	
\begin{frame}{What is a comparative case study}

\begin{itemize}
\item Single treated unit -- country, state, firm
\item Social scientists tackle such situations in two ways: qualitatively and quantitatively
\item In political science, probably others, you see as a stark dividing line between camps, (economics doesn't have a qualitative tradition)
\end{itemize}

\end{frame}


\begin{frame}{Qualitative comparative case studies}
	
	\begin{itemize}
	\item In qualitative comparative case studies, the goal might be to reason \emph{inductively} the causal effects of events or characteristics of a single unit on some outcome, oftentimes through logic and historical analysis.  
		\begin{itemize}
		\item May not answer the causal questions at all because there is rarely a counterfactual, or if so, it's ad hoc.
		\item Classic example of comparative case study approach is Alexis de Toqueville's \underline{Democracy in America} (but he is regularly comparing the US to France)
		\end{itemize}
	\item You won't find someone claiming that some event caused GDP to fall \$1500 when compared against France in qualitative
	\end{itemize}
\end{frame}

\begin{frame}{Traditional quantitative comparative case studies}

\begin{itemize}
	\item Traditional quantitative comparative case studies are explicitly causal designs in that there is a treatment and control
	\item Usually treatment is based on a natural experiment applied to a single aggregate unit (e.g., city, school, firm, state, country)
	\item Method compares the evolution of an aggregate outcome for the unit affected by the intervention to the evolution of the same \emph{ad hoc} aggregate control group (Card 1990; Card and Krueger 1994)
	\item It'll essentially be diff-in-diff, but it may not use the event study, and the point is the choice of controls is a subset of all possible controls
\end{itemize}

\end{frame}

\begin{frame}{Pros and cons}
	
	\begin{itemize}
	\item Pros:
		\begin{itemize}
		\item Takes advantage of policy interventions that take place at an aggregate level (which is common and so this is useful)
		\item Aggregate/macro data are often available (which may be all we have)
		\end{itemize}
	\item Cons:
		\begin{itemize}
		\item Selection of control group is \emph{ad hoc} -- opens up researcher biases, even unconscious
		\item Standard errors do not reflect uncertainty about the ability of the control group to reproduce the counterfactual of interest
		\end{itemize}
	\end{itemize}
\end{frame}

\begin{frame}{Description of the Mariel Boatlift}
	
	\begin{itemize}
	\item Card (1990) uses the Mariel Boatlift of 1980 as a natural experiment to measure the effect of a sudden influx of immigrants on unemployment among less-skilled natives
	\item His question was how do inflows of immigrants affect the wages and employment of natives in local US labor markets?
	\item The Mariel Boatlift brought 100,000 Cubans to Miami which increased the Miami labor force by 7\%
	\item Individual-level data on unemployment from the Current Population Survey (CPS) for Miami and comparison cities
	\end{itemize}
\end{frame}


\begin{frame}[plain]
	\begin{figure}
	\includegraphics[scale=0.25]{./lecture_includes/boatlift2.png}
	\end{figure}
\end{frame}

\begin{frame}[plain]
	\begin{figure}
	\includegraphics[scale=0.25]{./lecture_includes/boatlift3.png}
	\end{figure}
\end{frame}

\begin{frame}[plain]
	\begin{figure}
	\includegraphics[scale=0.25]{./lecture_includes/boatlift4.png}
	\end{figure}
\end{frame}


\begin{frame}{Selecting control groups}

\begin{itemize}

\item His treatment group was low skill workers in Miami since that's where Cubans went
\item But which control group?
\item He chose Atlanta, Los Angeles, Houston, Tampa-St. Petersburg

\end{itemize}

\end{frame}



\begin{frame}{Why these four?}

	\begin{figure}
	\includegraphics[scale=0.25]{./lecture_includes/card_illr.png}
	\end{figure}

\end{frame}

\begin{frame}{Card's main results used diff-in-diff}
	
	\begin{figure}
	\includegraphics[scale=0.75]{./lecture_includes/abadie_2.pdf}
	\end{figure}
\end{frame}

\begin{frame}{Parallel trends}

\begin{itemize}
\item His estimate is unbiased if the change in $Y^0$ for the comparison cities correctly approximates the unobserved $\textcolor{red}{\Delta Y^0}$ for the treatment group
\item But Card largely focused on covariates, and in a relatively casual way (``similar growth'') -- this predates using event studies (also Card was not particularly impressed by the study so didn't seem to really expect it to make much of a splash)
\item The Black result would have been positive, too, were it not that the comparison cities growth was smaller
\item Is there anything principled we could do that doesn't give the researcher so much control over control group?
\end{itemize}

\end{frame}


\begin{frame}{Synthetic Control}
	
	\begin{itemize}
	\item Abadie and Gardeazabal (2003) introduces synthetic control in a study of a terrorist attack in Spain (Basque) on GDP
	\item Revisited again in a 2010 JASA with Diamond and Hainmueller, two political scientists who were PhD students at Harvard (more proofs and inference)
	\item Paper has over 5000 cites and is growing in influence over time -- very popular in tech
	\item A combination of comparison units often does a better job reproducing the characteristics of a treated unit than single comparison unit alone
	\end{itemize}
\end{frame}


\begin{frame}{Researcher's objectives}

\begin{itemize}
	\item Our goal here is to reproduce the counterfactual of a treated unit by finding the combination of untreated units that best resembles the treated unit \emph{before} the intervention in terms of the values of $k$ relevant covariates (predictors of the outcome of interest)
	\item Method selects \emph{weighted average of all potential comparison units} that best resembles the characteristics of the treated unit(s) - called the ``synthetic control''
\end{itemize}

\end{frame}

\begin{frame}{Synthetic control method: advantages}
	
	\begin{itemize}
	\item Precludes extrapolation (unlike regression) because counterfactual forms a convex hull
	\item Does not require access to post-treatment outcomes in the ``design'' phase of the study - no peeking
	\item Makes explicit the contribution of each comparison unit to the counterfactual 
	\item Formalizing the way comparison units are chosen has direct implications for inference
	\end{itemize}
\end{frame}


\begin{frame}{Synthetic control method: disadvantages}

\begin{enumerate}
\item Subjective researcher bias kicked down to the model selection stage
\item Significant diversity at the moment as to how to principally select models - from machine learning to modifications - as well as estimation and software
\end{enumerate}



\end{frame}

\begin{frame}{Avoiding cherry picking synthetic controls}

\begin{itemize}
\item Ferman, Pinto and Possbaum (2020) note that there's a ton of diversity in how the models are fit
\item Opens the door for ``cherry picking'' and remember -- part of the purpose of this procedure is to reduce subjective researcher bias
\item They conducted Monte Carlo simulations and make several recommendations for specifications and reporting all of them

\end{itemize}

\end{frame}

\begin{frame}{Avoiding cherry picking}

	\begin{figure}
	\includegraphics[scale=0.5]{./lecture_includes/cherry_picking_1.png}
	\end{figure}

\end{frame}

\begin{frame}{Avoiding cherry picking}

	\begin{figure}
	\includegraphics[scale=0.5]{./lecture_includes/cherry_picking_2.png}
	\end{figure}

\end{frame}


\begin{frame}{Synthetic control method: estimation}
	
Suppose that we observe $J+1$ units in periods $1, 2, \dots, T$
		\begin{itemize}
		\item Unit ``one'' is exposed to the intervention of interest (that is, ``treated'') during periods $T_0+1, \dots, T$
		\item The remaining $J$ are an untreated reservoir of potential controls (a ``donor pool'')
		\end{itemize}	
\end{frame}


\begin{frame}{Potential outcomes notation}

		\begin{itemize}
		\item Let $Y_{it}^0$ be the outcome that would be observed for unit $i$ at time $t$ in the absence of the intervention
		\item Let $Y_{it}^1$ be the outcome that would be observed for unit $i$ at time $t$ if unit $i$ is exposed to the intervention in periods $T_0+1$ to $T$.
		\end{itemize}

\end{frame}

\begin{frame}{Dynamic ATT}

Treatment effect parameter is defined as dynamic ATT where 

\begin{eqnarray*}
\delta_{1t}&=&Y_{1t}^1 - Y_{1t}^0 \\
&=& Y_{1t} - Y_{1t}^0 
\end{eqnarray*} for each post-treatment period, $t>T_0$ and $Y_{1t}$ is the outcome for unit one at time $t$. We will estimate $Y^0_{1t}$ using the $J$ units in the donor pool 

\end{frame}

\begin{frame}{Estimating optimal weights}
	
	\begin{itemize}
	\item Let $W=(w_2, \dots, w_{J+1})'$ with $w_j\geq 0$ for $j=2, \dots, J+1$ and $w_2+\dots+w_{j+1}=1$. Each value of $W$ represents a potential synthetic control
	\item Let $X_1$ be a $(k\times 1)$ vector of pre-intervention characteristics for the treated unit.  Similarly, let $X_0$ be a $(k\times J)$ matrix which contains the same variables for the unaffected units.
	\item The vector $W^*=(w_2^*, \dots, w_{J+1}^*)'$ is chosen to minimize $||X_1-X_0W||$, subject to our weight constraints
	\end{itemize}
\end{frame}

\begin{frame}{Optimal weights differ by another weighting matrix}
	
Abadie, et al. consider $$||X_1 - X_0W||=\sqrt{(X_1-X_0W)'V(X_1-X_0W)}$$where $X_{jm}$ is the value of the $m$-th covariates for unit $j$ and $V$ is some $(k\times k)$ symmetric and positive semidefinite matrix

\end{frame}

\begin{frame}{More on the V matrix}

Typically, $V$ is diagonal with main diagonal $v_1, \dots, v_k$.  Then, the synthetic control weights $w_2^*, \dots, w_{J+1}^*$ minimize: $$\sum_{m=1}^k v_m \bigg(X_{1m} - \sum_{j=2}^{J+1}w_jX_{jm}\bigg)^2$$ where $v_m$ is a weight that reflects the relative importance that we assign to the $m$-th variable when we measure the discrepancy between the treated unit and the synthetic controls

\end{frame}

\begin{frame}{Choice of $V$ is critical}
	
		\begin{itemize}
		\item The synthetic control $W^*(V^*)$ is meant to reproduce the behavior of the outcome variable for the treated unit in the absence of the treatment
		\item Therefore, the $V^*$ weights directly shape $W^*$
		\end{itemize}
\end{frame}

\begin{frame}{Estimating the $V$ matrix}
	
 Choice of $v_1, \dots, v_k$ can be based on
		\begin{itemize}
		\item Assess the predictive power of the covariates using regression
		\item Subjectively assess the predictive power of each of the covariates, or calibration inspecting how different values for $v_1, \dots, v_k$ affect the discrepancies between the treated unit and the synthetic control
		\item Minimize mean square prediction error (MSPE) for the pre-treatment period (default):
			\begin{eqnarray*}
			\sum_{t=1}^{T_0} \bigg(Y_{1t} - \sum_{j=2}^J w_j^*(V^*)Y_{jt} \bigg)^2
			\end{eqnarray*}
		\end{itemize}
\end{frame}

\begin{frame}{Cross validation}

\begin{itemize}
		\item Divide the pre-treatment period into an initial \textbf{training} period and a subsequent \textbf{validation} period
		\item For any given $V$, calculate $W^*(V)$ in the training period.
		\item Minimize the MSPE of $W^*(V)$ in the validation period
\end{itemize}

\end{frame}


\begin{frame}{Suppose $Y^0$ is given by a factor model}

What about unmeasured factors affecting the outcome variables as well as heterogeneity in the effect of observed and unobserved factors?
\begin{eqnarray*}
Y_{it}^0 = \alpha_t + \theta_t Z_i + \lambda_t u_i + \varepsilon_{it}
\end{eqnarray*}where $\alpha_t$ is an unknown common factor with constant factor loadings across units, and $\lambda_t$ is a vector of unobserved common factors

\end{frame}

\begin{frame}{With some manipulation}

\begin{eqnarray*}
Y^0_{1t} - \sum^{J+1}_{j=2}w^*_jY_{jt} &=& \sum_{j=2}^{J+1} w_j^* \sum_{s=1}^{T_0} \lambda_t \bigg ( \sum_{n=1}^{T_0} \lambda_n'\lambda_n \bigg )
^{-1} \lambda_s'(\varepsilon_{js} - \varepsilon_{1s} ) \\
&& - \sum_{j=2}^{J+1} w_j^* (\varepsilon_{jt} - \varepsilon_{1t})
\end{eqnarray*}

\begin{itemize}
\item If $\sum_{t=1}^{T_0} \lambda_t' \lambda_t$ is nonsingular, then RHS will be close to zero if number of preintervention periods is ``large''  relative to size of transitory shocks 
\item Only units that are alike in observables and unobservables should produce similar trajectories of the outcome variable over extended periods of time
\item Proof in Appendix B of ADH (2011)
\end{itemize}


\end{frame}


\begin{frame}{Example: California's Proposition 99}
	
	\begin{itemize}
	\item In 1988, California first passed comprehensive tobacco control legislation:
		\begin{itemize}
		\item increased cigarette tax by 25 cents/pack
		\item earmarked tax revenues to health and anti-smoking budgets
		\item funded anti-smoking media campaigns
		\item spurred clean-air ordinances throughout the state
		\item produced more than \$100 million per year in anti-tobacco projects
		\end{itemize}
	\item Other states that subsequently passed control programs are excluded from donor pool of controls (AK, AZ, FL, HI, MA, MD, MI, NJ, OR, WA, DC)
	\end{itemize}
\end{frame}

\begin{frame}{Cigarette Consumption: CA and the Rest of the US}
	
	\begin{figure}
	\includegraphics[scale=0.75]{./lecture_includes/abadie_3.pdf}
	\end{figure}
\end{frame}

\begin{frame}{Cigarette Consumption: CA and synthetic CA}
	
	\begin{figure}
	\includegraphics[scale=0.75]{./lecture_includes/abadie_4.pdf}
	\end{figure}
\end{frame}

\begin{frame}{Predictor Means: Actual vs. Synthetic California}
	
	\begin{figure}
	\includegraphics[scale=0.75]{./lecture_includes/abadie_5.pdf}
	\end{figure}
\end{frame}

\begin{frame}{Smoking Gap between CA and synthetic CA}
	
	\begin{figure}
	\includegraphics[scale=0.75]{./lecture_includes/abadie_6.pdf}
	\end{figure}
\end{frame}

\begin{frame}{Inference}
	
	\begin{itemize}
	\item To assess significance, we calculate exact p-values under Fisher's sharp null using a test statistic equal to after to before ratio of RMSPE
	\item Exact p-value method
		\begin{itemize}
		\item Iteratively apply the synthetic method to each country/state in the donor pool and obtain a distribution of placebo effects
		\item Compare the gap (RMSPE) for California to the distribution of the placebo gaps. For example the post-Prop. 99 RMSPE is: 
			\begin{eqnarray*}
			RMSPE = \bigg(\frac{1}{T-T_0} \sum_{t=T_0+1}^T \bigg(Y_{1t} - \sum_{j=2}^{J+1} w_j^* Y_{jt}\bigg)^2 \bigg)^{\frac{1}{2}}
			\end{eqnarray*}and the exact p-value is the treatment unit rank divided by $J$
		\end{itemize}
	\end{itemize}
\end{frame}

\begin{frame}{Smoking Gap for CA and 38 control states}
	
	\begin{figure}
	\includegraphics[scale=0.75]{./lecture_includes/abadie_7.pdf}
	\end{figure}
\end{frame}

\begin{frame}{Smoking Gap for CA and 34 control states}
	
	\begin{figure}
	\includegraphics[scale=0.75]{./lecture_includes/abadie_8.pdf}
	\end{figure}
\end{frame}

\begin{frame}{Smoking Gap for CA and 29 control states}
	
	\begin{figure}
	\includegraphics[scale=0.75]{./lecture_includes/abadie_9.pdf}
	\end{figure}
\end{frame}

\begin{frame}{Smoking Gap for CA and 19 control states}
	
	\begin{figure}
	\includegraphics[scale=0.75]{./lecture_includes/abadie_10.pdf}
	\end{figure}
\end{frame}

\begin{frame}{Ratio Post-Prop. 99 RMSPE to Pre-Prop. 99 RMSPE}

	\begin{figure}
	\includegraphics[scale=0.75]{./lecture_includes/abadie_11.pdf}
	\end{figure}
\end{frame}


\begin{frame}{Criticism of weights}

\begin{itemize}
\item Prominent economist discussed Abadie, Diamond and Hainmueller (2015), a study on the reunification of Germany
\item Incredulous and dismissive about the Japan weight of 0.16 for Germany
\item But compare the weights used by regression with the weights used by synthetic control -- OLS weight on Japan is larger plus we get all these negative weights
\item We're going to return to this issue about over fitting, negative weighting and extrapolation with regression -- but for now I thought it was interesting that synth was punished for being too honest
\end{itemize}

\end{frame}


\begin{frame}{Negative weights}

\begin{figure}
\includegraphics[scale=0.35]{./lecture_includes/abadie2015_table1}
\end{figure}
\end{frame}


\begin{frame}{Coding exercise: Texas prison construction}
	
	\begin{itemize}
	\item The US has the highest prison population of any OECD country in the world 
	\item 2 million are currently incarcerated in US federal and state prisons and county jails
	\item Another 4.75 million are on parole
	\item From the early 1970s to the present, incarceration and prison admission rates quintupled in size
	\end{itemize}
\end{frame}



\begin{frame}[plain]

\begin{figure}
\includegraphics[scale=0.5]{./lecture_includes/cook2010.pdf}
\end{figure}
\end{frame}


\begin{frame}{Managing inmate growth}

	
	\begin{itemize}
	\item Prisons were at capacity so states had to make changes to adjust for growth by doing one or all of the following
		\begin{itemize}
		\item Prison construction
		\item Overcrowding
		\item Paroles
		\end{itemize}
	\item Guess how Texas was managing its capacity problems
	\end{itemize}
\end{frame}



\begin{frame}{Texas Dept of Corrections loses human rights case}

Ruiz v. Estelle 1980  
		\begin{itemize}
		\item Class action lawsuit against TX Dept of Corrections (Estelle, warden). 
		\item TDC surprisingly lost, led to years of appeals and legal decrees
		\item Lengthy period of time relying on paroles to manage flows
		\end{itemize}
\end{frame}



\begin{frame}[shrink=30,plain]

\begin{figure}
\includegraphics{./lecture_includes/flow_rate_figure.pdf}
\end{figure}

Rising paroles, but what's that red line?
\end{frame}


	

\begin{frame}{Texas prison expansion}

 Governor Ann Richards (D) serves from 1991-1995
		\begin{itemize}
		\item Secures funding from Congress to build new prisons (public and private)
		\item Operation prison capacity increased 30-35\% in 1993, 1994 and 1995. 
		\item Or put differently, prison capacity increased from 55,000 in 1992 to 130,000 in 1995.  
		\end{itemize} 
\end{frame}


\begin{frame}[shrink=30,plain]

\begin{figure}
\includegraphics{./lecture_includes/tdcj.pdf}
\end{figure}
\end{frame}


\begin{frame}[shrink=30,plain]
\begin{figure}
\includegraphics{./lecture_includes/capacity_operational_texas.pdf}
\end{figure}
\end{frame}

\begin{frame}[shrink=30,plain]


\begin{figure}
\includegraphics{./lecture_includes/total_incarceration.pdf}
\end{figure}
\end{frame}

\begin{frame}{Synthetic control}

\begin{itemize}
\item Let's work together to see how synthetic control would handle this
\item Using the ADH canonical model, we will be estimating the dynamic ATT parameters per year for Texas by comparing with a non-negatively weighted control group
\item We will use \texttt{synth} in Stata, R and python, but I'll teach with Stata first as it's faster
\end{itemize}

\end{frame}







\subsection{Augmented Synthetic Control}

\begin{frame}{Comments about negative weights}

\begin{itemize}

\item Abadie cautions against negative weights, even if it improves fit, because it necessarily extrapolates beyond the support of the data
\item King and Zeng (2006) note, ``if we learn that a counterfactual question involves extrapolation, we still might wish to proceed if the question is sufficiently important, but we would be aware of how much model dependent our answers would be.''
\item Much of the newer work has been to carefully balance issues around poor fit (bias) with negative weights (extrapolation)
\item Now we introduce the augmented synthetic control model (Ben-Michael, Feller and Rothstein 2021, JASA)

\end{itemize}

\end{frame}


\begin{frame}{Why Fix What Isn't Broken?}

\begin{quote}
``The applicability of the [ADH2010] method requires a sizable number of pre-intervention periods. The reason is that the credibility of a synthetic control depends upon how well it tracks the treated unit’s characteristics and outcomes over an extended period of time prior to the treatment. \textbf{We do not recommend using this method when the pretreatment fit is poor or the number of pretreatment periods is small}. A sizable number of post-intervention periods may also be required in cases when the effect of the intervention emerges gradually after the intervention or changes over time.'' (my emphasis, Abadie, et al. 2015)
\end{quote}

\end{frame}

\begin{frame}{What is augmented synthetic control?}

\begin{itemize}
\item Recall the quote from earlier by Athey and Imbens -- ``synthetic control is the most important innovation in causal inference of the last 15 years''
\item A lot of activity on synth in last few years, not counting two ``augmented synthetic control'' papers by Eli Ben-Michael, Avi Feller and Jesse Rothstein
\item First model will apply in single case situations with one treatment data with pre-treatment imbalance
\end{itemize}

\end{frame}


\begin{frame}{Why and when to augment}

\begin{itemize}
\item ADH will be biased in practical settings because it needs perfect pre-treatment fit and that's difficult to achieve in many applications 
\item Their augmentation estimates the bias caused by covariate imbalance with an outcome model
\item Introduces ridge regularization linear regression to estimate new weights that reweight synth
\item It's similar to ``bias reduction'' by Abadie and Imbens (2011), it will have doubly robust properties and inverse probability weighting
\item When synth is imbalanced, augmented synth will reduce bias reweighting and bias correction, and when synth is balanced, they are the same

\end{itemize}

\end{frame}

\begin{frame}{Comments on negative weights}

	\begin{itemize}
	\item Relaxes original ADH constraints that required non-negative weights at the expense of allowing extrapolation
	\item This is done to correct for the bias caused by pre-treatment imbalance (``poor fit'')
	\item Negative weights puts the treated unit back in the convex hull through extrapolation
	\end{itemize}

\end{frame}



\begin{frame}{Same Notation}

\begin{itemize}
\item Observe $J+1$ units over $T$ time periods
\item Unit $1$ will be treated at time period $T_0=T-1$ (we allow for unit $1$ to be an average over treated units)
\item Units $j=2 $ to $J+1$ (using ADH original notation) are ``never treated''
\item $D_j$ is the treatment indicator
\end{itemize}

\end{frame}


\begin{frame}[plain,shrink=20]
\begin{center}
\textbf{Pre-treatment covariates and outcomes}
\end{center}

\begin{center}
\[ \left( \begin{array}{ccccc}
    Y_{11} & Y_{12} & Y_{13} & \dots  & Y_{1T}^1 \\
    Y_{21} & Y_{22} & Y_{23} & \dots  & Y_{2T}^0  \\
    \vdots & \vdots & \vdots & \ddots & \vdots \\
    Y_{N1} & Y_{i2} & Y_{i3} & \dots  & Y_{NT}^0
\end{array} \right) \equiv
\left( \begin{array}{ccccc}
    X_{11} & X_{12} & X_{13} & \dots  & Y_{1} \\
    X_{21} & X_{22} & X_{23} & \dots  & Y_{2}  \\
    \vdots & \vdots & \vdots & \ddots & \vdots \\
    X_{N1} & X_{i2} & X_{i3} & \dots  & Y_{N}
\end{array} \right) \equiv
\left( \begin{array}{cc}
    X_{1} & Y_{1} \\
    X_{0} & Y_{0}  \\
\end{array} \right)
\]

\end{center}

This is a 2x2 in diff-in-diff parlance (i.e., single last period block structure, no staggered roll out)

\bigskip

The last column is always post-treatment and switches from $Y^1$ to $Y$ via switching equation

\bigskip

The last column is just showing a top row of the treated unit 1 and the bottom row of all the donor pool (i.e., we will use $X_0$ and $Y_0$ to represent all the donor pool units)

\end{frame}


\begin{frame}{Optimal weights}

Synth minimizes the following norm:

\begin{eqnarray*}
\textrm{min}_w = || V_X^{1/2} (X_1 - X_0'w) ||_2^2 + \psi \sum_{D_j=0}f(w_j)\\
\textrm{s.t. }\sum_{j=2}^N w_{j} =1 \textrm{ and } w_j \geq 0
\end{eqnarray*}

$Y_0'w*$ (i.e., optimally weighted donor pool) is the unit 1 ``synthetic control'' 

\end{frame}


\begin{frame}{Predicting counterfactuals}

Synth minimizes the following norm:

\begin{eqnarray*}
\textrm{min}_w = || V_X^{1/2} (X_1 - X_0'w) ||_2^2 + \psi \sum_{D_j=0}f(w_j)\\
\textrm{s.t. }\sum_{j=2}^N w_{j} =1 \textrm{ and } w_j \geq 0
\end{eqnarray*}

We are hoping that $\widehat{Y}_1^0$ with $Y_0' {w}^{*}$ based on ``perfect fit'' pre-treatment

\end{frame}




\begin{frame}{$V_X$ matrix}

Synth minimizes the following norm:

\begin{eqnarray*}
\textrm{min}_w = || V_X^{1/2} (X_1 - X_0'w) ||_2^2 + \psi \sum_{D_j=0}f(w_j)\\
\textrm{s.t. }\sum_{j=2}^N w_{j} =1 \textrm{ and } w_j \geq 0
\end{eqnarray*}

$V_x$ is the ``importance'' matrix on $X_0$ (Some packages default to choose $V^*_x$ that minimizes pre-treatment MSE, but it can be identity matrix for simplicity).
\end{frame}

\begin{frame}{Penalizing the weights with ridge}

Synth minimizes the following norm:

\begin{eqnarray*}
\textrm{min}_w = || V_X^{1/2} (X_1 - X_0'w) ||_2^2 + \psi \sum_{D_j=0}f(w_j)\\
\textrm{s.t. }\sum_{j=2}^N w_{j} =1 \textrm{ and } w_j \geq 0
\end{eqnarray*}

Modification to the original synthetic control model is the inclusion of the penalty term. ``The choice of penalty is less central when weights are constrained to be on the simplex, but becomes more important when we relax this constraint.''

\end{frame}

\begin{frame}{Convex hull}

Synth minimizes the following norm:

\begin{eqnarray*}
\textrm{min}_w = || V_X^{1/2} (X_1 - X_0'w) ||_2^2 + \psi \sum_{D_j=0}f(w_j)\\
\textrm{s.t. }\sum_{j=2}^N w_{j} =1 \textrm{ and } w_j \geq 0
\end{eqnarray*}

These weights are used to address imbalance, not to weight the donor units, bc we are using this method is for when the weighted controls are still outside the convex hull (poor fit)

\end{frame}




\begin{frame}{Original ADH factor model and bias}

\begin{eqnarray*}
Y_{it}^0 = \alpha_t + \theta_t Z_i + \lambda_t u_i + \varepsilon_{it}
\end{eqnarray*}

\bigskip

Original synth factor model (with ADH notation)

\bigskip

\begin{eqnarray*}
Y^0_{1t} - \sum^{J+1}_{j=2}w^*_jY_{jt} &=& \sum_{j=2}^{J+1} w_j^* \sum_{s=1}^{T_0} \lambda_t \bigg ( \sum_{n=1}^{T_0} \lambda_n'\lambda_n \bigg )
^{-1} \lambda_s'(\varepsilon_{js} - \varepsilon_{1s} ) \\
&& - \sum_{j=2}^{J+1} w_j^* (\varepsilon_{jt} - \varepsilon_{1t})
\end{eqnarray*}

\bigskip

The bias of ADH synthetic control


\end{frame}




\begin{frame}{Perfect fit is necessary}

\begin{eqnarray*}
Y^0_{1t} - \sum^{J+1}_{j=2}w^*_jY_{jt} &=& \sum_{j=2}^{J+1} w_j^* \sum_{s=1}^{T_0} \lambda_t \bigg ( \sum_{n=1}^{T_0} \lambda_n'\lambda_n \bigg )
^{-1} \lambda_s'(\varepsilon_{js} - \varepsilon_{1s} ) \\
&& - \sum_{j=2}^{J+1} w_j^* (\varepsilon_{jt} - \varepsilon_{1t})
\end{eqnarray*}

\bigskip

Recall that the bias of ADH required ``perfect fit'' using their factor model (I'll change $\lambda$ factor loadings in a minute)

\end{frame}




\begin{frame}{Perfect fit models heterogeneity}


\begin{eqnarray*}
Y^0_{1t} - \sum^{J+1}_{j=2}w^*_jY_{jt} &=& \sum_{j=2}^{J+1} w_j^* \sum_{s=1}^{T_0} \lambda_t \bigg ( \sum_{n=1}^{T_0} \lambda_n'\lambda_n \bigg )
^{-1} \lambda_s'(\varepsilon_{js} - \varepsilon_{1s} ) \\
&& - \sum_{j=2}^{J+1} w_j^* (\varepsilon_{jt} - \varepsilon_{1t})
\end{eqnarray*}

Only units that are alike in observables and unobservables should produce similar trajectories of the outcome variable over extended periods of time


\end{frame}


\begin{frame}{Call back to the earlier ADH15 quote}

\begin{quote}
``The applicability of the [ADH2010] method requires a sizable number of pre-intervention periods. The reason is that the credibility of a synthetic control depends upon how well it tracks the treated unit’s characteristics and outcomes over an extended period of time prior to the treatment. \textbf{We do not recommend using this method when the pretreatment fit is poor or the number of pretreatment periods is small}. A sizable number of post-intervention periods may also be required in cases when the effect of the intervention emerges gradually after the intervention or changes over time.'' (my emphasis, Abadie, et al. 2015)
\end{quote}

\end{frame}

\begin{frame}{Slight change in synth notation}

\begin{itemize}
\item Assume that our outcome, $Y_{jt}$, follows a factor model where $m(.)$ are pre-treatment outcomes: $$ Y_{jt}^0 = m_{jt} + \varepsilon_{jt}$$
\item Since $\widehat{m(.)}$ estimates the post-treatment outcome, let's view it as estimated bias, analogous to bias correction for inexact matching (Abadie and Imbens 2011)
\end{itemize}

\end{frame}



\begin{frame}{Bias correction}

 $$ Y_{jt}^0 = m_{jt} + \varepsilon_{jt}$$

\begin{itemize}
\item When the weights achieve exact balance, the bias of synthetic control decreases with $T$
\item The intuition is that for a large $T$ ($T$ not transitory shocks), you achieve balance by balancing the latent parameter on the unobserved heterogeneity in our factor model
\end{itemize}

\end{frame}



\begin{frame}{Common practice}

\begin{itemize}
\item Length of panel $T$ varies by dataset and question
\item Oftent the number of time periods isn't much larger than the number of units
\item But even when it does, ``exact balance'' is exception not the rule
\item Without exact balance, unobserved heterogeneity won't get deleted and bias is introduced
\end{itemize}

\end{frame}


\begin{frame}{Treatment and control units}

	\begin{figure}
	\includegraphics[scale=0.07]{./lecture_includes/convexhull_1.png}
	\end{figure}

\end{frame}

\begin{frame}{Convex hull (good fit)}

	\begin{figure}
	\includegraphics[scale=0.07]{./lecture_includes/convexhull_2.png}
	\end{figure}

\end{frame}

\begin{frame}{Outside the convex hull (bad fit)}

	\begin{figure}
	\includegraphics[scale=0.07]{./lecture_includes/convexhull_3.png}
	\end{figure}

\end{frame}

\begin{frame}{Use negative weights to get back}

	\begin{figure}
	\includegraphics[scale=0.07]{./lecture_includes/convexhull_4.png}
	\end{figure}

\end{frame}


\begin{frame}{Estimating the bias}

\begin{itemize}
\item So the idea is to adjust the synthetic control when we have poor fit pre-treatment.
\item Use $\widehat{m}_{jT}$ from our factor model as an estimator for the post-treatment control potential outcome $Y_{jT}^0$.
\item What does this look like?
\end{itemize}


\end{frame}




\begin{frame}{Setup of the estimator}

Let's adjust synthetic control for this bias.  First we'll apply the \textbf{bias correction}.  Then we'll do the doubly robust augmented \textbf{inverse probability weighting}. Let $Y_1^{aug,0}$ be the augmented potential outcome

\begin{eqnarray*}
Y_1^{aug,0} &=& \sum_{D_j=0} \widehat{w}_j^{synth} Y_{j} + \widehat{m}(X_1) - \sum_{D_j=0} \widehat{w}_j \widehat{m}(X_j) \\
&=& \widehat{m}(X_1) + \sum_{D_j=0} \widehat{w_j}(Y_j - \widehat{m}(X_j))
\end{eqnarray*}

\end{frame}


\begin{frame}{Interpreting line 1}

\begin{eqnarray*}
Y_1^{aug,0} &=& \underbrace{\sum_{D_j=0} \widehat{w}_j^{synth} Y_{jT}}_{\mathclap{Original\ ADH}} + \bigg (\widehat{m}_{1T} - \sum_{D_j=0} \widehat{w}_j^{synth}\widehat{m}_{jT} \bigg ) \\
&=& \widehat{m}_{1T} + \sum_{D_j=0} \widehat{w}_j^{synth} (Y_{jT} - \widehat{m}_{jT})
\end{eqnarray*}

(1) Note how in the first line the traditional synthetic control weighted outcomes are corrected by the imbalance in a particular function of the pre-treatment outcomes $\widehat{m}$. 
\end{frame}




\begin{frame}{Interpreting line 1}

\begin{eqnarray*}
Y_1^{aug,0}  &=& \sum_{D_j=0} \widehat{w}_j^{synth} Y_{jT} + \bigg (\underbrace{\widehat{m}_{1T}}_{Bias\ correction} - \sum_{D_j=0} \widehat{w}_j^{synth}\widehat{m}_{jT} \bigg ) \\
&=& \widehat{m}_{1T} + \sum_{D_j=0} \widehat{w}_j^{synth} (Y_{jT} - \widehat{m}_{jT})
\end{eqnarray*}

(1) Since $\widehat{m}$ estimates the post-treatment outcome, we can view this as an estimate of the bias due to imbalance, which is similar to how you address imbalance in matching with a bias correction formula (Abadie and Imbens 2011). 

\end{frame}







\begin{frame}{Interpreting line 1}
\begin{eqnarray*}
Y_1^{aug,0}  &=& \sum_{D_j=0} \widehat{w}_j^{synth} Y_{jT} + \bigg (\widehat{m}_{1T} - \sum_{D_j=0} \widehat{w}_j^{synth}\widehat{m}_{jT} \bigg ) \\
&=& \widehat{m}_{1T} + \sum_{D_j=0} \widehat{w}_j^{synth} (Y_{jT} - \widehat{m}_{jT})
\end{eqnarray*}

(1) So if the bias is small, then synthetic control and augmented synthetic control will be similar because that interior term will be zero.

\end{frame}

\begin{frame}{Interpreting line 2}

\begin{eqnarray*}
Y_1^{aug,0}  &=& \sum_{D_j=0} \widehat{w}_j^{synth} Y_{jT} + \bigg (\widehat{m}_{1T} - \sum_{D_j=0} \widehat{w}_j^{synth}\widehat{m}_{jT} \bigg ) \\
&=& \widehat{m}_{1T} + \sum_{D_j=0} \widehat{w}_j^{synth} (Y_{jT} - \widehat{m}_{jT})
\end{eqnarray*}

(2) The second equation is equivalent to a double robust estimation which begins with an outcome model but then re-weights it to balance residuals.


\end{frame}



\begin{frame}{Interpreting line 2}

\begin{eqnarray*}
Y_1^{aug,0}  &=& \sum_{D_j=0} \widehat{w}_j^{synth} Y_{jT} + \bigg (\widehat{m}_{1T} - \sum_{D_j=0} \widehat{w}_j^{synth}\widehat{m}_{jT} \bigg ) \\
&=& \widehat{m}_{1T} + \sum_{D_j=0} \widehat{w}_j^{synth} (Y_{jT} - \widehat{m}_{jT})
\end{eqnarray*}

(2) The second equation has a connection to inverse probability weighting (they show this in an appendix)


\end{frame}
\begin{frame}{Ridge Augmented SCM}

\begin{eqnarray*}
\textrm{arg min}_{\eta_0,\eta} \frac{1}{2} \sum_{D_j=0} (Y_j - (\eta_0 + X_j'\eta))^2 + \lambda^{ridge} || \eta ||_2^2
\end{eqnarray*}Here we estimate $\widehat{m}(X_j)$ with ridge regularized linear model and penalty hyper parameter $\lambda^{ridge}$. (Sorry -- this is not the same $\lambda$ as in ADH10; author team is using the same $\lambda$ for different things).

\bigskip

Once we have those, we adjust for imbalance using the $\widehat{\eta}^{ridge}$ parameter as a weight on the outcome model itself. 

\end{frame}

\begin{frame}{Ridge Augmented SCM}

\begin{eqnarray*}
\textrm{arg min}_{\eta_0,\eta} \frac{1}{2} \sum_{D_j=0} (Y_j - (\eta_0 + X_j'\eta))^2 + \lambda^{ridge} || \eta ||_2^2
\end{eqnarray*}Once we have those, we adjust for imbalance using the $\widehat{\eta}^{ridge}$ parameter as a weight on the outcome model itself. 

\end{frame}




\begin{frame}{Go back to that weighting but use the ridge parameters}

\begin{eqnarray*}
Y_1^{aug,0} &=& \sum_{D_j=0} \widehat{w}_j^{synth} Y_{j} + \bigg ( X_1 - \sum_{D_j=0} \widehat{w}_j^{synth} X_j \bigg ) \widehat{\eta}^{ridge} \\
&=& \sum_{D_j=0} \widehat{w}_j^{aug}Y_j
\end{eqnarray*}What you're trying to do is adjust with the $\widehat{w}_j^{aug}$ weights to improve balance.  

\end{frame}


\begin{frame}{The ridge weights are key to the augmentation}

\begin{eqnarray*}
\widehat{w}_j^{aug} = \widehat{w}_j^{synth} + (X_j - X_0' \widehat{w}_j^{synth}) ' (X_0'X_0 + \lambda I_{T_0})^{-1}X_i
\end{eqnarray*}

The second term is adjusting the original synthetic control weights, $w_j^{synth}$ for better balance. 

\bigskip

Don't lose sight of the goal: we are trying to address the bias due to imbalance. 

\bigskip

We always could get better balance, but at a cost of higher variance and negative weights. 

\end{frame}



\begin{frame}{Ridge will allow negative weights via extrapolation}

\begin{eqnarray*}
\widehat{w}_j^{aug} = \widehat{w}_j^{synth} + (X_j - X_0' \widehat{w}_j^{synth}) ' (X_0'X_0 + \lambda I_{T_0})^{-1}X_i
\end{eqnarray*}

Relaxing the constraint from synth that weights be non-negative, as non-negative weights prohibit extrapolation. Since we aren't in the convex hull, if we are to proceed at all, we \emph{must} extrapolate, otherwise synth will be biased.

\end{frame}



\begin{frame}{Summarizing and some comments}

\begin{itemize}
\item When the treated unit lies in the convex hull of the control units so that the synth weights exactly balance lagged outcomes, then SCM and Ridge ASCM are the same
\item When synth weights do not achieve exact balance, Ridge ASCM will use negative weights to extrapolate from the convex hull to the control units
\item The amount of extrapolation will be determined by how much imbalance we're talking about and the estimated hyperparameter $\widehat{\lambda}^{ridge}$
\item When synth has good pre-treatment fit or when $\lambda^{ridge}$ is large, then adjustment will be small and the augmented weights will be close to the SCM weights
\end{itemize}

\end{frame}



\begin{frame}{Intuition}

Ridge begins at the center of control units, while Ridge ASCM begins at the synth solution. Both move towards an exact fit solution as the hyperparameter is reduced. It is possible to achieve the same level of balance with non-negative weights.  Both ridge and Ridge ASCM extrapolate from the support of the data to improve pre-treatment fit relative to synth alone. Let's look at a picture!


\end{frame}



\begin{frame}[plain]

	\begin{figure}
	\includegraphics[scale=0.5]{./lecture_includes/aug_1.png}
	\end{figure}
	
\end{frame}	


\begin{frame}{Conformal Inference}

Inference will be based on ``conformal inference'' method by Chernozhukov et al. (2019).  We will get 95\% point-wide confidence intervals. They also outline a jackknife method by Barber et al (2019). 

\end{frame}


\begin{frame}{Steps of conformal Inference}

\begin{enumerate}
\item [1.] Choose a sharp null (i.e., no unit-level treatment effects, $\delta_0=0$)
	\begin{itemize}
	\item Enforce the null by creating an adjusted post-treatment outcome for the treated unit equal to $Y_{1T}-\delta_0$ (in other words, we get CI on the post-treatment outcomes, not the pre-treatment)
	\item Augment the original dataset to include the post-treatment time period $T$ with the adjusted outcome and use the estimator to obtain the adjusted weights $\widehat{w(\delta_0)}$
	\item Compute a p-value by assessing whether the adjusted residual conforms with the pre-treatment residuals (see Appendix A for the exact formula)
	\end{itemize}
\end{enumerate}


\end{frame}


\begin{frame}{Steps of conformal Inference}

\begin{enumerate}
\item [2.] Compute a level $\alpha$ for $\delta$ by inverting the hypothesis test (see Appendix A for the exact formula)
	\begin{itemize}
	\item Chernozhukov et al. (2019) provide several conditions for which approximate or exact finite-sample validity of the $p$-values (and hence coverage of the predicted confidence intervals) can be achieved
	\end{itemize}
\end{enumerate}

See Appendix A for more details

\end{frame}


\begin{frame}{Simulations (summarized)}

\begin{itemize}
\item They examine the performance of synth against ridge, Augmented synth with ridge regularization, demeaned synth, and fixed effects under four DGP
\item Augmenting synth with a ridge outcome regression reduces bias relative to synth alone in all four simulations
\item This underscores the importance of the recommendation Abadie, et al. (2015) make which is that synth should be used in settings with excellent pre-treatment fit
\item They also examine a real situation involving Kansas tax cuts in 2012
\end{itemize}

\end{frame}

\imageframe{./lecture_includes/aug_5.png}

\imageframe{./lecture_includes/aug_6.png}

\begin{frame}{Couple of minor points}

\begin{itemize}
\item Hyper parameter chosen using cross validation
\item This can be extended to auxiliary covariates as opposed to just lagged outcomes (section 6)
\end{itemize}

\end{frame}



\begin{frame}{Some minor points}

\begin{itemize}
\item We've motivated augmented synth as a kind of bias correction, but you can also think of it as correcting synth with an inverse probability weight (Appendix E)
\item There's an implicit estimate of a propensity score model with ridge regularization
\item Weights are odds of treatment (they're ATT weights), i.e., they're the inverse probability weighting scheme from Abadie (2005)
\end{itemize}

\end{frame}


\begin{frame}{Concluding remarks}

\begin{itemize}
\item Synthetic control is used with small $N$ comparison units, long pre-treatment $T$ periods and excellent pre-treatment fit
\item ADH interpolates its counterfactual by restricting weights to be non-negative and sum to one
\item When fit is poor, ridge regression augmentation introduces some degree of extrapolation to correct for imbalance bias using negative weighting
\item Augmented synth is no worse than ADH and with imbalance dominates
\end{itemize}

\end{frame}

\begin{frame}{Python code}

Shaleen Swarup (2021), ``Understanding Causal Inference with Synthetic Control method and implementing it in Python'', 
\url{https://towardsdatascience.com/causal-inference-with-synthetic-control-in-python-4a79ee636325}

\bigskip

Matheus Facure Alves (2022), \underline{Causal Inference for The Brave and True}, \url{https://matheusfacure.github.io/python-causality-handbook/15-Synthetic-Control.html} 

\bigskip

 \url{https://matheusfacure.github.io/python-causality-handbook/Conformal-Inference-for-Synthetic-Control.html}

\end{frame}



\begin{frame}{R code}

R vignette: \url{https://github.com/ebenmichael/augsynth/blob/master/vignettes/singlesynth-vignette.md}

\bigskip

Let's look at the Texas example ourselves so we can study the R syntax

\end{frame}

\subsection{Augmented Synthetic Control with Staggered Rollout}

\begin{frame}

\begin{itemize}
\item ADH was designed for a single treated unit, no extrapolation, non-negative weights summed to one
\item Previous Ben-Michael, Feller and Rothstein (2021a) paper addressed imperfect fit in the pre-trends using regularization
\item Ben-Michael, Feller and Rothstein (2021b) focus on the single unit by allowing differential timing
	\begin{itemize}
	\item Augmented synth is a double-robust style (or bias corrected) estimator
	\item This synth is similar to ``shrunken''/empirical Bayes/random effects estimation
	\end{itemize}
\item It sort of fits with the newer differential timing papers, like matrix completion with nuclear norm regularization had, even though neither are technically DiD
\item More machine learning regularization as we've been seeing
\end{itemize}

\end{frame}

\begin{frame}{Paper's Contribution}

\begin{enumerate}
	\item Extend synth to staggered adoption (as opposed to one unit)
	\item Show results using an example of unions on spending
	\item Propensity score weighting with shrinkage
\end{enumerate}

\end{frame}

\begin{frame}{Motivation}


Synthetic control is not a very good propensity score estimator
\begin{itemize}
\item Uses pre-treatment outcomes as covariates (some use other $X$ covariates)
\item Small $N$ and large $K$ means propensity score must be regularized with probability approaching 1 so perfect balance is not achieved
\item You can borrow information to reduce bias from imbalance -- for instance from an outcome model (e.g., a regression) and/or other treated units
\item You can then combine to produced a weighted event study estimator
\end{itemize}

\end{frame}

\begin{frame}{Stepping back}

\begin{itemize}
\item A unit is treated at some period $t$ and we want to know that event's effect on $Y$
\item Standard approach is DiD and event studies
	\begin{itemize}
	\item Tons of papers recently (e.g., Goodman-Bacon 2021, Callaway and Sant'Anna 2020)
	\item But what units all DiD papers is \emph{parallel trends}
	\end{itemize}
\item That assumption may be wrong, in which case the models' findings will be wrong
\end{itemize}

\end{frame}


\begin{frame}{Synth with staggered adoption}

\begin{itemize}
\item Synth was designed originally for the comparative case study -- i.e., one treatment group
\item What do we do when there's more than one treatment group?
\item People in the past tried different things
	\begin{itemize}
	\item If they were all treated at the same time, they'd average the treated units and construct a synthetic control for the average
	\item If it was staggered, then they'd fit synth for each treated unit separately, then average those estimates
	\end{itemize}
\item They're going to propose optimizing a weighted average of the \emph{global balance} (for the average treated unit) and the sum of \emph{unit-specific balance} for each treated unit
\end{itemize}

\end{frame}

\begin{frame}{Intuition}

We want to balance the average of the underlying factor loadings

\begin{itemize}
\item Balancing individual units may cause large imbalance in the average if errors all go in the same direction
\item Balancing the average outcome may not balance factor loadings if imbalance for different treated units offset each other
\end{itemize}

\end{frame}


\begin{frame}{Teacher unions and teacher salaries/spending}

Their application is about teacher unions
\begin{itemize}
\item 1964-1987: 33 states grant collective bargaining rights to teachers
\item Long literature exploited the timing (Hoxby 1996; Lovenheim 2009)
\item Impact on student spending, teacher salaries
	\begin{itemize}
	\item Hoxby (1996) finds increased spending by 12\%
	\item Paglayan (2019) estimates precise zero in an event study model using ever-treated states
	\end{itemize}
\item They're going to re-analyze using all states and synth models
\end{itemize}

\end{frame}



\imageframe{./lecture_includes/augsynth_7.png}

\begin{frame}{Their paper}

\begin{enumerate}
\item \textbf{Methods}: Extend synth to staggered adoption
\item \textbf{Substance}: Application will find minimal effect of unions on spending
\item \textbf{Connection}: Propensity score weighting with shrinkage
\end{enumerate}

\end{frame}



\imageframe{./lecture_includes/augsynth_8.png}

\begin{frame}{Synth treatment effect: average all the synths}

\begin{itemize}
\item Suppose the first $J$ units are treated at times $T_1, \dots, T_J$
\item Suppose we find a synthetic control for each, with $w_{ij}$ the weight on donor unit $i$ for treated unit $j$
\item Our estimate of the ATT at event time $k$ will then be
\end{itemize}

\begin{eqnarray*}
\widehat{\delta} = \frac{1}{J} \sum_{j=1}^{J+1} \bigg ( Y_{j,T_j+k} - \sum_i w_{ij} Y_{i,T_j+k} \bigg )
\end{eqnarray*}Average of $J$ separate synth estimates

\end{frame}


\begin{frame}{Synth treatment effect: average treated unit}

Or, we can think of it as Synth estimate for average treated unit

\begin{eqnarray*}
\widehat{\delta} = \frac{1}{J} \sum_{j=2}^{J+1} Y_{j,T_j+k} - \frac{1}{J} \bigg ( \sum_{j=2}^{J+1} \sum_i w_{ij} Y_{i,T_j+k} \bigg )
\end{eqnarray*}

\end{frame}


\begin{frame}{Two definitions of ATT}


\begin{eqnarray*}
\widehat{\delta} &=& \frac{1}{J} \sum_{j=1}^J \bigg ( Y_{j,T_j+k} - \sum_i w_{ij} Y_{i,T_j+k} \bigg ) \\
&=& \frac{1}{J} \sum_{j=2}^{J+1} Y_{j,T_j+k} - \frac{1}{J} \bigg ( \sum_{j=2}^{J+1} \sum_i w_{ij} Y_{i,T_j+k} \bigg )
\end{eqnarray*}

\end{frame}

\begin{frame}{Optimization problem}

Do we want to optimize the sum of the separate imbalances or the imbalance of the sum (the pooled imbalance)?

\begin{eqnarray*}
\sum_{j=2}^{J+1} || X_j - \sum_i w_{ij}X_i ||^2\textrm{   or  } ||\sum_{j=2}^{J+1} X_j - \sum_i w_{ij} X_i ||^2
\end{eqnarray*}where $j$ is treatment group and $i$ is donor pool units. Notice summations are inside or outside the norm
\end{frame}

\imageframe{./lecture_includes/augsynth_9.png}

\imageframe{./lecture_includes/augsynth_10.png}

\imageframe{./lecture_includes/augsynth_11.png}

\imageframe{./lecture_includes/augsynth_12.png}

\imageframe{./lecture_includes/augsynth_13.png}

\imageframe{./lecture_includes/augsynth_14.png}

\imageframe{./lecture_includes/augsynth_15.png}

\imageframe{./lecture_includes/augsynth_16.png}

\imageframe{./lecture_includes/augsynth_17.png}

\imageframe{./lecture_includes/augsynth_18.png}



\begin{frame}{Proposal: Partially pool synth}

Instead of minimizing pooled imbalance or average state imbalance, minimize a \emph{weighted average}:

\begin{eqnarray*}
\textrm{min  }_{\Gamma \in \Delta^{synth}}  &&v|| \textrm{Pooled balance} ||^2_2 \\
&&+ (1-v) \frac{1}{J} \sum_{j=2}^{J+1} ||\textrm{ State balance } ||^2_2 \\
&&+ \textrm{ penalty }
\end{eqnarray*}``Returns'' to this are highly convex: setting $v$ just a little below 1 yields a big improvement in state-level imbalance with very little cost in pooled imbalance

\end{frame}



\imageframe{./lecture_includes/augsynth_19.png}

\imageframe{./lecture_includes/augsynth_20.png}

\imageframe{./lecture_includes/augsynth_21.png}

\imageframe{./lecture_includes/augsynth_22.png}

\imageframe{./lecture_includes/augsynth_23.png}

\imageframe{./lecture_includes/augsynth_24.png}

\imageframe{./lecture_includes/augsynth_25.png}


\begin{frame}{Augment staggered adoption}

\begin{enumerate}
\item Estimate an outcome model
\item Estimate the partially pooled synth model
\item Use the outcome model to adjust synth for imbalance (bias correction) or alternatively just use synth on the residuals from the outcome model (double robust)
\end{enumerate}

\end{frame}

\begin{frame}{Special case: weighted event study}

\begin{itemize}
\item Estimate unit fixed effects via pre-treatment average: $\overline{Y}_{i,T_j}^{pre}$
\item Estimate synth using residuals (Doudchenko and Imbens 2017; Ferman and Pinto 2018)
\end{itemize}


\begin{eqnarray*}
\widehat{Y}^{aug}_{j,T_j+k}(0) = \overline{Y}_{j,T_j}^{pre} + \sum_{i=1}^N \widehat{w}_{ij} \bigg ( Y_{i,T_j+k} - \overline{Y}_{i,T_j}^{pre} \bigg )
\end{eqnarray*}where $Y(0)=Y^0$

\end{frame}

\begin{frame}{Special case: weighted event study}

Treatment effect estimate is \textbf{weighted diff-in-diff}:

\begin{eqnarray*}
\widehat{\delta}_{jk}^{aug} = \bigg ( Y_{j,T_j+k} - \overline{Y}_{j,T_j}^{pre} \bigg ) - \sum_{i=1}^N \widehat{w}_{ij} \bigg (Y_{i,T_j+k} - \overline{Y}_{i,T_j}^{pre} \bigg )
\end{eqnarray*}Uniform weights correspond to ``standard DiD''

\end{frame}


\begin{frame}[plain]

	\begin{figure}
	\includegraphics[scale=0.4]{./lecture_includes/augsynth_26.png}
	\end{figure}

\end{frame}
\begin{frame}[plain]

	\begin{figure}
	\includegraphics[scale=0.4]{./lecture_includes/augsynth_27.png}
	\end{figure}

\end{frame}
\begin{frame}[plain]

	\begin{figure}
	\includegraphics[scale=0.4]{./lecture_includes/augsynth_28.png}
	\end{figure}

\end{frame}
\begin{frame}[plain]

	\begin{figure}
	\includegraphics[scale=0.4]{./lecture_includes/augsynth_29.png}
	\end{figure}

\end{frame}
\begin{frame}[plain]

	\begin{figure}
	\includegraphics[scale=0.4]{./lecture_includes/augsynth_30.png}
	\end{figure}

\end{frame}
\begin{frame}[plain]

	\begin{figure}
	\includegraphics[scale=0.4]{./lecture_includes/augsynth_31.png}
	\end{figure}

\end{frame}

\begin{frame}{Conclusions}

\begin{itemize}
\item Augmented synth allows us to salvage the method, using an outcome model to remove bias from imperfect balance
\item Partially pooled synth allows extension to the staggered adoption setting
\item Combining the two methods gives us the best hope 
	\begin{itemize}
	\item A simple fixed effect outcome model leads to a weighted event study
	\item This generalizes recent recommendations for two-way fixed effects
	\end{itemize}
\end{itemize}

\end{frame}



\subsection{Event studies in finance}

\begin{frame}{Possibilities for detecting corruption}

\begin{itemize}

\item Event studies in finance have been used to detect abnormal patterns around ``events'' involving single firms
\item Baker and Gelbach (2020) proposes a type of synthetic control estimator that uses machine learning to estimate a counterfactual, as opposed to imposing strong parametric assumptions
\item Examples of its use have been applied to disruptions with the Elon Musk Twitter deal which while not corruption does involve estimating potential damages from stock price movements

\end{itemize}

\end{frame}

\begin{frame}{Largest Securities Class Action Settlements}

\begin{enumerate}

\item Enron: \$7.2b
\item WorldCom Inc: \$6.1b
\item Tyco International Ltd.: \$3.2b
\item Cendant Corporation: \$3.2b

\end{enumerate}

\end{frame}

\begin{frame}{Over time}

\begin{figure}
\includegraphics[scale=0.35]{./lecture_includes/baker_gelbach_1}
\end{figure}
\end{frame}

\begin{frame}{Event studies and securities litigation}

\begin{itemize}

\item Historically, the ``event study'' estimated ``abnormal'' returns under strong parametric assumptions (e.g., normality), but non-normal returns are normal

\begin{quote}
``The abnormal returns are the parameters that determine the damage estimates in securities suits, it is worthwhile to explore whether methods exist that can provide more accurate estimates of the abnormal return itself.''
\end{quote}

\item They argue that the event study is an out-of-sample prediction problem, which ML is used for, but it is also an extension of the synth modeling framework

\end{itemize}

\end{frame}

\begin{frame}{Basic idea}

\begin{figure}
\includegraphics[scale=0.35]{./lecture_includes/baker_gelbach_2}
\end{figure}
\end{frame}


\begin{frame}{Event studies as a prediction problem}

\begin{itemize}
\item Let the daily return for firm $i$ on date $t$ be $r_{i,t}$ and variables used for prediction is $X_{i,t}$ (e.g., market return, Fama-French and Carhart factors, a 1 for intercept, etc.)
\item Suppose an event reveals fraud.  It's effect on daily return is $r^1_{i,t} - r^0_{i,t}$ and we want to estimate $r^0_{i,t}$ with $\widehat{r}^0_{i,t}$
\item Construct a predicted residual as $\widehat{\varepsilon}_{i,t} = r_{i,t} - \widehat{r}^0_{i,t}$
\item Typically people would estimate this with OLS $$r_{i,t} = \alpha + \beta_1 X_{i,t} + \varepsilon_{i,t}$$
\end{itemize}

\end{frame}

\begin{frame}{OLS, ML, MSE, Bias, Variance}

\begin{itemize}
\item MSE of predicted abnormal return for $\widehat{\varepsilon}_{i,t} = r_{i,t} - \widehat{\beta}X_{i,t}$ is the sum of a squared bias term and a variance term
\item It's possible that the variance of one specification is lower enough than another to make up for a difference in bias
\item OLS also suffers because it overfits data when used for prediction -- it is best unbiased linear predictor but at the price of greater out-of-sample variance linear prediction
\item Since MSE is the basis for measuring prediction accuracy, ML estimators may outperform conventional OLS as we can explore increasing bias and reducing variance
\item ML methods accept bias in exchange for reduced variance out-of-sample accomplished through ``training''
\end{itemize}

\end{frame}

\begin{frame}{Paper's punchline}

\begin{quote}
``Using real stock return data, we demonstrate that a number of out-of-the-box statistical approaches that are relatively easy to interpret perform better than the standard, OLS-based event study specifications used in court proceedings.

\bigskip

We find that specifications using penalized regression generally perform well.  Specifications that adjust for daily market performance using data-driven peer indexes also generally perform well.

\bigskip

Finally, we obtain generally good performance from specifications that use a cross-validation technique that is robust to otherwise unmodeled time-series properties of the DGP. The best specifications provide noticeable improvements over event study approaches conventionally used in securities litigation. 

\end{quote}

\end{frame}

\begin{frame}{Peer index}

\begin{itemize}
\item They note that the best-performing specification makes use of both penalized regression and data-driven peer firm choice.
\item They call this the ``reasonable peer index'', and they show that ML methods can usefully serve as a basis for choosing \emph{which} peer firms to include in an event study (again, making this a synth-like method) which can mitigate the subjective researcher bias that synth is meant to overcome
\item Rather than subjectively picking which firms represent the counterfactual (over which there can be debate clearly, some disingenuous given the amount of money at stake), they propose letting the data say who the best peer is
\item But using \emph{any} peer index appears to mitigate this too
\end{itemize}

\end{frame}

\begin{frame}{Ranking all the ML methods}

\begin{figure}
\includegraphics[scale=0.35]{./lecture_includes/baker_gelbach_5}
\end{figure}
\end{frame}

\begin{frame}{Elon Musk example}

\begin{itemize}
\item In an unpublished analysis, Baker examined Elon Musk's attempt to buy Twitter on Twitter's stock price
\item Unlike his published paper, he's only going to use one form of ``penalized'' machine learning called ridge regression (which constrains what the coefficients can be in his model)
\item He will use peer index and the S\&P500 for prediction purposes
\end{itemize}

\end{frame}

\begin{frame}{Purpose of the exercise}

\begin{quote}
``The goal here is to get a rough estimate of what TWTR would be trading at had Elon never put the stock in play. Note, this does not mean that the prediction is equivalent to what TWTR would trade at were the deal to not go through (without any damage payments), as Elon has likely destroyed value in the process. This prediction could in fact be used as a baseline price in any tort-type damages claim that the company would want to bring against Elon after the process is over.''
\end{quote}

\end{frame}

\begin{frame}{Basic idea}

\begin{figure}
\includegraphics[scale=0.35]{./lecture_includes/baker_gelbach_3}
\end{figure}
\end{frame}

\begin{frame}{Basic idea}

\begin{figure}
\includegraphics[scale=0.35]{./lecture_includes/baker_gelbach_4}
\end{figure}
\end{frame}




\begin{frame}{Summarizing}

\begin{itemize}
\item Randomized treatments are great but not always available 
\item Causal inference methods can utilize naturally occurring variation, but still must make adequate adjustments to find suitable controls
\item Synthetic control and recent work on event studies can be possibilities
\item Thank you!

\end{itemize}

\end{frame}




\subsection{Matrix completion with nuclear norm}

\begin{frame}{Big idea}

\begin{quote}
``The main part of the article is about the statistical problem of imputing the missing values of $Y$.  Once these are imputed, we can estimate the causal effect of interest, $\delta$.''
\end{quote}

\bigskip

\begin{quote}
``To estimate average causal effect of the treatment on the treated units, we impute the missing potential control outcomes'' -- Athey, et al. (2021)
\end{quote}


\end{frame}

\begin{frame}{Overview}

\begin{itemize}
\item Athey, et al. (2021) unites two literatures -- unconfoundedness and synthetic control
\item Combines computer science with statistics to create the matrix completion with nuclear norm (MCNN) estimator
\item Nuclear norm regularization is used for the imputation
\end{itemize}

\end{frame}

\begin{frame}{What is matrix completion}

\begin{itemize}
\item Completing a matrix means guessing at the correct values that are missing 
\item Hence the ``completion'' is just another name for ``filling in'' the matrix 
\item In causal inference, if the matrix is a matrix of potential outcomes (e.g., $Y^0$), then missingness is caused by treatment assignment
\end{itemize}
\end{frame}

\begin{frame}[plain]


Here's a matrix of potential outcomes, $Y^0$, representing units at time $t$ that had not been treated. 
\begin{center}
\[ Y^0_{it}  =\begin{pmatrix}
    Y^0_{11} & Y^0_{12} & Y^0_{13} & \dots  & Y^0_{1t} \\
    Y^0_{21} & Y^0_{22} & Y^0_{23} & \dots  & Y^0_{2t} \\
    \vdots & \vdots & \vdots & \ddots & \vdots \\
    Y^0_{i1} & Y^0_{i2} & Y^0_{i3} & \dots  & Y^0_{it}
\end{pmatrix}\]
\end{center}

Now imagine a treatment assignment, SUTVA, that flips treatment from 0 to 1 in the last period $t$:

\begin{eqnarray*}
Y=DY^0 + (1-D)Y^1
\end{eqnarray*}

\end{frame}

\begin{frame}[plain]

Ask yourself: why are there question marks in the last column? 

\begin{center}
\[ Y^0_{it}  =\begin{pmatrix}
    Y^0_{11} & Y^0_{12} & Y^0_{13} & \dots  & ? \\
    Y^0_{21} & Y^0_{22} & Y^0_{23} & \dots  & ? \\
    \vdots & \vdots & \vdots & \ddots & \vdots \\
    Y^0_{i1} & Y^0_{i2} & Y^0_{i3} & \dots  & ?
\end{pmatrix}\]
\end{center}
Matrix completion seeks to do the following:



\end{frame}


\begin{frame}[plain]

Matrix completion with nuclear norm will impute the last column using regularized regression:

\begin{center}
\[ Y^0_{it}  =\begin{pmatrix}
    Y^0_{11} & Y^0_{12} & Y^0_{13} & \dots  & \widehat{Y^0_{1t}} \\
    Y^0_{21} & Y^0_{22} & Y^0_{23} & \dots  & \widehat{Y^0_{2t}} \\
    \vdots & \vdots & \vdots & \ddots & \vdots \\
    Y^0_{i1} & Y^0_{i2} & Y^0_{i3} & \dots  & \widehat{Y^0_{it}}
\end{pmatrix}\]
\end{center}

And once you have those, you can calculate individual level treatment effects that can be used to aggregate to the ATT

\end{frame}


\begin{frame}{History of matrix completion}

\begin{itemize}
\item Open competition by Netflix in 2006 -- winner would get \$1m if they could improve predictive model by ten points on RMSE
\item Invited a ton of competition -- from MIT teams to regular everyday joes working out of their home office
\item Everyone was given a database which was then tested by Netflix on a holdout dataset
\item Quick progress was made followed by very slow advances
\item Winner was announced in 2009
\end{itemize}

\end{frame}

\begin{frame}{Netflix prize}

\begin{itemize}
\item Gigantic sparsely populated matrix (100m users ranking 100k movies)
\item I like \underline{Silver Linings Playbook} and \underline{Lars and the Real Girl} and you like \underline{Silver Linings Playbook}
\item Probably you'll also like \underline {Lars and the Real Girl}
\item So we are using correlations in the columns to ``complete'' missing values
\item When you think about it, while it seems predictive (and it is), isn't it really a causal design?
\item ``If I watch \underline{Lars and the Real Girl}, will I like it?''
\end{itemize}

\end{frame}

\begin{frame}{Types of imputation}

\begin{itemize}
\item I didn't always think of causal inference in terms of imputation because often the method was just taking existing values and manipulating them, rather than filling in missing values
\item But the fundamental problem of causal inference states that causal inference is a missing data problem, so it makes sense you'd be imputing
\item I tend to think therefore in terms of implicit and explicit imputation methods
\item Borusyak, et al. (2021) and Athey, et al. (2021) both seem more like ``explicit'' imputation methods
\item Callaway and Sant'Anna (2020) on the other hand is an implicit method, as is did methods more generally
\end{itemize}

\end{frame}

\begin{frame}{Two literatures}

\begin{itemize}
\item Lots of moving parts in this interesting paper, so my goal here is purely explainer and mostly high level at that. 
\item I want you to be competent and conversant in it so we also have some R code
\item There's two literatures they want you to have in your mind:
	\begin{enumerate}
	\item Unconfoundedness -- $(Y^0,Y^1)\independent D|X$ -- sometimes explicitly imputes (nearest neighbor), sometimes more implicit (inverse probability weighting)
	\item Synthetic control -- literally calculating a counterfactual as a weighted average over all donor pool units
	\end{enumerate}
\item Their MCNN method will show that both are ``nested'' within the general framework they've developed making them actually special cases
\end{itemize}

\end{frame}


\begin{frame}{Differences}

\begin{itemize}
\item Conceptually different in the way they exploit patterns for causal inference
\item Unconfoundedness assumes that \textbf{patterns over time}are stable \emph{across units}
\item Synth assumes \textbf{patterns across units} are stable \emph{over time}
\item Regularization nests them both
\item Nuclear norm ensures a low rank matrix needed for sensible imputations

\end{itemize}

\end{frame}

\begin{frame}{The Gist}

\begin{itemize}
\item Factor models and interactive effects model the observed outcome as the sum of a linear function of covariates and a unobserved component that is a low rank matrix plus noise
\item Estimates are typically based on minimizing the sum of squared errors given the rank of the matrix of unobserved components with the rank itself estimated
\item Nuclear norm regularization will be used for imputing the potential outcomes, $Y^0$, for all treated units
\item Estimate plots and overall ATT using the estimated treatment effects
\end{itemize}

\end{frame}

\begin{frame}{Three contributions}

\begin{enumerate}
\item Formal results for non-random missingness when block structure allows for correlation over time.  Nuclear norm is important here
\item Shows unconfoundedness and synth are in fact matrix completion methods 
	\begin{itemize} 
	\item they all have the same objective function based on the Frobenius norm for the difference between the latent matrix and the observed matrix
	\item Each approach imposes different sets of restrictions on the factors in the matrix factorization
	\item MCNN by contrast doesn't impose any restrictions -- just regularization to characterize the estimator
	\end{itemize}
\item Applies the method to two datasets, but I'm going to skip it though for now
\end{enumerate}

\end{frame}


\begin{frame}{Block structure}

\begin{itemize}
\item Lots of jargon in this article -- unconfoundedness, vertical and horizontal regression, fat and thin matrices.  
\item Unfortunately, you need to learn it all so let me try and organize it
\item We define the matrix first in terms of its block structure which is describing where and when the missingness is occurring in the matrix
\end{itemize}

\end{frame}

\begin{frame}{Unconfoundedness}

\begin{itemize}
\item Much of the unconfoundedness literature estimates an ATE under unconfoundedness 
\item But it tends to focus only on a simple setup where the missingness is the last period
\item Think about LaLonde (1986) -- NSW treats the workers, and then you don't observe $Y^0$ for the treated group in the \emph{last period}
\item This is the ``single-treated-period block structure'' because only one \emph{period} is missing
\end{itemize}

\end{frame}

\begin{frame}{Single-treated-period block structure}

\begin{center}
\[ Y^0_{it}  =\begin{pmatrix}
    Y^0_{11} & Y^0_{12} & Y^0_{13} & \dots  & ? \\
    Y^0_{21} & Y^0_{22} & Y^0_{23} & \dots  & ? \\
    \vdots & \vdots & \vdots & \ddots & \vdots \\
    Y^0_{i1} & Y^0_{i2} & Y^0_{i3} & \dots  & ?
\end{pmatrix}\]
\end{center}

\end{frame}


\begin{frame}{Single-treated-unit block structure}

\begin{center}
\[ Y^0_{it}  =\begin{pmatrix}
    Y^0_{11} & Y^0_{12} & Y^0_{13} & \dots  & Y^0_{1t} \\
    Y^0_{21} & Y^0_{22} & Y^0_{23} & \dots  & Y^0_{2t}  \\
    \vdots & \vdots & \vdots & \ddots & \vdots \\
    Y^0_{i1} & Y^0_{i2} & ? & \dots  & ?
\end{pmatrix}\]
\end{center}

Notice, this is the synthetic control design because a single unit (unit $i$) is missing $Y^0$ for the 3rd and $t$th periods.

\end{frame}

\begin{frame}{Staggered adoption}

\begin{center}
\[ Y^0_{it}  =\begin{pmatrix}
    Y^0_{11} & ? & ? & \dots  & ? \\
    Y^0_{21} & Y^0_{22} & Y^0_{23} & \dots  & ? \\
    \vdots & \vdots & \vdots & \ddots & \vdots \\
    Y^0_{i1} & Y^0_{i2} & ? & \dots  & ?
\end{pmatrix}\]
\end{center}

So all of these so-called designs can be expressed in terms of missingness in the block structure, and our job therefore is to find an estimator that is general enough to manage all of them.  Their MCNN will be that.

\end{frame}


\begin{frame}{Thin and Fat matrices}

\begin{itemize}
\item We also have to consider the relative number of panel units $N$ and time periods $T$ because this also shapes which regression style will be used for imputation
\item Thin matrices are basically where $N>>T$, but fat matrices are ones where $T>>N$
\item Approximately square ones are where $T$ is approximately equal to $N$
\end{itemize}

\end{frame}


\begin{frame}{Vertical and horizontal regression}

\begin{itemize}
\item Two special combinations of missing data patterns and matrix shape need special attention because they are the focus of large but separate literatures
\item Unconfoundedness has that single-treated period block structure with a thin matrix ($N>>T$). 
\item You use a large number of units and impute missing potential outcomes in the last period using controls with similar lagged outcomes
\item This is the horizontal regression -- imagine just running OLS on the lags and taking predicted values
\item The horizontal regression holds under unconfoundedness
\end{itemize}

\end{frame}

\begin{frame}{Vertical regression}

Doudchenko and Imbens (2016) and Pinto and Furman (2019) show that Abadie, Diamond and Hainmueller (2011) can be interpreted as regressing the outcomes for the treated prior to treatment on the outcomes for controls in the same period

\end{frame}

\begin{frame}{Fixed effects and factor models}

\begin{itemize}
\item Both horizontal and vertical regressions exploit other patterns
\item An alternative to each of them though is to consider an approach that allows for the exploitation of both stable patterns over time and stable patterns across units
\item This is where their matrix completion with nearest neighbor model comes in -- it does that very thing
\end{itemize}

\end{frame}

\begin{frame}{Matrix completion with nuclear norm}

\begin{itemize}
\item Model the $N \times T$ matrix of complete outcomes data matrix $Y$ as: $$Y = L* + e$$where $E[e|L*]=0$
\item The error term can be thought of as measurement error if you need a frame to think about it
\item So you have this complete matrix, L*, and zero mean conditional independence holds
\end{itemize}

\end{frame}

\begin{frame}{Assumption 1}

Apart from the unconfoundedness assumption, we have this weird assumption!

\begin{block}{Assumption 1}
$e$ is independent of $L*$ and the elements of $e$ are $\sigma$-sub-Gaussian and independent of each other
\end{block}

Lots of matrix forms can be defined this way.  But let's not get lost in the weeds -- we are still just trying to estimate $L*$!  That's the main storyline, not the side quest, to use Red Dead Redemption words I understand

\end{frame}

\begin{frame}{All imputations are wrong but some are useful}	
	
\begin{itemize}
\item You can impute something a million different ways.  
\item $1+1+1+1 = 4$ is an imputation of the fifth unknown element and frankly just looking at it, seems wrong.
\item You could minimize the sum of squared differences but if the objective function doesn't depend on $L*$, the estimator would just spit back $Y$ and $\delta=0$. 
\item They add a penalty term $||\lambda||$ to the objective function, but even then, not all of them do well. 
\item Turns out, it actually matters whether you regularize the fixed effects or not (just like it matters whether you regularize the constant in LASSO apparently -- I decided to take their word for it)
\end{itemize}

\end{frame}

\begin{frame}{Estimator}

\begin{eqnarray*}
L* = \widehat{L} + \widehat{\Gamma}1_T^T + I_N\widehat{\Delta}^T
\end{eqnarray*}where the objective function is:

\begin{eqnarray*}
= arg\text{ }min_{L,\Gamma,\Delta} \bigg \{ \frac{1}{O} || P_0(Y-L-\Gamma 1_T^T - 1_N\Delta^T)||_F^2 + \Lambda||L|| \bigg \}
\end{eqnarray*}

\end{frame}

\begin{frame}{Fixed effects and regularization}

\begin{itemize}
\item The penalty will likely be the nuclear norm but notice that the fixed effects are outside the penalty term.  You could subsume them into $L$, they say, but they recommend you not doing this.
\item Fraction of observations is relatively high and so the fixed effects can actually be estimated separately (apparently that is one difference between MCNN and the rest of the MC literature)
\item The penalty will be chosen using cross-validation
\end{itemize}

\end{frame}

\begin{frame}{Other norms}

\begin{itemize}
\item One thing I thought was interesting was that the nuclear norm allowed for the construction of a low rank $L*$ matrix, but other norms actually would have weird properties
\item I remember once me asking Imbens (like I had even a clue what I was talking about), ``Why not use elastic net?  Why are you using the nuclear norm?'' He said elastic net would spit out all zeroes.  I remember thinking ``Why did I think I would understand what he told me?''
\item One advantage of NN is its fast and convex optimization programs will do it, whereas some others won't because of the large $N$ or $T$ issues
\item There's almost like a cross walk, too, between this and Borusyak, et al. (2021) but I don't quite see it except they both leverage imputation
\end{itemize}
\end{frame}

\begin{frame}{Conclusion}

\begin{itemize}
\item Ultimately, this is just another model though that can be used for differential timing but at the moment, no one knows how it performs in simulations alongside Borusyak, et al. (2021), Callaway and Sant'Anna (2020) or any of the others
\item So I can't really answer questions about when to use it and not to -- it comes down to these very narrow assumptions
\item You choose the estimator based on the problem you're studying and the assumptions -- you must justify it, no one else can, but you do so by appealing to assumptions
\end{itemize}

\end{frame}

\begin{frame}{Code}

R: \url{https://github.com/xuyiqing/gsynth}

\bigskip

Stata: ??

\end{frame}



\subsection{Synthetic difference-in-differences}

\begin{frame}{New developments}

\begin{itemize}
\item We have been interested in the effect of the treatment on the treated (ATT), and have been using panel data largely to do this -- first with the DiD, then the synthetic control model
\item Athey and Imbens have been very active in applying machine learning methods to causal inference (``most important innovation in causal inference of the last 15 years'' - Athey and Imbens)
\item Their work on synthetic control has been in this spirit
\item The synthetic DiD bears some similarities to their MCNN model, but focuses on estimating weights, not the $L^*$ matrix
\end{itemize}

\end{frame}

\begin{frame}{When to use this}

\begin{itemize}
\item Matrix completion with nuclear norm regularization allows for staggered adoption
\item Use this setting for 2x2 situations with more than one treatment group
\item So the ``block'' is a binary treatment with units treated in some late period
\item It will dominate the Abadie, Diamond and Hainmueller (2010) as they will show and addresses overfitting and other things through estimating oracle weights (which I'll explain towards the latter half)
\item Very technical paper -- this is not your mother's econometrics. We are shifting towards more structural estimation; perhaps a trend
\item Although parallel trends was in some ways already structural because it placed restrictions on the parallel trends (as opposed to relying on randomization)
\end{itemize}

\end{frame}

\begin{frame}{Model selection}

\begin{itemize}
\item Another thing is that ADH tends to rely on an ``eyeball test'' for the pre-trend fitting
\item ``Researcher degrees of freedom'' vs ``reducing subjective researcher bias''
\item This will allow for a more principled approach
\end{itemize}

\end{frame}

\begin{frame}{Imperfect fits}

\begin{itemize}
\item Recall that ADH needs to fit a pre-treatment convex hull to model the heterogeneity
\item Often, though, the fit is imperfect for various reason because weights are constrained to be non-negative and sum to one
\item But this can be problematic if the treatment group can't be approximated by a weighted average of other units since the weights are fractions 
\item So they're going to allow for a constant level shift to ``get there''
\end{itemize}

\end{frame}



\begin{frame}{Diff-in-diff, parallel trends and pre-trends}

\begin{itemize}
\item Recall the identifying assumption in DiD -- parallel trends
\item Untestable, but we often use pre-trends for an indirect test
\item But in the smoking example, parallel trends didn't hold for many states
\item Choice of control units matter -- the average trends for many control states are roughly parallel, but not all 
\end{itemize}

	
	\begin{figure}
	\includegraphics[scale=0.65]{./lecture_includes/abadie_3.pdf}
	\end{figure}


\end{frame}

\begin{frame}{ADH Synth}

\begin{itemize}
\item ADH sought a weighted average over the control units to recreate the pre-trend through a fitting exercise
\item Synthetic control becomes the weighted average of controls, and then the focus is just on estimating weights
\item All we ask is that the weighted average follow the same dynamic path as treatment group (a fit for each period)
\item Doudchenko and Imbens (2015) note that this is just a ``vertical regression'' which yields coefficients on the control units (as opposed to the lags in $T$)
\begin{eqnarray*}
Y^0_{1,t} = \sum_{j=2}^{J+1} \widehat{\omega_j} \times Y_{j+1,t}
\end{eqnarray*}
\item To the degree the fit is good pre-treatment, then the gaps post-treatment measure ATT at a point in time

\end{itemize}

\end{frame}

\begin{frame}{Weighting across controls}

Assume that the synthetic control at any period is $Y_{1,t} \approx \sum_{j=2}^{J+2} w_i \times Y_{j}$
\begin{itemize}
\item Synthetic control -- weights, $\widehat{w}$, control units to get weighted average controls
	\begin{enumerate}
	\item Use the pre-treatment data to find the optimal weights that when aggregated over control units predict treatment group outcomes (``fit'')
	\item Assumes that there's a stable relationship over time, though, because this is going to be our estimated counterfactual post-treatment
	\end{enumerate}
\item This is shown to be equivalent to a ``vertical regression'' where you regress units against the higher column units to get those weights
\item May require regularization in the regression (if there are more units than time periods)
\end{itemize}

\end{frame}

\begin{frame}{Weighting across time dimensions}

\begin{itemize}
\item Forecasting -- time weights, $\widehat{\lambda}$,  periods to get weighted average periods
	\begin{enumerate}
	\item Use the controls to learn an average of periods that forecast what we see post-treatment
	\item Imagine a regression, in other words, that yields coefficients on covariates, not on units, to predict future counterfactual
	\item Assumes that this relationship remains valid for the treated and we use the same average of periods to impute the $Y^0$ for our treatment group
	\end{enumerate}
\item This is equivalent to a ``horizontal regression'' where you regress outcomes against the leads (i.e., $Y_{it}$ against $Y_{i,t-1}$) -- this is what was meant by unconfoundedness from the MCNN lecture
\item Again may need regularization if there are more time periods than units
\end{itemize}

\end{frame}

\begin{frame}{Difference-in-differences model}

\begin{itemize}
\item They tend to equate DiD with a TWFE model $$Y(0)_{it} = \mu + \alpha_i + \gamma_t + \varepsilon_{it}$$ and solve for the unknown parameters
\item More generally, these are the factor models 
\end{itemize}

\end{frame}

\begin{frame}{Reconciling these things}

\begin{itemize}
\item Vertical regression (i.e., the ADH synth approach) assumes there is a stable relationship between units over time (hence why the weights accurately estimate counterfactuals post-treatment)
\item Horizontal regression (i.e., the unconfoundedness approach) is similar, but assumes a stable relationship between outcomes in the treatment period and pre-treatment periods that is the same for all units
\item DiD regression (TWFE): assumes an additive outcome model that captures differences between time and units
\end{itemize}

\bigskip

So the focus becomes about choosing between these methods

\end{frame}



\begin{frame}{Synthetic DiD}

Synthetic DID takes synth and forecasting to create a \emph{synthetic DiD} version
\begin{itemize}
\item Combine these two -- weighting controls using pre-treatment and weighting time using controls, then applying a type of DiD differencing -- to create the synthetic DiD model
\item There is a focus, just like ADH, on estimating appropriate weights
\item It's doubly robust -- only one has to remain valid
\item Constant effects will get differenced out and the synthetic control can be \emph{parallel} to treatment, as opposed to \emph{identical} in pre-treatment period
\end{itemize}

\end{frame}


\begin{frame}{Estimation of SDiD}

Synthetic DiD is DiD with a synthetic control and a pre-treatment period (on the baseline, just like CS). 
	\begin{enumerate}
	\item[1. ] Compute the regularization parameter to match the size of a typical one-period outcome change, $\Delta_{it} = Y_{i(t+1)} - Y_{it}$, for unexposed 
	\end{enumerate}

\end{frame}


\begin{frame}{Estimation of SDiD}

	\begin{enumerate}

	\item[2. ] Estimate unit weights $\widehat{w}$ defining a synthetic control unit (just like Abadie, Diamond and Hainmueller 2010) using the pre-treatment data $$\widehat{w}_1 + \widehat{w}^TY_{j,pre} \approx Y_{1,pre}$$ but they allow for an intercept term so that now the weights no longer need to make the unexposed pre-trends \emph{perfectly} match the treatment group (hence convex hull can fail to hold)
	\end{enumerate}

\end{frame}

\begin{frame}{Estimation of SDiD}

	\begin{enumerate}

	\item[3. ] Estimate the time weights $\widehat{\lambda}$ defining a synthetic pre-treatment period using control data$$\widehat{\lambda}_{j=1} + Y_{1,pre} \widehat{\lambda} \approx Y_{1,post}$$
	\end{enumerate}

\end{frame}


\begin{frame}{Estimation}

\begin{enumerate}
\item[4. ] Computer the SDID estimator via the weighted DID regression
\end{enumerate}

\begin{eqnarray*}
\textrm{arg min}_{\tau, \mu, \alpha, \beta} = \bigg \{ \sum_{i=1}^N \sum_{t=1}^T ( Y_{it} - \mu - \alpha_i -\beta_t - W_{it}\tau ) ^2 \widehat{w}_i^{sdid} \widehat{\lambda_t}^{sdid}  \bigg \}
\end{eqnarray*}


\end{frame}

\begin{frame}{Estimating the weights}

Our focus then becomes about estimating $\widehat{w}$ and $\widehat{\lambda}$

\begin{enumerate}
\item[5. ] Estimate the control weights, $\widehat{w}$, defining the control group unit via constrained least squares on the pre-treatment data. This requires weights to be non-negative and sum to one and allows for a level shift with regularization.  Synthetic control is a weighted average like in ADH
\end{enumerate}

\end{frame}


\begin{frame}{Estimating the weights}

\begin{enumerate}

\item[6. ] We then estimate the time weights. $\widehat{\lambda}$, defining the synthetic pre-treatment period via constrained least squares on the control data with analogous time constraints

\end{enumerate}

\end{frame}






\begin{frame}{More formalization}


Assumed data generating process -- outcome is ``low rank matrix'' (MCNN) plus noise

\bigskip


\begin{eqnarray*}
Y = L + \tau D  + E
\end{eqnarray*}

\bigskip

where $L$ is the systematic component and the conditional expectation of the error matrix $E$ given the assignment matrix $D$ and the systematic component of $L$ is zero.  

\bigskip

We won't estimate $L^*$ though, unlike MCNN

\end{frame}


\begin{frame}{Data generating process -- noise and signal}

\begin{eqnarray*}
Y = L + \tau D  + E
\end{eqnarray*}

\bigskip

The treatment cannot depend on the error term, but may depend on the systematic elements of $L$ (i.e., $D$ is not randomized). Think of $L$ as the signal, $\tau$ a matrix of treatment effects and $E$ the noise with no autocorrelation over time or between units. The only thing random is $E$, our noise matrix.

\end{frame}

\begin{frame}{Estimating the weights -- high level}

\begin{itemize}
\item Modify synthetic control weights -- use penalized least squares to get a weighted average of control units with pre-trends ``parallel'' to the treated unit average
\item But they'll allow for a constant, unlike ADH synth
\item And then they'll do the same thing for the time weights, but this time they won't regularize because they want to weight more intensively the periods ``just before'' -- ridge, they note, would ``spread out the weights'' over multiple time periods and they don't want that
\item I'll get more into this with the oracle weights, but for now I'll just note it conceptually
\end{itemize}

\end{frame}

\begin{frame}{Picture}

	\begin{figure}
	\includegraphics[scale=0.3]{./lecture_includes/hirshberg_sdid_1.png}
	\end{figure}

(credit: David Hirshberg January 2020 slides because I can't make this picture to save my life)

\end{frame}

\begin{frame}{Regression}

\begin{itemize}
\item SC is weighted linear regression with no unit FEs:$$\tau^{sc} = \textrm{argmin}_{\tau, \lambda}  \sum_{i,t} (Y_{it} - \lambda_t - \tau D_{it})^2 \times w_i^{sc} $$
\item DiD is unweighted regression with unit FEs and time FEs:$$\textrm{argmin}_{\tau, \lambda, \alpha}  \sum_{i,t} (Y_{it} - \lambda_t - \alpha_i -  \tau D_{it})^2 $$
\item SDiD is weighted regression with unit FEs and time FEs:$$ \textrm{argmin}_{\tau, \lambda, \alpha} \sum_{i,t} (Y_{it} - \lambda_t - \alpha_i - \tau D_{it})^2 \times w_i \times \lambda_t$$
\end{itemize}

\end{frame}

\begin{frame}{Formal results overview}

\begin{itemize}
\item Formal results will show SDiD is ``doubly robust'' (recall Sant'Anna and Zhao 2020)
\item Factor model  on the outcome can be a latent factor model but true model is that signal model and it'll still be consistent
\item Asymptotic normality of $\widehat{\tau}^{SDiD}$
\item With oracle weights, SDiD will have ``good weights''
\item You can do inference through resampling like jackknife, bootstrap and randomization inference
\end{itemize}

\end{frame}


\begin{frame}[plain]

	\begin{figure}
	\includegraphics[scale=0.65]{./lecture_includes/sdid_2.png}
	\end{figure}
	
Estimated decrease: -27.3 (17.7)

\end{frame}


\begin{frame}[plain]

	\begin{figure}
	\includegraphics[scale=0.65]{./lecture_includes/sdid_1.png}
	\end{figure}

Estimated decrease: -19.6 (9.9); bad fit just prior bc weights are fitting everywhere

\end{frame}

\begin{frame}[plain]

	\begin{figure}
	\includegraphics[scale=0.65]{./lecture_includes/sdid_3.png}
	\end{figure}
	
Estimated decrease: -15.4 (8.4). Jagged line left of 1988 is the weighting of those years 	

\end{frame}






\begin{frame}{Practical problems}


\begin{itemize}
\item Underfitting. What if I can't get a parallel synthetic control?  I know because it's visible. This is an underfitting problem.  We need more controls, better controls, or another method. 
\item Omitted variable bias.  Something else happens exactly when the treatment occurs.  Sorry -- there isn't a solution, because you're not identified. 
\item Overfitting. We get a synthetic control, but it's because the plot over fit the data. This means that you've not approximated the counterfactual post-treatment.  No different than in RDD when you're unable to identify the counterfactual due to functional form problems. 
\end{itemize}

\end{frame}

\begin{frame}{How to rule out overfitting: oracle weights}

\begin{itemize}
\item Their estimator is equivalent to an ``oracle estimator'' which cannot overfit
\item Oracle uses unit and time weights that don't depend on the noise
\item Weights minimize MSE; oracle weights minimize \textbf{expected} SE
\end{itemize}

\end{frame}

\begin{frame}{Decomposing the bias of SDID}

\begin{eqnarray*}
\widehat{\tau}^{sdid} - \tau  &=& \varepsilon(\widetilde{w}, \widetilde{\lambda}) + B(\widetilde{w}, \widetilde{\lambda}) + \widehat{\tau}(\widehat{w},\widehat{\lambda}) - \widehat{\tau}(\widetilde{w},\widetilde{\lambda}) \\
&=& \textrm{oracle noise} +  \\
&& \textrm{oracle confounding bias} + \\
&&\textrm{deviation from oracle}
\end{eqnarray*}

\bigskip

So they characterize these terms

\end{frame}


\begin{frame}{Oracle noise}

First term: the oracle noise

\bigskip

\begin{eqnarray*}
\varepsilon(\widetilde{w}, \widetilde{\lambda})
\end{eqnarray*}

\bigskip

Tends to be small when the weights are small and there are a sufficient number of exposed units and time periods. 

\end{frame}

\begin{frame}{Oracle confounding bias (rows / units)}

\begin{eqnarray*}
B(\widetilde{w}, \widetilde{\lambda}) 
\end{eqnarray*}

\bigskip

Will be small when the pre-exposure oracle row (units) regression fits well and generalizes to the exposed rows :
 
 $$\widetilde{w_1} + \widetilde{w_{j}}^TL_{j,pre} \approx \widetilde{w}_1^TL_{1,pre}$$and
 
 $$\widetilde{w_1} + \widetilde{w_{j}}^TL_{j,post} \approx \widetilde{w}_1^TL_{1,post}$$
 
 \end{frame}
 
 \begin{frame}{Oracle confounding bias (columns / time)}

\begin{eqnarray*}
B(\widetilde{w}, \widetilde{\lambda}) 
\end{eqnarray*}

\bigskip

Will be small when the pre-exposure oracle column (time) regression fits well and generalizes to the exposed columns :
 
 $$\widetilde{\lambda_1} + \widetilde{\lambda_{j}}^TL_{j,pre} \approx \widetilde{\lambda}_1^TL_{1,pre}$$, and
 
 $$\widetilde{\lambda_1} + \widetilde{\lambda_{j}}^TL_{j,post} \approx \widetilde{\lambda}_1^TL_{1,post}$$
 
 \end{frame}


\begin{frame}{Oracle confounding bias -- neither do well}

What if neither model generalizes well on its own, then there is a doubly robust property

\bigskip

It is sufficient for one model to predict the generalization error of the other

\bigskip

``The upshot is even if one of the sets of weights fails to remove the bias from the presence of $L$, the combination of oracle unit and time weights can compensate for such failures''

\end{frame}

\begin{frame}{Deviation from Oracle}

Core theoretical claim (All formalized in their asymptotic analysis): SDID estimator will be close to the oracle when

\begin{itemize}
\item  The oracle time and unit weights look promising on their respective training sets

\begin{eqnarray*}
\widetilde{w_1} + \widetilde{w_j}^TL_{j,pre} \approx \widetilde{w}_1^T L_{1,pre} \\
\widetilde{\lambda_1} + \widetilde{\lambda_j}^TL_{j,pre} \approx \widetilde{\lambda_1}^T L_{1,pre} 
\end{eqnarray*}

\item and regularization is not too large for either weight
\end{itemize}

\end{frame}

\begin{frame}{Properties}

Under some assumptions, they provide then that SID:

\begin{enumerate}
\item SDiD is approximately unbiased and normal
\item SDiD has a variance that is optimal and estimable via clustered bootstrap
\end{enumerate}


\end{frame}




\begin{frame}{Placebo Simulation}

\begin{itemize}
\item Big picture still -- they do a simulation to evaluate bias, RMSE of estimates compared to the observed outcome, but they don't want to use randomization because that may not catch the distinct time trend
\item They want the simulation to be ``realistic'' not ``ideal'' (i.e., design based identification using randomized treatment dates)
\item  Bertrand, et al. (2004)  randomly assigned a set of states in the CPS to a placebo treatment and the rest the control and examine how well different approaches to inference for DiD covered the true effect of zero
\item Only methods that were robust to serial correlation of repeated observations for a given unit (e.g., clustering by level of treatment) attained valid coverage
\end{itemize}

\end{frame}



\begin{frame}{Treatment assignment process}

\begin{itemize}
\item Policy: abortion laws, gun laws, minimum wages with outcome hours and unemployment rate
\item Logistic regression to predict presence of regulation on four state factors from simulation outcome model $M$
\item Goodness of fit shows that treatment assignment responds strongly to unobserved latent factors
\item Assign treatment to states with probabilities from the logistic model
\end{itemize}

\end{frame}



\begin{frame}{Some details of this placebo simulation}

\begin{itemize}
\item They calculate average earnings over 40 years and 50 states by subtracting the overall mean and dividing by the standard deviation to get a matrix $Y$ with $||Y||^2_2 = 1$
\item They fit a rank 4 factor model $M$ 
\item They then extract TWFE from there based on unit and time fixed effects $F$
\item Extract low rank matrix as  $L=M-F$
\item Calculate residuals $E=Y-M$ on an AR(2) model
\item Compared SDID, DiD, synthetic control and matrix completion under different baseline scenarios and SDID tends to better

\end{itemize}

\end{frame}


\imageframe{./lecture_includes/sdid_5.png}


\imageframe{./lecture_includes/sdid_4.png}


\begin{frame}{Inference}


This can be used to motivate practical methods for large-sample for inference.  You can use conventional confidence intervals to conduct asymptotically valid inference, and they discuss three ways: jackknife, bootstrap, and placebo variance estimation.

\end{frame}





\imageframe{./lecture_includes/sdid_7.png}

\begin{frame}{Some practical considerations}

More treated units is worse -- when we add treated units, the oracle standard deviation decreases faster leaving too little room for other sources of error to disappear in the noise

\end{frame}

\begin{frame}{More practical considerations}

Circumstances are ideal if the signal matrix $L$ admits a good oracle synthetic control and synthetic pre-treatment period and it's too complex

	\begin{itemize}
	\item What is good?  Oracle control weights distribute mass over enough control units
	\item Oracle time weights should distribute the rest of its mass over enough time periods
	\end{itemize}


\end{frame}

\begin{frame}{More practical considerations}

Interestingly, this is an overlap assumption (like common support in matching and CS DiD):
	\begin{itemize}
	\item Many control units are like the treated ones
	\item Many pre-treatment periods are comparable to post-treatment ones
	\end{itemize}

\end{frame}


\begin{frame}{More practical considerations}

What is ``not too complex'' signal matrix $L$? It's one that looks different from the matrix of noise

\begin{itemize}
	\item More about the rank of the matrix -- it must be moderate rank
	\item Moderate means smaller than the square root of the number of control units
	\item A state's behavior isn't idiosyncratic, but characterized by a blend of industries, etc. of relatively few trends
\end{itemize}

\end{frame}


\begin{frame}{More practical considerations}

\begin{itemize}
\item Including more controls won't hurt you bc the set of weights is small and the error is insensitive to dimension
\item Less than ideal circumstances can be problematic. The error gets worse:
	\begin{itemize}
	\item Signal is too complex
	\item Fit and dispersion of the oracle weights is poor
	\end{itemize}
\end{itemize}

\end{frame}



\begin{frame}{Some comments}

\begin{itemize}
\item Conceptually, this is ADH synth combined with a simple 2x2 DiD where the weights are based on estimated time and control group weights
\item Oracle weights will make improvements that don't suffer from some of the practical problems, like overfitting, that we said
\item Synth DiD dominates synthetic control
\item Still remains to be seen how we are going to go about choosing between these, but some things we may need to put down (ADH)
\end{itemize}

\end{frame}

\begin{frame}{R code: synthdid}

Let's look at the code together

\bigskip

Code: \url{https://github.com/synth-inference/synthdid} 

\bigskip

Vignettes: \url{https://synth-inference.github.io/synthdid/articles/more-plotting.html}

\end{frame}

\section{Concluding remarks}



\begin{frame}{Conclusions}

\begin{itemize}
\item Synth is useful for very difficult problems in which parallel trends is implausible
\item With large $T$ and perfect balance, you can use synth to get approximately unbiased treatment effect estimates under reasonable DGPs (we saw in the original ADH)
\item But perfect balance is a unicorn and doesn't happen in most settings
\item What do we do when it doesn't?  Give up?  Salvage the estimates somehow? How?
\end{itemize}

\end{frame}








\end{document}
