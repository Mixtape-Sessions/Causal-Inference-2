\PassOptionsToPackage{table}{xcolor}
\documentclass{beamer}

\input{preamble.tex}
\usepackage{breqn} % Breaks lines

\usepackage{amsmath}
\usepackage{mathtools}

\usepackage{pdfpages} % \includepdf

\usepackage{listings} % R code
\usepackage{verbatim} % verbatim

% Video stuff
\usepackage{media9}

% packages for bibs and cites
\usepackage{natbib}
\usepackage{har2nat}
\newcommand{\possessivecite}[1]{\citeauthor{#1}'s \citeyearpar{#1}}
\usepackage{breakcites}
\usepackage{alltt}

% Setup math operators
\DeclareMathOperator{\E}{E} \DeclareMathOperator{\tr}{tr} \DeclareMathOperator{\se}{se} \DeclareMathOperator{\I}{I} \DeclareMathOperator{\sign}{sign} \DeclareMathOperator{\supp}{supp} \DeclareMathOperator{\plim}{plim}
\DeclareMathOperator*{\dlim}{\mathnormal{d}\mkern2mu-lim}
\newcommand\independent{\protect\mathpalette{\protect\independenT}{\perp}}
   \def\independenT#1#2{\mathrel{\rlap{$#1#2$}\mkern2mu{#1#2}}}
\newcommand*\colvec[1]{\begin{pmatrix}#1\end{pmatrix}}

\newcommand{\myurlshort}[2]{\href{#1}{\textcolor{gray}{\textsf{#2}}}}


\begin{document}

\imageframe{./lecture_includes/mixtape_did_cover.png}


% ---- Content ----

\section{Introduction}

\subsection{Managing expectations}


\begin{frame}{Introduction}

\begin{itemize}
\item Welcome to 2nd Annual CodeChella 2025 (hosted by Mixtape Sessions and CUNEF)!
\item Speakers are Scott Cunningham (Baylor) Kyle Butts (University of Arkansas), Dan Rees (UC3M) and Mark Anderson (Montana State University)
\item Four days of difference-in-differences and other panel estimators!
\end{itemize}

\end{frame}


\begin{frame}{What my pedagogy is like}

\begin{itemize}
\item Workshop is intended to take someone from knowing nothing about difference-in-differences to a broad level of competency on advanced topics
\item My stuff will mix basic things with subtle interpretation, my own subjective beliefs about "good science" (but not necessarily good papers), and developments
\item Lecture, discussion, exercises, application, coding along the way
\item Ask questions at any point; I'll do my best to answer them but will sometimes defer it to the break (or I may just ask Kyle, Mark and/or Dan what they think)
\end{itemize}

\end{frame}


\begin{frame}{Class goals}
Pedagogical goal is to explain diff-in-diff and other panel estimators, but as a secondary objective to help foster in you:

  \begin{enumerate}
    \item \textbf{Confidence}: You will feel like you have a good enough understanding of diff-in-diff, both in its basics and some more contemporary issues, so that it seems like an intuitive and useful tool you could imagine using
    \item \textbf{Comprehension}: You will have learned a lot both conceptually and in the specifics, particularly with regards to issues around identification and estimation in the diff-in-diff 
    \item \textbf{Competency}: You will have more knowledge of programming syntax in Stata and R so that later you can apply this in your own work
  \end{enumerate}

\end{frame}



\begin{frame}{Day 1 outline}

Basics of difference-in-differences (Beginners but also my own personal take on this)
	\begin{itemize}
	\item The Core -- 
		\begin{itemize}
		\item what is difference-in-differences, how do you calculate it, what does it mean, what are its assumptions
		\item Why do we weight?
		\item Four Means or One Regression?
		\end{itemize}
	\item Testing for parallel trends -- event studies, falsifications
	\item Violations of parallel trends and solutions:
		\begin{itemize}
		\item Compositional change and corrections
		\item Triple differences
		\item Handling covariates with regression adjustment, IPW, double robust vs traditional OLS specifications
		\end{itemize}
	\end{itemize}

\end{frame}


\begin{frame}{Day 2 outline}

	\begin{itemize}
	\item Continuous treatments in diff-in-diff framework (Kyle Butts)
	\item Traditional fixed effects regressions and Bacon decomposition
	\end{itemize}

\end{frame}


\begin{frame}{Day 3 outline}


\begin{itemize}
\item Callaway and Sant'Anna (2020) for differential timing
\item Decomposition event study leads and lags under staggered adoption (Sun and Abraham 2021)
\item Mark and Dan "hidden curriculum" applied research workshop
\end{itemize}

\end{frame}

\begin{frame}{Day 4 outline}

\begin{itemize}

\item Imputation methods and synthetic control (Kyle Butts)
\item Mark and Dan "hidden curriculum" applied research workshop and Q\&A

\end{itemize}

\end{frame}



\subsection{Diff-in-Diff Populairity}

\begin{frame}{Causal Claims in Economics}

	\begin{figure}
	\includegraphics[scale=0.4]{./lecture_includes/causal_claims}
	\end{figure}

\end{frame}


\begin{frame}{Diff-in-Diff Over Time}

	\begin{figure}
	\caption{Currie, et al. (2020)}
	\includegraphics[scale=0.25]{./lecture_includes/currie_did.png}
	\end{figure}

\end{frame}

\begin{frame}{Diff-in-Diff by Field}

	\begin{figure}
	\caption{Goldsmith-Pinkham (2024)}
	\includegraphics[scale=0.75]{./lecture_includes/paul_did}
	\end{figure}

\end{frame}

\begin{frame}{Diff-in-Diff by Field}

	\begin{figure}
	\caption{Garg and Fetzer (2025)}
	\includegraphics[scale=0.25]{./lecture_includes/did_growth2}
	\end{figure}

\end{frame}


\begin{frame}{Diff-in-Diff Over Time}

	\begin{figure}
	\caption{Garg and Fetzer (2025)}
	\includegraphics[scale=0.25]{./lecture_includes/did_growth1}
	\end{figure}

\end{frame}



\begin{frame}{What is difference-in-differences}

\begin{itemize}
\item DiD is when a group of units are assigned some treatment and then compared to a group of units that weren't before and after
\item One of the most widely used quasi-experimental methods in economics and increasingly in industry
\item Predates the randomized experiment by 80 years, but uses basic experimental ideas about treatment and control groups (just not randomized)
\item Uses panel or repeated cross section datasets, binary treatments usually, and often covariates
\end{itemize}
\end{frame}


\subsection{Origins of diff-in-diff in public health}


\begin{frame}{Ignaz Semmelweis and washing hands}

\begin{itemize}
\item Early 1820s, Vienna passed legislation requiring that if a pregnant women giving birth went to a public hospital (free care), then depending on the day of week and time of day, she would be routed to either the midwife wing or the physician wing (most likely resulting in random assignment)
\item But by the 1840s, Ignaz Semmelweis noticed that pregnant women died after delivery in the (male) wing at a rate of 13-18\%, but only 3\% in the (female) midwife wing -- cause was puerperal or “childbed” fever
\item Somehow this was also we known -- women would give birth in the street rather than go to the physician if they were unlucky enough to have their water break on the wrong day and time
\end{itemize}

\end{frame}

\begin{frame}{Ignaz Semmelweis and washing hands}

\begin{itemize}
\item Ignaz Semmelweis conjectures after a lot of observation that the cause is the teaching faculty teaching anatomy using cadavers and then delivering babies \emph{without washing hands}
\item New training happens to one but not the other and Semmelweis thinks the mortality is caused by working with cadavers
\item Convinced the hospital to have physicians wash their hands in chlorine but not the midwives, creating a type of difference-in-differences design 
\end{itemize}

\end{frame}

\begin{frame}{Semmelweis diff-in-diff evidence}

	\begin{figure}
	\includegraphics[scale=0.4]{./lecture_includes/semmelweis_graphic.png}
	\end{figure}


\end{frame}

\begin{frame}{Evidence Rejected}

\begin{itemize}

\item Diff-in-diff evidence was rejected by Semmelweis' superiors claiming it was the hospital's new ventilation system
\item Dominant theory of disease spread was caused by "odors" or miasma or "humors"
\item Semmelweis began showing signs of irritability, perhaps onset of dementia, became publicly abusive, was committed to a mental hospital and within two weeks died from wounds he received while in residence
\item Despite the strength of evidence, difference-in-differences was rejected -- a theme we will see continue

\end{itemize}

\end{frame}






\begin{frame}{John Snow and cholera}

\begin{itemize}
\item Three major waves of cholera in the early to mid 1800s in London, largely thought to be spread by miasma (``dirty air'')
\item John Snow believed cholera was spread through the Thames water supply through an invisible creature that entered the body through food and drink, caused the body to expel water, placing the creature back in the Thames and causing epidemic waves
\item London passes ordinance requiring water utility companies to move inlet pipe further up the Thames, above the city center, but not everyone complies
\item Natural experiment: Lambeth water company moves its pipe between 1849 and 1854; Southwark and Vauxhall water company delayed
\end{itemize}

\end{frame}


\begin{frame}

	\begin{figure}
	\caption{Two water utility companies in London 1854}
	\includegraphics[scale=0.225]{./lecture_includes/lambeth.png}
	\end{figure}


\end{frame}



\begin{frame}{Difference-in-differences}

\begin{table}\centering
\scriptsize
		\caption{Lambeth and Southwark and Vauxhall, 1849 and 1854}
		\begin{center}
		\begin{tabular}{lll|lc}
		\toprule
		\multicolumn{1}{l}{\textbf{Companies}}&
		\multicolumn{1}{c}{\textbf{Time}}&
		\multicolumn{1}{c}{\textbf{Outcome}}&
		\multicolumn{1}{c}{$D_1$}&
		\multicolumn{1}{c}{$D_2$}\\
		\midrule
		Lambeth & Before & $Y=L$ \\
		& After & $Y=L + L_t + D$ & $\textcolor{red}{L_t}+D$\\
		\midrule
		& & & & $D + (\textcolor{red}{L_t}- SV_t)$ \\
		\midrule
		Southwark and Vauxhall & Before & $Y=SV$ \\
		& After & $Y=SV + SV_t$ & $SV_t$\\
		\bottomrule
		\end{tabular}
		\end{center}
	\end{table}

\begin{eqnarray*}
\widehat{\delta}_{did} = D + (\textcolor{red}{L_t}- SV_t)
\end{eqnarray*}This method yields an unbiased estimate of D if $\textcolor{red}{L_t} = SV_t$, but note that $\textcolor{red}{L_t}$ is a counterfactual trend and therefore not known

\end{frame}

\section{Diff-in-Diff Fundamentals}

\subsection{Average Treatment Effect on the Treated}

\begin{frame}{Introducing Potential Outcomes to DiD}

\begin{itemize}
\item Research question versus causal question -- not the same thing
	\begin{itemize}
	\item Research question would be you are wanting to know effect of job training programs on earnings
	\item Causal question is expressed using potential outcomes
	\end{itemize}
\item Causal questions are usually averages of individual treatment effects for a specific population of units
\end{itemize}

\end{frame}




\begin{frame}{Identification vs Estimation}

\begin{itemize}
\item We must start by making a distinction between the parameter we are attempting to identify and the manner in which we will estimate it
\item Identification requires first stating explicitly our goal expressed using potential outcomes
\item But often people skip this step and go directly to the  $2 \times 2$ calculation

\end{itemize}

\end{frame}






\begin{frame}{Potential outcomes notation}
	
	\begin{itemize}
	\item Let the treatment be a binary variable: $$D_{i,t} =\begin{cases} 1 \text{ if in job training program $t$} \\ 0 \text{ if not in job training program at time $t$} \end{cases}$$where $i$ indexes an individual observation, such as a person

	\end{itemize}
\end{frame}

\begin{frame}{Potential outcomes notation}
	
	\begin{itemize}

	\item Potential outcomes: $$Y_{i,t}^j =\begin{cases} 1 \text{: wages at time $t$ if trained} \\ 0 \text{: wages at time $t$ if not trained} \end{cases}$$where $j$ indexes a state of the world where the treatment happened or did not happen

	\end{itemize}
\end{frame}



\begin{frame}{Treatment effect definitions}


	\begin{block}{Individual treatment effect}
	    The individual treatment effect,  $\delta_i$, equals $Y_i^1-Y_i^0$
	\end{block}

Missing data problem:  No data on the counterfactual 
	
\end{frame}


\begin{frame}{Average Treatment Effects for the Treated Subpopulation}	
	\begin{block}{Average Treatment Effect on the Treated (ATT)}
	The average treatment effect on the treatment group is equal to the average treatment effect conditional on being a treatment group member:
		\begin{eqnarray*}
		E[\delta|D=1]&=&E[Y^1-Y^0|D=1] \nonumber \\
		&=&E[Y^1|D=1]-\textcolor{red}{E[Y^0|D=1]}
		\end{eqnarray*}
	\end{block}
	
	\bigskip

It's the average causal effect but only for the people exposed to some intervention; notice we can't calculate it, also, because we are missing the red term

	
\end{frame}

\begin{frame}{ATT vs ATE}

\begin{itemize}
\item Potential outcomes is weirdly intuitive and complicated at the same time
\item Simple exercises tend to help bring some of that confusion to the surface
\item Let's look at the "Potential Outcomes" tab at our Diff-in-Diff Worksheet 
\item \url{https://docs.google.com/spreadsheets/d/1onabpc14JdrGo6NFv0zCWo-nuWDLLV2L1qNogDT9SBw/edit?usp=sharing}
\item Follow instructions and fill all of it out
\end{itemize}

\end{frame}




\subsection{Core Diff-in-Diff Assumptions}

\begin{frame}{Deriving identification assumptions}

\begin{itemize}

\item Diff-in-diff is two things 
	\begin{enumerate}
	\item It's \textcolor{blue}{always} a calculation (i.e., four averages)
	\item It \textcolor{red}{sometimes} has a causal interpretation
	\end{enumerate}
\item I'm going to walk us through two main assumptions using potential outcomes and algebra
\item Then we will look at what happens when our control group has been treated (not an assumption, more like a promise)
\end{itemize}

\end{frame}


\begin{frame}{DiD equation is the 2x2}

\begin{itemize}
\item The building block of diff-in-diff calculations is ``four averages and three subtractions''
\item Needs two groups, two time periods, which is four averages
\item Often called the $2 \times 2$ for that reason (Goodman-Bacon 2021)
\end{itemize}

\begin{eqnarray*}
\widehat{\delta} = \bigg ( E[Y_k|Post] - E[Y_k|Pre] \bigg ) - \bigg ( E[Y_U | Post ] - E[ Y_U | Pre] \bigg) \\
\end{eqnarray*}$k$ are the people in the job training program, $U$ are the untreated people not in the program, $Post$ is after the trainees took the class, $Pre$ is the period just before they took the class, and $E[y]$ is mean earnings. 

\end{frame}



\begin{frame}{Replace with potential outcomes and add a zero}

\begin{eqnarray*}
\widehat{\delta} &=& \bigg ( \underbrace{E[Y^1_k|Post] - E[Y^0_k|Pre] \bigg ) - \bigg ( E[Y^0_U | Post ] - E[ Y^0_U | Pre]}_{\mathclap{\text{Switching equation}}} \bigg)  \\
&&+ \underbrace{\textcolor{red}{E[Y_k^0 |Post] - E[Y^0_k | Post]}}_{\mathclap{\text{Adding zero}}} 
\end{eqnarray*}

\end{frame}

\begin{frame}{Parallel trends bias}

\begin{eqnarray*}
\widehat{\delta} &=& \underbrace{E[Y^1_k | Post] - \textcolor{red}{E[Y^0_k | Post]}}_{\mathclap{\text{ATT}}} \\
&& + \bigg [  \underbrace{\textcolor{red}{E[Y^0_k | Post]} - E[Y^0_k | Pre] \bigg ] - \bigg [ E[Y^0_U | Post] - E[Y_U^0 | Pre] }_{\mathclap{\text{Non-parallel trends bias in 2x2 case}}} \bigg ]
\end{eqnarray*}


\end{frame}

\begin{frame}{Identification through parallel trends}
	

	\begin{block}{Parallel trends}
	Assume two groups, treated and comparison group, then we define parallel trends as:	 $$\textcolor{red}{E(}\textcolor{red}{\Delta Y^0_k)} = E(\Delta Y^0_U)$$
	\end{block}

\textbf{In words}: ``The \textcolor{red}{evolution of earnings for our trainees \emph{had they not trained}} is the same as the evolution of mean earnings for non-trainees''.  

\bigskip

It's in \textcolor{red}{red} because parallel trends is untestable and critically important to estimation of the ATT using any method, OLS or ``four averages and three subtractions''

\end{frame}

\begin{frame}{Don't Need More Than Two Periods}

\begin{itemize}

\item Notice that diff-in-diff identifies the ATT \emph{with only} two time periods
\item Diff-in-diff does not "need" a long pre-treatment time series (unlike synthetic control which does) -- two periods and two groups is enough
\item But there is another assumption aside from parallel trends we want to learn

\end{itemize}

\end{frame}

\begin{frame}{No Anticipation}

\begin{itemize}
\item First interpretation of No Anticipation (NA) is this:
	\begin{eqnarray*}
	Y_{i,t-1}=Y^0_{i,t-1} 
	\end{eqnarray*}
\item Future treatments do not spill over to the past -- that is, a unit is not treated until the day it is treated
\item A better word might be "no pre-treatment", rather than "no anticipation" because "anticipation" is a behavior and that confuses things
\item What if a government announces a minimum wage increase that will take place in one year?

\end{itemize}

\end{frame}

\begin{frame}{No Anticipation}

\begin{itemize}
\item Second interpretation of NA means this:
	\begin{eqnarray*}
	Y^1_{i,t-1} - Y^0_{i,t-1}=0
	\end{eqnarray*}which means "all pre-treatment treatment effects are zero.  
\item This means even if future minimum wage increases were known, their pre-treatment treatment effects were zero
\item So no -- knowing that something is going to happen does not automatically violate NA (but it absolutely could)
\item Let's formalize what happens when NA is violated

\end{itemize}

\end{frame}




\begin{frame}{No Anticipation Violation}


\begin{eqnarray*}
\widehat{\delta} &=& \bigg ( E[Y_k|Post] - E[Y_k|Pre] \bigg ) - \bigg ( E[Y_U | Post ] - E[ Y_U | Pre] \bigg) \\
\end{eqnarray*}What if the $k$ group had been treated at the baseline ``pre'' period in our 2x2?  

\bigskip

Add in \textbf{two zeroes} instead of one, substitute and rearrange.

\begin{eqnarray*}
&+& \textcolor{red}{E[Y^0_k|Post] - E[Y^0_k|Post]} \\
&+& \textcolor{red}{E[Y^0_k|Pre] - E[Y^0_k|Pre]} 
\end{eqnarray*}

\end{frame}

\begin{frame}{No Anticipation Violation}

If the baseline period is treated, then the simple 2x2 identifies the following three terms:

\begin{eqnarray*}
\delta &=& ATT_k(Post) \\&&+ \text{Non PT bias} \\&&- \textcolor{blue}{ATT_k(Pre)}
\end{eqnarray*}

First row is the ATT in the post period; middle row is parallel trends; third row subtracts the baseline ATT from the calculation. If treatment effects are constant, then the DiD coefficient will be zero despite positive treatment effects.  Let's look in \texttt{na.do}.

\end{frame}

\begin{frame}{Do not use already treated controls}
Using already treated units as controls is bad practice and here's why:

\begin{eqnarray*}
\widehat{\delta} &=& \bigg ( E[Y_k|Post] - E[Y_k|Pre] \bigg ) - \bigg ( E[Y_U | Post ] - E[ Y_U | Pre] \bigg) \\
\end{eqnarray*}What if the $U$ group had always been treated in both periods? Is parallel trends enough to identify the ATT?

\bigskip

Add in \textbf{three zeroes} instead of one, substitute and rearrange.

\begin{eqnarray*}
&+& \textcolor{red}{E[Y^0_k|Post] - E[Y^0_k|Post]} \\
&+& \textcolor{red}{E[Y^0_U|Post] - E[Y^0_U|Post]}  \\
&+& \textcolor{red}{E[Y^0_U|Pre] - E[Y^0_U|Pre]} 
\end{eqnarray*}

\end{frame}

\begin{frame}{Already Treated Control Group}

If the baseline period is treated, then the simple 2x2 identifies the sum of the following three terms:

\begin{eqnarray*}
\delta &=& ATT_k(Post) \\&&+ \text{Non PT bias} \\&&- \textcolor{red}{\Delta ATT_U}
\end{eqnarray*}

Again, first row is the target parameter, plus parallel trends term, \textcolor{red}{minus} the changing ATT in our control group

\end{frame}

\subsection{Clearly Defined Control Status}

\begin{frame}{Clearly Defined Control Status}

Look again at the result when you have NA and no treated control group:

\begin{eqnarray*}
\widehat{\delta} &=& \underbrace{E[Y^1_k | Post] - \textcolor{red}{E[Y^0_k | Post]}}_{\mathclap{\text{ATT}}} \\
&& + \bigg [  \underbrace{\textcolor{red}{E[Y^0_k | Post]} - E[\textcolor{blue}{Y^0_k} | Pre] \bigg ] - \bigg [ E[Y^0_U | Post] - E[\textcolor{blue}{Y_U^0} | Pre] }_{\mathclap{\text{Non-parallel trends bias in 2x2 case}}} \bigg ]
\end{eqnarray*}

What does the zero mean exactly?  At baseline, before $k$ was treated, it was in the $Y^0$ state of the world -- that wasn't "not treated", it was rather a different treatment status.  What was it?  Whatever it was, group $U$ had the same one

\end{frame}

\begin{frame}{Clearly Defined Control Status}

\begin{itemize}
\item Researcher wants to know the effect of \emph{recreational marijuana} on opiate addiction
\item Write down the treatment effect $$\delta_i = Y^1_i - \textcolor{blue}{Y^0_i}$$
\item I know what $Y^1$ is -- that's a person exposed recreational cannabis regime; but what is $\textcolor{blue}{Y^0_i}$?

\end{itemize}

\end{frame}

\begin{frame}{Clearly Defined Control Status}

\begin{itemize}
\item Texas prohibits all cannabis -- last week passed a bill even reversing a 2018 Farm Bill that had inadvertently legalized hemp-based THC
\item But Oklahoma has medical marijuana, which is neither recreational marijuana nor is it prohibited cannabis use -- it's a \emph{third treatment regime}
\item Look closely again though at the form of the $2 \times 2$ identification and assumptions
\end{itemize}


\end{frame}

\begin{frame}{Clearly Defined Control Status}

\begin{eqnarray*}
\widehat{\delta} &=& \underbrace{E[Y^1_k | Post] - \textcolor{red}{E[Y^0_k | Post]}}_{\mathclap{\text{ATT}}} \\
&& + \bigg [  \underbrace{\textcolor{red}{E[Y^0_k | Post]} - E[\textcolor{blue}{Y^0_k} | Pre] \bigg ] - \bigg [ E[Y^0_U | Post] - E[\textcolor{blue}{Y_U^0} | Pre] }_{\mathclap{\text{Non-parallel trends bias in 2x2 case}}} \bigg ]
\end{eqnarray*}

\begin{itemize}
\item Diff-in-diff has a limitation that not all the other designs suffer from -- the control group \emph{must} have the same control status as the treatment group at baseline
\item No state in the US has ever gone from "prohibition" to "recreational" marijuana
\item Which means the best you can do with diff-in-diff is identify the effect of recreational marijuana compared to medical marijuana since every US state transitioned in that order
\end{itemize}

\end{frame}

\begin{frame}{Target Parameter}

\begin{itemize}
\item Diff-in-diff can \emph{only} in the United States identify the "relative to medical marijuana" because recreational cannabis states have medical marijuana at baseline -- so control groups must too (not prohibition states)
\item $\delta_i=Y^1_i - Y^0_i$ might be the causal effect of recreational marijuana on opiate addiction when $Y^0_i$ is "prohibition" or it may be "medical marijuana"
\item Either is fine -- but they are not the same $Y^0$, is my point, and therefore they are not the same parameter
\end{itemize}

\end{frame}



\begin{frame}{Summarizing the basics}

\begin{itemize}
\item With two time periods (before and after treatment), and two groups (one treated and one not), then the $2 \times 2$ identifies the ATT if parallel trends and no anticipation holds so long as you do not have an already treated group as a control
\item Let's now look at some basic data requirements and what happens when you don't meet those basic things

\end{itemize}

\end{frame}

\subsection{Data Requirements}

\begin{frame}{Longitudinal Data}

\begin{itemize}

\item Diff-in-diff requires four means -- pre and post for two groups
\item Traditionally, the "pre" is a baseline mean at year just prior to treatment, $t-1$ or $b$ depending on author
	\begin{itemize}
	\item Though sometimes you will see people present any interaction as diff-in-diff I will stick to time
	\item Just remember the interaction regression we presented is calculating four means and three subtractions
	\item Interpreting parallel trends is a bit stranger otherwise
	\end{itemize}
\end{itemize}

\end{frame}

\begin{frame}{Longitudinal Data}

\begin{itemize}

\item Two types of longitudinal data:
    \begin{itemize}
      \item Panel Data: same units tracked over time (e.g., National Longitudinal Survey of Youth 1997)
      \item Repeated Cross-Sections: different units sampled at each time (e.g., Census, Current Population Survey)
    \end{itemize}
    \item Violations of parallel trends can arise differently across data types.
\end{itemize}

\end{frame}


% Slide 2
\begin{frame}{What is an Imbalanced Panel?}
  \begin{itemize}
    \item Balanced panel is when all units observed in every period.
    \item Imbalanced panel is when units missing in some periods.
		\begin{itemize}
		\item Anthony is in waves 1-3, 
		\item Bob is in waves 1-3, 
		\item Inez is in waves 1 and 3 only, 
		\item Dignan is in waves 1 and 2 only
		\end{itemize}
    \item Missingness alone does not violate parallel trends, though it does change the parameter.
  \end{itemize}
\end{frame}

\begin{frame}{Example of Balanced ATE}
  \centering
  \includegraphics[height=0.85\textheight,keepaspectratio]{./lecture_includes/balanced.png}
\end{frame}


\begin{frame}{How does it change the parameter?}

\begin{itemize}
\item Remember -- in the potential outcomes framework, a treatment effect is defined at the individual level, $\delta_{it}$
\item So if you are missing a person, $i$,  in a period, $t$, then it does not contribute
\item The more heterogeneity in the treatment effects, the more the broken panel will shift away from what you think you're after
\end{itemize}

\end{frame}


\begin{frame}{Imbalanced ATE}
  \centering
  \includegraphics[height=0.85\textheight,keepaspectratio]{./lecture_includes/imbalanced.png}
\end{frame}

% Slide 3
\begin{frame}{Missing at Random (MAR)}
  \begin{itemize}
    \item \textbf{Wooldridge (2010):} MAR does not bias estimation under constant treatment effects.
    \item If missingness is independent of $E[\Delta Y^0]$, then parallel trends will still hold in the large sample.
    \item Wooldridge calls this \emph{Missing At Random (MAR)}
    \item But maybe this is not plausible always so is there anything else we can do?
  \end{itemize}
\end{frame}


% Slide 5
\begin{frame}{Conditional Missing at Random}
  \begin{itemize}
	\item What if the missing is conditionally random $$Y^0  \perp M \mid X$$
    \item This implies:
    $$E[Y^0|M=1,X] = E[Y^0|M=0,X]$$
	\item So you can impute $\widehat{Y^0}$ for missing units using $Y^0 \sim X$ with your non-missing data with a regression estimator
	\item More needs to be done to think about the $Y^1$ imputation so I'll just focus on this for now
  \end{itemize}
\end{frame}

\begin{frame}{Conditional Missing at Random}

\begin{itemize}
    \item Note though -- this is a "missing based on observables" type of unconfoundedness assumption
    \item I haven't worked this out yet, but you'll need I'm sure an additional overlap assumption or some functional form assumption
    \item You might explore whether there is something here for you, as it was an idea I had but I haven't worked it out

\end{itemize}

\end{frame}



\section{Weighting and Target Parameters}



\begin{frame}{Choosing Target Parameter}

\begin{itemize}
\item Causal parameters are \emph{averaged treatment effects} and that has two elements
	\begin{itemize}
	\item What population's treatment effects are you averaging
	\item What weight are you using to do that averaging?
	\end{itemize}
\item I'll use as an example a gun law in the United States called "concealed carry" or "right to carry" that lets people carry weapons "concealed" on their bodies or in their vehicles 
\end{itemize}

\end{frame}


\begin{frame}{Average Treatment Effects}

\begin{table}[htbp]\centering
\caption{Illustration of Potential Outcomes and Treatment Effects}\label{tab:step1_table}
\begin{tabular}{lccc|c}
\toprule
\textbf{Name} & \textbf{$Y^1$} & \textbf{$Y^0$} & \textbf{$\delta$} & \textbf{D}  \\
\midrule
Alan    & 1 & 0 & 1  & 1  \\
Betty   & 0 & 1 & -1 & 1  \\
Chad    & 1 & 1 & 0  & 1  \\
Daniel  & 0 & 0 & 0  & 0  \\
Edith   & 1 & 0 & 1  & 0  \\
Frank   & 1 & 0 & 1  & 0  \\
\midrule
$ATE =$&&& 0.33 \\
$ATT =$&&&0 \\
\bottomrule
\end{tabular}
\end{table}  

\end{frame}

\begin{frame}{Average Treatment Effects}

\begin{figure}
    \centering
    \includegraphics[height=0.8\textheight]{./lecture_includes/step1_y1y0}
\end{figure}


\end{frame}

\begin{frame}{Average Treatment Effect on the Treated}

\begin{figure}
    \centering
    \includegraphics[height=0.8\textheight]{./lecture_includes/step1_att}
\end{figure}


\end{frame}



\begin{frame}{Which causal effect do you want?}

\begin{itemize}
\item We know diff-in-diff identifies the ATT not the ATE but there are still more than one ATT!
\item Solon, Haider and Wooldridge (2015), "What are we weighting for?"
\item Why you weight in surveys and why you weight in causal inference are different
	\begin{itemize}
	\item Survey weights are to make estimates nationally representative
	\item Population weighting in causal inference is because you want a different parameter
	\end{itemize}
\item How do we interpret adjustments made with population weights versus not?
\end{itemize}

\end{frame}




\begin{frame}{Different Levels of Aggregation, Different Weights}

\begin{table}[htbp]\centering
\footnotesize
\begin{tabular}{lcccc}
\toprule
\textbf{Individuals} & \textbf{Y\textsubscript{1}} & \textbf{Y\textsubscript{0}} & \textbf{$\delta$} & \textbf{County} \\
\midrule
Alan    & 1 & 0 & 1  & \cellcolor{yellow!30}1 \\
Betty   & 1 & 1 & 0  & \cellcolor{yellow!30}1 \\
Chad    & 1 & 1 & 0  & \cellcolor{yellow!30}1 \\
Daniel  & 0 & 0 & 0  & \cellcolor{yellow!30}1 \\
Edith   & 1 & 0 & 1  & \cellcolor{yellow!30}1 \\
Frank   & 1 & 0 & 1  & \cellcolor{yellow!30}1 \\
George  & 0 & 0 & 0  & \cellcolor{yellow!30}1 \\
Hank    & 1 & 0 & 1  & \cellcolor{yellow!30}1 \\
Ida     & 0 & 1 & -1 & \cellcolor{green!20}2 \\
Janet   & 0 & 1 & -1 & \cellcolor{green!20}2 \\
\midrule
\textbf{County} & & & \textbf{ATE\textsubscript{c}} & \\
\cellcolor{yellow!30}1 & & & \cellcolor{yellow!30}0.5 & \\
\cellcolor{green!20}2 & & & \cellcolor{green!20}-1 & \\
\midrule
\multicolumn{3}{l}{ATE for average county} & -0.25 & \\
\multicolumn{3}{l}{ATE for average person}   & 0.2 & \\
\bottomrule
\end{tabular}
\end{table}


\end{frame}

\begin{frame}{Weighting formula}

\begin{itemize}
\item What if you have city or county level data but you want the ATE for the average person? 
\item Then that is when you weight by population -- because you want to know the effect for the \emph{average person}
\end{itemize}

\begin{equation}
ATE_{\text{people}} = \frac{\sum_c ATE_c \cdot N_c}{\sum_c N_c}
\end{equation}

\begin{eqnarray*}
ATE_p &=& \frac{(0.5 \times 8) + (-1 \times 2)}{8 + 2} \\
      &=& \frac{4 - 2}{10} \\
      &=& 0.2
\end{eqnarray*}

\end{frame}




\begin{frame}{Different Levels of Aggregation, Different Weights}


\begin{itemize}
\item Heterogenous treatment effects is causing this, but so is weighting
\item Consider Texas
	\begin{itemize}
	\item Texas has 31 million residents
	\item Texas 254 counties
	\end{itemize}
\item Where do they live?
	\begin{itemize}
	\item 13 million live in Harris, Dallas, Fort Worth, San Antonio and Austin, or rather 41\% 
	\end{itemize}
\item What if concealed carry increases firearm deaths in cities, but reduces them in counties, because of sorting by treatment effects?
\end{itemize}

\end{frame}

\begin{frame}{Simulation}

\begin{itemize}
\item Assume a state with 30 million people and 254 counties
	\begin{itemize}
	\item 15 million live in 5 counties
	\item 15 million live equally spread in the other 249 counties (around 60,000 each)
	\end{itemize}
\item Assume that $\delta_i$ varies, sometimes positive and sometimes negative and $E[Y^1 - Y^0] = 2$
\end{itemize}

\end{frame}

\begin{frame}{Selection into Counties is Random}

\begin{itemize}
\item People choose where to live in Texas, but the mechanism by which they do so will have implications for our datasets
\item What if they sort into counties (i.e., where they live) by lottery
\item Every county will therefore have the same distribution of $Y^1$ and $Y^0$
\item Every county will have an ATE of $2$
\end{itemize}

\end{frame}



\begin{frame}{Average County ATE and Overall ATE Are the Same }

\begin{figure}
    \centering
    \includegraphics[height=0.7\textheight]{./lecture_includes/tiebout_roy1}
\end{figure}

\end{frame}

\begin{frame}{Selection into Counties is Based on Potential Outcomes}

\begin{itemize}
\item Now assume that people sort into five largest counties have the highest treatment effects ($\delta_i$)
\item Thus the five largest counties are those people for whom concealed carry guns law cause homicides
\item Remaining 249 small counties, in decreasing order, get residents with lower treatment effects (i.e., for whom concealed carry reduces murders)
\end{itemize}

\end{frame}

\begin{frame}{Average County ATE and Overall ATE Differ }

\begin{figure}
    \centering
    \includegraphics[height=0.7\textheight]{./lecture_includes/tiebout_roy2}
\end{figure}

\end{frame}




\begin{frame}{Both Are Valid, But Not Necessarily Your Research Question}

\begin{itemize}
\item This has implications for what to learn from your dataset which is likely to be aggregated at some level (e.g., individual, city, county, state, country)
\item Do you want to know the ATT for the \emph{average person} or the \emph{average county}?
	\begin{itemize}
	\item It depends on what your study is about
	\item If it about the average person, then you want the overall ATT (i.e., the first case)
	\item If it is about the average county, then you want the county average ATT (i.e., the second case)
	\end{itemize}
\item Since you're averaging over \emph{units} in \emph{data}, it's imperative you make a decision early on as it changes what you decide
\item You can always use population weights but in causal inference, you ask what your target parameter is, and then decide your weights
\end{itemize}

\end{frame}

\begin{frame}{Choosing your ATT parameter}

\begin{itemize}

\item This means that even diff-in-diff has more than one ATT -- the \emph{average county} or \emph{average person}
\item As a rule, I think your goal is probably to imagine who your audience is, and what you both agree the policy levers are
\item In the US, local municipalities have a lot of discretion to pass their own laws -- even in areas where you might think it was impossible to imagine like the decriminalization of drugs
\item It may be \emph{local} policy makers want to know the causal effect on \emph{local communities} in which case you \textbf{don't weight}

\end{itemize}

\end{frame}

\begin{frame}{Choosing your ATT parameter}

\begin{itemize}
\item But the argument isn't to weight to make it nationally representative -- it's "is the average effect on the average county theoretically important, policy relevant and how one is planning to use it
\item Others are also possible
\item Quantile treatment effects (Athey and Imbens 2006; Callaway and Li 2019) and distribution regression target features of the marginal distributions of $Y^1_{i,t}$ and $Y^0_{i,t}$(Fernandez-Val, Meier, van Vuuren and Vella 2024a)
\end{itemize}

\end{frame}

\section{Four Means or One Regression}




\begin{frame}{Why do Diff-in-Diff}

\begin{itemize}
\item Appeal of diff-in-diff has been its simplicity, its transparency, and its ease of conveying analysis to an audience 
	\begin{itemize}
	\item Orley Ashenfelter used it in the 1970s to explain regressions with fixed effects to Bureaucrats in DC 
	\end{itemize}
\item Diff-in-diff is four averages and three subtractions and everyone knows what those are
$$\widehat{\textcolor{blue}{\delta}} = \bigg ( \overline{y}_k^{post(k)} - \overline{y}_k^{pre(k)} \bigg ) - \bigg ( \overline{y}_U^{post(k)} - \overline{y}_U^{pre(k)} \bigg ) $$
\item But $\widehat{\delta}$ is just the OLS coefficient in this regression:
$$Y_{ist} = \alpha_0 + \alpha_1 Treat_{is} + \alpha_2 Post_{t} + \textcolor{blue}{\delta} (Treat_{is} \times Post_t) + \varepsilon_{ist} $$
\end{itemize}

\end{frame}

\begin{frame}{Minimum wages}

\begin{itemize}
\item Card and Krueger (1994) have a famous study estimating causal effect of minimum wages on employment
\item  New Jersey raises its minimum wage in April 1992 (between February and November) but neighboring Pennsylvania does not
\item Using diff-in-diff, they do not find a negative effect of the minimum wage on employment leading to complex reactions from economists
\end{itemize}

\end{frame}





\begin{frame}{OLS specification of the DiD equation}
	
	\begin{itemize}
	\item The correctly specified OLS regression is an interaction with time and group fixed effects:$$Y_{its} = \alpha + \gamma NJ_s + \lambda d_t + \delta (NJ \times d)_{st} + \varepsilon_{its}$$
		\begin{itemize}
		\item NJ is a dummy equal to 1 if the observation is from NJ
		\item d is a dummy equal to 1 if the observation is from November (the post period)
		\end{itemize}
	\item This equation takes the following values
		\begin{itemize}
		\item PA Pre: $\alpha$
		\item PA Post: $\alpha + \lambda$
		\item NJ Pre: $\alpha + \gamma$
		\item NJ Post: $\alpha + \gamma + \lambda + \delta$
		\end{itemize}
	\item DiD equation: (NJ Post - NJ Pre) - (PA Post - PA Pre) $= \delta$
	\end{itemize}
\end{frame}




\begin{frame}[plain]
	$$Y_{ist} = \alpha + \gamma NJ_s + \lambda d_t + \delta(NJ\times d)_{st} + \varepsilon_{ist}$$
	\begin{figure}
	\includegraphics[scale=0.90]{./lecture_includes/waldinger_dd_5.pdf}
	\end{figure}
\end{frame}


\begin{frame}[plain]
	$$Y_{ist} = \alpha + \gamma NJ_s + \lambda d_t + \delta(NJ\times d)_{st} + \varepsilon_{ist}$$
	\begin{figure}
	\includegraphics[scale=0.90]{./lecture_includes/waldinger_dd_5.pdf}
	\end{figure}

Notice how OLS is ``imputing'' $E[Y^0|D=1,Post]$ for the treatment group in the post period? It is only ``correct'', though, if parallel trends is a good approximation

\end{frame}


\begin{frame}{Data example}

\begin{itemize}
\item Medicaid is a US policy for the poor -- provides access to healthcare to the poor -- and it's been a research question whether it has any effect on mortality
	\begin{itemize}
	\item Finkelstein, et al. (2012) found no effect on mortality from RCT, but it was probably underpowered to find small effects
	\item Borgschulte and Vogler (2020) used county level publicly available mortality data and found evidence that it did
	\item Miller, Johnson and Wherry (2022) used linked administrative data and also found it did
	\end{itemize}
\item Let's review the Borgschulte and Vogler (2020) approach

\end{itemize}

\end{frame}

\begin{frame}{Simple 2x2 DD}

\begin{figure}
    \centering
    \includegraphics[height=0.5\textheight]{./lecture_includes/simple2x2.png}
\end{figure}


\end{frame}

\begin{frame}{Three Regressions}

\begin{itemize}
\item Three regression specifications give you those exact same numbers
	\begin{enumerate}
	\item Regress mortality onto treatment dummy, post dummy and interaction (no fixed effects)
	\item Regress mortality onto interaction with county and year fixed effects (but no constant)
	\item Regress long difference (i.e., post value minus pre) onto a treatment dummy
	\end{enumerate}
\item Those are all numerically identical to "four averages and three subtractions"
\end{itemize}

\end{frame}

\begin{frame}

\begin{figure}
    \centering
    \includegraphics[height=0.7\textheight]{./lecture_includes/regression2x2.png}
\end{figure}

\end{frame}

\begin{frame}{Equivalence}

\begin{itemize}

\item Equivalence between calculating these 2x2s by hand or with a regression has appealing features
\item Regressions are simple to run, and they do the averaging behind the scenes
\item They also allow us to use statistical inference tools from OLS like clustering decisions
\item  Bertrand, Duflo and Mullainathan (2004) show that conventional standard errors will often severely understate the standard deviation of the estimators and propose clustering standard errors at the aggregate unit level of treatment
\end{itemize}

\end{frame}



\end{document}


